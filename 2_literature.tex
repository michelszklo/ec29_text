\section{Related Literature}\label{sec:inst}
%\pagenumbering{arabic}
\setstretch{1.5}

This study relates to literature on the impact of public health expenditure on mortality. Cross-country studies usually confirms this relationship, but the magnitude of effects often varies, with some studies presenting statistically insignificant effects . \cite{filmer1999} uses an instrumental variable approach on a global panel data to find no significant impacts on infant and child mortality. \cite{bokhari2007}, using a similar approach on a cross-section for the year 2000, find small but significant effects on child and maternal mortality. More recently, \cite{moreno2015} find very similar effects, reinforcing this evidence. \cite{nixon2006}, using a 15 year panel data for 15 European Union members, finds that increases in health spending are associated with large reduction in infant mortality. \cite{gupta2002effectiveness} analyse this relationship for 50 developing and transition countries and find effects on infant and child mortality that are not very robust to different specifications. Working with larger and richer data set of developing and transition countries, \cite{Gupta2003} estimates suggests that the  effects of health spending on infant and child mortality are twice as large among the poor. Notwithstanding, a recent review of cross-country studies suggests that, in general, these cross-country results are very sensitive to robustness checks \citep{Nakamura2020}. 

The identification issues that the cross-country studies face are usually address by the micro-level literature. \cite{cremieux1999} findings suggest that increases in health expenditure are associated with decreases in the infant mortality rate and increases in life expectancy, on a panel of data for Canadian provinces. \cite{sonia2007}, working with a rich panel data at the individual level in India, presents probit models estimates of the impact of health expenditures on the risk of infant mortality that suggest no significant contemporaneous effect, but long term and small effects for rural residents. 

There is some relevant recent work for Brazil \citep{castro2021effects}. Using a panel of small municipalities for the period of 2002-2012 and a regression discontinuity design approach, this study finds large and significant effects of health spending on infant mortality, with elasticities ranging from $-0.5$ to $-0.9$. Moreover, they show that health spending presents strong spatial externalities, with the population of neighboring municipalities also benefiting from increases in health spending.

Additionally, descriptive evidence for Brazil suggests that increases in health spending are associated with increases in primary care coverage and the number of mothers attending seven or more prenatal visits, and with decreases in infant mortality rates, especially for the poorer municipalities \citep{paixao2012,Castro2019}. 

It is hard to imagine scenarios in which increasing spending would not lead to improvements in outcomes, but the overall evidence on the impacts of health spending on health outcomes is quite mixed. That could be the case of weak links in the chain connecting health spending to outcomes \citep{filmer2000weak}. Is money being spent efficiently? How is spending translating into health inputs? How these inputs relate to outcomes? \cite{Rajkumar2008} is to our knowledge one of the first articles that attempts to study these links. The article provides evidence on the importance of governance\footnote{The level of governance is measured by the level of corruption and quality of the bureaucracy.} for this link, and suggest that public health spending has stronger effects in countries that have good governance. Furthermore, in countries with ineffective bureaucracy or rated as very corrupt, at the margin, public health spending will be ineffective.