\section{Empirical Findings}\label{sec:results}
\setstretch{1.5}

The goal of this section is to understand the impacts of health spending on health outcomes, and the pathways through which the impact take place. For that, we first present the estimates of the impact of EC/29 on fiscal and spending outcomes. Later, we analyse how health expenditure increases translate into health inputs. Lastly, we examine the impacts on infant mortality rates and birth outcomes. All outcomes were analyzed as rates and that is how effects are presented in our regression tables and graphs. However, in our discussion of results, we will focus on the percent variation relative to baseline means of a representative municipality with a distance of 10\% to the EC/29 target of the share of own resource spent in health. This distance is equivalent to the distance to the target of the municipalities in bottom quartile of the distribution of the share of own resource spent in health, which is the group of municipalities that presented the most pronounced increase in health spending after the EC/29 was enacted.


\subsection{Municipalities' Fiscal Response to the EC/29}\label{sec:results_fiscal}

Table \ref{table:fiscal} shows the estimates for total public revenue and spending, public spending by category, and public health spending, total, by source and type. In column 1 we present our baseline estimates, a continuous DiD with municipality and state-year fixed effects. Column 2 adds to the baseline specification a set of baseline controls interacted with time. Column 3 adds socioeconomic time varying controls, and column 4, our preferred specification and most saturated, adds time varying fiscal controls. 

Panel A shows that the EC/29 had no significant impact on total revenues and total spending per capita. Though the points estimates are positive, Finbra data, specially for the pre-reform years, is in general quite noisy\footnote{Appendix Figure \ref{fig:b1} plots the dynamic effects for these outcomes estimated with Equation \ref{eq:2}.}. Next, we look at public health spending by category (Panel B). The only category that has been significantly impacted by the amendment is Health and Sanitation spending per capita, and the results are quite robust across different specifications. 

In our preferred specification (column 4), the estimate of around 300 suggests a increase of R\$30 in health spending per capita for our representative municipality, equivalent to a increase of around 14\% relative to the baseline health and sanitation spending per capita (see Table \ref{table:stats}). This distance is roughly the distance to the target of the bottom quartile of the distribution of the share of own resources spent in health in the baseline.

As discussed in Section \ref{sec:emp}, the validation of our research design relies partially on evaluating the presence of pre-trends. Even though SIOPS is a much better data source to study health spending, it is only available after the year 2000. Therefore, we will use Finbra data mainly to evaluate the presence of pre-trends in health spending and the move to analyse health spending and resource allocation within the public health sector using SIOPS data. Figure \ref{fig:6a} plots the dynamic effects estimated with Equation \ref{eq:2} for the equivalent of the specifications presented in Column 1 and 4, for Health and Sanitation per capita. We find no pre-trends and a clear and significant pattern of increase in spending, with each of the first years after the EC/29 presenting stronger effects, that stabilize after 2004. Appendix Figure \ref{fig:b2} plots the dynamic effects for all other categories of spending. Estimates are very imprecise for almost all categories and it is hard to extract much information. But in general, there seem to be no pre-trend, nor significant effects on other categories of spending besides Health and Sanitation. These results are extremely relevant because it will allow us to claim that any reductions we find in Infant Mortality rates are most certainly associated with increases in health expenditure and not increases in spending from other categories that could also affect mortality, such as social assistance and education. 


\begin{table}[h!]
\begin{footnotesize}
\begin{center}
\scalebox{0.8}{
\begin{threeparttable}[b]

  \centering
  \caption{Fiscal Reactions}
     \begin{tabular}{rrrrrr}
          &       &       &       &       &  \\
          &       &       &       &       &  \\
    \midrule
    \midrule
          &       & \multicolumn{1}{c}{(1)} & \multicolumn{1}{c}{(2)} & \multicolumn{1}{c}{(3)} & \multicolumn{1}{c}{(4)} \\
    \midrule
    \multicolumn{1}{p{23.285em}}{\textbf{A. Public Revenue and Spending per capita (Finbra)}} &       &       &       &       &  \\
    \multicolumn{1}{p{23.285em}}{Total Revenue} &       & \multicolumn{1}{c}{841.224} & \multicolumn{1}{c}{868.151} & \multicolumn{1}{c}{911.342} & \multicolumn{1}{c}{929.681} \\
          &       & \multicolumn{1}{c}{(1248.728)} & \multicolumn{1}{c}{(1264.873)} & \multicolumn{1}{c}{(1262.793)} & \multicolumn{1}{c}{(1264.449)} \\
    \multicolumn{1}{p{23.285em}}{Total Spending} &       & \multicolumn{1}{c}{1089.977} & \multicolumn{1}{c}{1116.507} & \multicolumn{1}{c}{1153.877} & \multicolumn{1}{c}{1120.384} \\
          &       & \multicolumn{1}{c}{(1450.455)} & \multicolumn{1}{c}{(1468.068)} & \multicolumn{1}{c}{(1466.367)} & \multicolumn{1}{c}{(1467.661)} \\
          &       &       &       &       &  \\
    \midrule
    \multicolumn{1}{p{23.285em}}{\textbf{B. Public Spending By Category (Finbra)}} &       &       &       &       &  \\
    \multicolumn{1}{p{23.285em}}{Health and Sanitation Spending} &       & \multicolumn{1}{c}{302.751***} & \multicolumn{1}{c}{307.022***} & \multicolumn{1}{c}{314.312***} & \multicolumn{1}{c}{308.104***} \\
          &       & \multicolumn{1}{c}{(94.499)} & \multicolumn{1}{c}{(95.853)} & \multicolumn{1}{c}{(94.851)} & \multicolumn{1}{c}{(94.901)} \\
    \multicolumn{1}{p{23.285em}}{Transport Spending} &       & \multicolumn{1}{c}{53.404} & \multicolumn{1}{c}{55.428} & \multicolumn{1}{c}{57.487} & \multicolumn{1}{c}{58.332} \\
          &       & \multicolumn{1}{c}{(64.588)} & \multicolumn{1}{c}{(65.717)} & \multicolumn{1}{c}{(65.673)} & \multicolumn{1}{c}{(65.738)} \\
    \multicolumn{1}{p{23.285em}}{Education and Culture Spending} &       & \multicolumn{1}{c}{181.786} & \multicolumn{1}{c}{193.281} & \multicolumn{1}{c}{203.347} & \multicolumn{1}{c}{195.657} \\
          &       & \multicolumn{1}{c}{(391.088)} & \multicolumn{1}{c}{(396.875)} & \multicolumn{1}{c}{(396.579)} & \multicolumn{1}{c}{(396.904)} \\
    \multicolumn{1}{p{23.285em}}{Housing and Urban Spending} &       & \multicolumn{1}{c}{106.441} & \multicolumn{1}{c}{103.719} & \multicolumn{1}{c}{107.997} & \multicolumn{1}{c}{105.891} \\
          &       & \multicolumn{1}{c}{(151.807)} & \multicolumn{1}{c}{(153.609)} & \multicolumn{1}{c}{(153.292)} & \multicolumn{1}{c}{(153.428)} \\
    \multicolumn{1}{p{23.285em}}{Social Assistance Spending per capita} &       & \multicolumn{1}{c}{189.327} & \multicolumn{1}{c}{197.649} & \multicolumn{1}{c}{200.966} & \multicolumn{1}{c}{200.701} \\
          &       & \multicolumn{1}{c}{(251.479)} & \multicolumn{1}{c}{(254.819)} & \multicolumn{1}{c}{(254.729)} & \multicolumn{1}{c}{(254.927)} \\
    \multicolumn{1}{p{23.285em}}{Spending in Other Categories per capita} &       & \multicolumn{1}{c}{362.594} & \multicolumn{1}{c}{365.701} & \multicolumn{1}{c}{381.213} & \multicolumn{1}{c}{365.486} \\
          &       & \multicolumn{1}{c}{(668.655)} & \multicolumn{1}{c}{(676.98)} & \multicolumn{1}{c}{(676.072)} & \multicolumn{1}{c}{(676.672)} \\
          &       &       &       &       &  \\
    \midrule
    \multicolumn{1}{p{23.285em}}{\textbf{C. Public Health Spending (SIOPS)}} &       &       &       &       &  \\
    \multicolumn{1}{p{23.285em}}{Total} &       & \multicolumn{1}{c}{529.375***} & \multicolumn{1}{c}{530.301***} & \multicolumn{1}{c}{530.936***} & \multicolumn{1}{c}{530.317***} \\
          &       & \multicolumn{1}{c}{(18.16)} & \multicolumn{1}{c}{(17.876)} & \multicolumn{1}{c}{(17.507)} & \multicolumn{1}{c}{(17.485)} \\
    \multicolumn{1}{p{23.285em}}{\textbf{By Source}} &       &       &       &       &  \\
    \multicolumn{1}{p{23.285em}}{Own Resources} &       & \multicolumn{1}{c}{580.644***} & \multicolumn{1}{c}{581.011***} & \multicolumn{1}{c}{581.215***} & \multicolumn{1}{c}{580.792***} \\
          &       & \multicolumn{1}{c}{(13.943)} & \multicolumn{1}{c}{(13.725)} & \multicolumn{1}{c}{(13.421)} & \multicolumn{1}{c}{(13.431)} \\
    \multicolumn{1}{p{23.285em}}{Transfers} &       & \multicolumn{1}{c}{-53.096***} & \multicolumn{1}{c}{-52.482***} & \multicolumn{1}{c}{-51.826***} & \multicolumn{1}{c}{-52.036***} \\
          &       & \multicolumn{1}{c}{(11.269)} & \multicolumn{1}{c}{(11.157)} & \multicolumn{1}{c}{(11.123)} & \multicolumn{1}{c}{(11.107)} \\
    \multicolumn{1}{p{23.285em}}{\textbf{By Type}} &       &       &       &       &  \\
    \multicolumn{1}{p{23.285em}}{Human Resources} &       & \multicolumn{1}{c}{96.997***} & \multicolumn{1}{c}{94.917***} & \multicolumn{1}{c}{95.188***} & \multicolumn{1}{c}{93.164***} \\
          &       & \multicolumn{1}{c}{(11.202)} & \multicolumn{1}{c}{(11.12)} & \multicolumn{1}{c}{(11)} & \multicolumn{1}{c}{(10.914)} \\
    \multicolumn{1}{p{23.285em}}{Investiment} &       & \multicolumn{1}{c}{132.947***} & \multicolumn{1}{c}{133.38***} & \multicolumn{1}{c}{133.42***} & \multicolumn{1}{c}{133.64***} \\
          &       & \multicolumn{1}{c}{(9.654)} & \multicolumn{1}{c}{(9.683)} & \multicolumn{1}{c}{(9.667)} & \multicolumn{1}{c}{(9.672)} \\
    \multicolumn{1}{p{23.285em}}{3rd parties services} &       & \multicolumn{1}{c}{55.911***} & \multicolumn{1}{c}{54.863***} & \multicolumn{1}{c}{54.756***} & \multicolumn{1}{c}{55.165***} \\
          &       & \multicolumn{1}{c}{(11.687)} & \multicolumn{1}{c}{(11.545)} & \multicolumn{1}{c}{(11.471)} & \multicolumn{1}{c}{(11.49)} \\
    \multicolumn{1}{p{23.285em}}{Other Expenditures} &       & \multicolumn{1}{c}{247.246***} & \multicolumn{1}{c}{250.325***} & \multicolumn{1}{c}{250.742***} & \multicolumn{1}{c}{251.5***} \\
          &       & \multicolumn{1}{c}{(11.365)} & \multicolumn{1}{c}{(11.417)} & \multicolumn{1}{c}{(11.403)} & \multicolumn{1}{c}{(11.37)} \\
          &       &       &       &       &  \\
    \bottomrule
    \bottomrule
    \end{tabular}%
    
    
  \label{table:fiscal}%
  
  \begin{tablenotes}
  \scriptsize{\underline{Notes}: The number of observations is 64470 for Finbra variables and 55810 for SIOPS variables.  DiD Estimates from Equation \ref{eq:1}. Independent variable is the distance to the EC/29 target in p.p. Column 1 presents the baseline model with municipality and state-year fixed effects. Column 2 adds baseline socioeconomic controls from the Census interacted with time. Column 3 adds controls for GDP per capita and \emph{Bolsa Familia} transfers per capita. Column 4 adds fiscal controls. Covariates omitted. Standard errors in brackets are clustered in the municipality level. ∗p < 0.10, ∗ ∗ p < 0.05, ∗ ∗ ∗p < 0.01}
  \end{tablenotes}

\end{threeparttable}
}
\end{center}
\end{footnotesize}
\end{table}

\begin{figure}[h!]
    \begin{center}
    \caption{Fiscal Reactions}\label{fig:6}
    \begin{subfigure}{0.49\textwidth}
        \caption{\scriptsize Total Revenue}\label{fig:6a}
        \centering
        \includegraphics[width=\textwidth]{plots/finbra_reccorr_pcapita_dist_ec29_baseline_dist_ec29_baseline_6.pdf}
    \end{subfigure}
    \begin{subfigure}{0.49\textwidth}
        \centering
        \caption{\scriptsize Total Public Spending}\label{fig:6b}
        \includegraphics[width=\textwidth]{plots/finbra_desp_o_pcapita_dist_ec29_baseline_dist_ec29_baseline_6.pdf}
    \end{subfigure}
    
    \end{center}
    
\end{figure}

Panel C in Table \ref{table:fiscal} presents the results for total health spending, health spending by source and health spending by type. Our estimations suggests an effect of R\$ 530 for total Health Spending per capita, which is equivalent to a 27\% increase in spending relative to the baseline for our representative municipality, almost twice the effect on Health and Sanitation per capita. Additionally, this effect comes almost entirely from increases in spending from own resources, a 50\% increase relative to own resource spending in the baseline. We also find some substitution effects, with municipalities reducing some of its spending from intergovernmental transfers in health. All types of health spending were responsible for this increase in total health spending, but the increase in investment is the noteworthy, specially in relative terms. This estimate is associated with a 90\% increase in health investments. Baseline statistics show very little resources allocated in investments within total public health spending, the great majority of resources were allocated in human resources and in other administrative expenses. Considering the importance of capital investments to the supply of medical resources and the quality of medical services, and the little amount of investments in the baseline, an effect of this size is really relevant. Other expenditures, that includes mainly administrative spending, presents the strongest effect in per capita terms, almost half of the total increase in health spending per capita, equivalent to a 34\% increase relative to the baseline for the representative municipality. In opposition to the relevance investment in a health production function, administrative expenditure plays a much minor role in affecting health outcomes.

\begin{figure}[h!]
    \begin{center}
    \caption{Effects Public Spending per capita - By Type}\label{fig:7}
    \begin{subfigure}{0.32\textwidth}
        \centering
        \caption{\scriptsize Human Resources}\label{fig:7a}
        \includegraphics[width=\textwidth]{plots/finbra_desp_pessoal_pcapita_dist_ec29_baseline_dist_ec29_baseline_7.pdf}
    \end{subfigure}
    \begin{subfigure}{0.32\textwidth}
        \centering
        \caption{\scriptsize Investments}\label{fig:7b}
        \includegraphics[width=\textwidth]{plots/finbra_desp_investimento_pcapita_dist_ec29_baseline_dist_ec29_baseline_7.pdf}
    \end{subfigure}
    \begin{subfigure}{0.32\textwidth}
        \centering
        \caption{\scriptsize Other}\label{fig:7c}
        \includegraphics[width=\textwidth]{plots/finbra_desp_outros_nature_pcapita_dist_ec29_baseline_dist_ec29_baseline_7.pdf}
    \end{subfigure}
    
    \end{center}
    
\end{figure}

The clear pattern of increases in health and sanitation spending depicted in Figure \ref{fig:6a} can also be seen for SIOPS total health spending in Figure \ref{fig:6b}, but at different levels. Figure \ref{fig:7} suggests that this pattern is mostly influenced by dynamic of the effects on human resources spending (Figure \ref{fig:7a}). Investments, 3rd parties and other expenditures present a sharp increase in spending in the first one or two years after the EC/29 an than stabilize.


\subsection{Effects on Health Inputs}

In this subsection we aim to explore the pathways that mediate the relationship between health spending and health outcomes. For that, we explore the impacts of the EC/29 on several health inputs: primary care coverage, human resources, hospital infrastructure, primary care related infrastructure, ambulatorial production, and access to health services. 

\begin{table}[H]
\begin{footnotesize}
\begin{center}
\scalebox{0.65}{
\begin{threeparttable}[b]

  \centering
  \caption{Primary Care Coverage, Health Infrastructure and Human Resources}
     \begin{tabular}{rrcccr}
          &       &       &       &       &  \\
    \midrule
    \midrule
          &       & \multicolumn{4}{c}{Distance to EC9 target} \\
\cmidrule{3-6}          &       & (1)   & (2)   & (3)   & \multicolumn{1}{c}{(4)} \\
    \midrule
    \multicolumn{1}{p{23.645em}}{\textbf{A. Primary Care Coverage -  Extensive Margin}} &       &       &       &       &  \\
    \multicolumn{1}{l}{\multirow{2}[0]{*}{Population covered (share) by Community Health Agents}} &       & 0.25*** & 0.245*** & 0.245*** & \multicolumn{1}{c}{0.245***} \\
          &       & (0.056) & (0.055) & (0.055) & \multicolumn{1}{c}{(0.055)} \\
    \multicolumn{1}{l}{\multirow{2}[0]{*}{Population covered (share) by Family Health Agents}} &       & 0.187*** & 0.197*** & 0.201*** & \multicolumn{1}{c}{0.2***} \\
          &       & (0.059) & (0.058) & (0.058) & \multicolumn{1}{c}{(0.058)} \\
          &       &       &       &       &  \\
    \midrule
    \multicolumn{1}{p{23.645em}}{\textbf{B. Primary Care Coverage -  Intensive Margin}} &       &       &       &       &  \\
    \multicolumn{1}{l}{\multirow{2}[0]{*}{N. of People Visited by Primary Care Agents (per capita)}} &       & 0.294*** & 0.287*** & 0.298*** & \multicolumn{1}{c}{0.297***} \\
          &       & (0.101) & (0.097) & (0.097) & \multicolumn{1}{c}{(0.097)} \\
    \multicolumn{1}{l}{\multirow{2}[0]{*}{N. of People Visited by Community Health Agents (per capita)}} &       & -0.028 & -0.025 & -0.026 & \multicolumn{1}{c}{-0.026} \\
          &       & (0.053) & (0.053) & (0.053) & \multicolumn{1}{c}{(0.053)} \\
    \multicolumn{1}{l}{\multirow{2}[0]{*}{N. of People Visited by Family Health Agents (per capita)}} &       & 0.321*** & 0.311*** & 0.323*** & \multicolumn{1}{c}{0.322***} \\
          &       & (0.098) & (0.093) & (0.093) & \multicolumn{1}{c}{(0.093)} \\
    \multicolumn{1}{l}{\multirow{2}[0]{*}{N. of Household Visits (per capita)}} &       & 1.059*** & 1.057*** & 1.085*** & \multicolumn{1}{c}{1.085***} \\
          &       & (0.325) & (0.325) & (0.324) & \multicolumn{1}{c}{(0.324)} \\
    \multicolumn{1}{l}{\multirow{2}[0]{*}{N. of Household Visits by Community Health Agents (per capita)}} &       & 0.396 & 0.375 & 0.368 & \multicolumn{1}{c}{0.37} \\
          &       & (0.277) & (0.279) & (0.279) & \multicolumn{1}{c}{(0.278)} \\
    \multicolumn{1}{l}{\multirow{2}[0]{*}{N. of Household Visits by Family Health Agents (per capita)}} &       & 0.653** & 0.676*** & 0.711*** & \multicolumn{1}{c}{0.71***} \\
          &       & (0.256) & (0.247) & (0.246) & \multicolumn{1}{c}{(0.246)} \\
    \multicolumn{1}{l}{\multirow{2}[0]{*}{N. of Appointments (per capita)}} &       & 0.181* & 0.181* & 0.186* & \multicolumn{1}{c}{0.187*} \\
          &       & (0.108) & (0.108) & (0.109) & \multicolumn{1}{c}{(0.109)} \\
    \multicolumn{1}{l}{\multirow{2}[0]{*}{N. of Appointments from Community Health Program (per capita)}} &       & -0.015 & -0.013 & -0.013 & \multicolumn{1}{c}{-0.013} \\
          &       & (0.02) & (0.021) & (0.021) & \multicolumn{1}{c}{(0.021)} \\
    \multicolumn{1}{l}{\multirow{2}[0]{*}{N. of Appointments from Family Health Program (per capita)}} &       & 0.192* & 0.188* & 0.193* & \multicolumn{1}{c}{0.194*} \\
          &       & (0.108) & (0.107) & (0.108) & \multicolumn{1}{c}{(0.108)} \\
          &       &       &       &       &  \\
    \midrule
    \multicolumn{1}{p{23.645em}}{\textbf{C. Number of Health Facilities (per capita * 1000) with}} &       &       &       &       &  \\
    \multicolumn{1}{l}{\multirow{2}[0]{*}{Ambulatory Service}} &       & -0.094** & -0.085** & -0.08* & \multicolumn{1}{c}{ -0.08* } \\
          &       & (0.042) & (0.043) & (0.042) & \multicolumn{1}{c}{ (0.042) } \\
    \multicolumn{1}{l}{\multirow{2}[0]{*}{Ambulatory Service and PSF Teams}} &       & 0.061** & 0.059** & 0.063** & \multicolumn{1}{c}{ 0.063** } \\
          &       & (0.029) & (0.028) & (0.028) & \multicolumn{1}{c}{ (0.028) } \\
    \multicolumn{1}{l}{\multirow{2}[0]{*}{Ambulatory Service and ACS Teams}} &       & 0.046 & 0.052 & 0.056* & \multicolumn{1}{c}{ 0.056* } \\
          &       & (0.033) & (0.032) & (0.032) & \multicolumn{1}{c}{ (0.032) } \\
    \multicolumn{1}{l}{\multirow{2}[0]{*}{Ambulatory Service and Community Doctors}} &       & 0.054* & 0.056* & 0.061** & \multicolumn{1}{c}{ 0.061** } \\
          &       & (0.032) & (0.031) & (0.031) & \multicolumn{1}{c}{ (0.031) } \\
    \multicolumn{1}{l}{\multirow{2}[0]{*}{Ambulatory Service and PSF Doctors}} &       & 0.047 & 0.051* & 0.056* & \multicolumn{1}{c}{ 0.056* } \\
          &       & (0.032) & (0.03) & (0.03) & \multicolumn{1}{c}{ (0.03) } \\
    \multicolumn{1}{l}{\multirow{2}[0]{*}{Ambulatory Service and PSF Nurses}} &       & 0.033* & 0.032 & 0.034* & \multicolumn{1}{c}{ 0.034* } \\
          &       & (0.02) & (0.02) & (0.02) & \multicolumn{1}{c}{ (0.02) } \\
    \multicolumn{1}{l}{\multirow{2}[0]{*}{Ambulatory Service and PSF Nursing Assistants}} &       & 0.061** & 0.066** & 0.07** & \multicolumn{1}{c}{ 0.07** } \\
          &       & (0.031) & (0.03) & (0.029) & \multicolumn{1}{c}{ (0.029) } \\
    \multicolumn{1}{l}{\multirow{2}[0]{*}{Ambulatory Service and ACS Nurses}} &       & 0.02  & 0.023 & 0.028 & \multicolumn{1}{c}{ 0.028 } \\
          &       & (0.03) & (0.029) & (0.029) & \multicolumn{1}{c}{ (0.029) } \\
          &       &       &       &       &  \\
    \midrule
    \multicolumn{1}{p{23.645em}}{\textbf{D. Hospital and Beds}} &       &       &       &       &  \\
    \multicolumn{1}{l}{\multirow{2}[0]{*}{N. of Hospital Beds (per capita * 1000)}} &       & -0.933** & -0.803** & -0.779** & \multicolumn{1}{c}{ -0.78** } \\
          &       & (0.373) & (0.373) & (0.372) & \multicolumn{1}{c}{ (0.372) } \\
    \multicolumn{1}{l}{\multirow{2}[0]{*}{Presence of Hospital}} &       & -0.11*** & -0.082** & -0.082** & \multicolumn{1}{c}{ -0.082** } \\
          &       & (0.041) & (0.039) & (0.039) & \multicolumn{1}{c}{ (0.039) } \\
          &       &       &       &       &  \\
    \midrule
    \midrule
          &       &       &       &       &  \\
    \end{tabular}%
    
    
  \label{table:infra}%

\end{threeparttable}
}
\end{center}
\end{footnotesize}
\end{table}

First we analyse the effects on primary care coverage at the extensive and intensive margin (Table \ref{table:infra}, Panel A and B). We find significant effects on the share of population covered by the Community Health program and by Family Health Program. Though significant and positive, these effects are quite small. The representative municipality only increased by 2 percentage points the share of the population covered by these primary care programs. On the other hand, the effects on the intensive margin are much more pronounced and our estimates suggests that they come mainly from the Family Health Program. We find significant increases in the number of people visited and in the number of household visits and appointments by Family Health Agents. They are equivalent to a 21\% and 11\% increase relative to the baseline, respectively, for our representative municipality. Figure \ref{fig:8} shows the dynamic effects for the extensive margin and Figure \ref{fig:9} the dynamic effects for the intensive margin of primary care coverage. The temporal patterns of this effects resemble the pattern in health spending increase, where the effect is increasing in the first years after the EC/29 and becomes steady after 2004. 

\begin{figure}[h!]
    \begin{center}
    \caption{Effects on Public Spending per capita - By Category}\label{fig:8}
    \begin{subfigure}{0.48\textwidth}
        \caption{\scriptsize Health and Sanitation}\label{fig:8a}
        \centering
        \includegraphics[width=\textwidth]{plots/finbra_desp_saude_san_pcapita_dist_ec29_baseline_dist_ec29_baseline_8.pdf}
    \end{subfigure}
    \begin{subfigure}{0.48\textwidth}
        \centering
        \caption{\scriptsize Education and Culture}\label{fig:8b}
        \includegraphics[width=\textwidth]{plots/finbra_desp_educ_cultura_pcapita_dist_ec29_baseline_dist_ec29_baseline_8.pdf}
    \end{subfigure}
    \begin{subfigure}{0.48\textwidth}
        \centering
        \caption{\scriptsize Social Assistance}\label{fig:8c}
        \includegraphics[width=\textwidth]{plots/finbra_desp_assist_prev_pcapita_dist_ec29_baseline_dist_ec29_baseline_8.pdf}
    \end{subfigure}
    \begin{subfigure}{0.48\textwidth}
        \centering
        \caption{\scriptsize Transport}\label{fig:8d}
        \includegraphics[width=\textwidth]{plots/finbra_desp_transporte_pcapita_dist_ec29_baseline_dist_ec29_baseline_8.pdf}
    \end{subfigure}
    \begin{subfigure}{0.48\textwidth}
        \centering
        \caption{\scriptsize Housing and Urban}\label{fig:8e}
        \includegraphics[width=\textwidth]{plots/finbra_desp_hab_urb_pcapita_dist_ec29_baseline_dist_ec29_baseline_8.pdf}
    \end{subfigure}
    \begin{subfigure}{0.48\textwidth}
        \centering
        \caption{\scriptsize Spending in Other Categories}\label{fig:8f}
        \includegraphics[width=\textwidth]{plots/finbra_desp_outros_area_pcapita_dist_ec29_baseline_dist_ec29_baseline_8.pdf}
    \end{subfigure}
    
    \end{center}
    
\end{figure}

\begin{figure}[h!]
    \begin{center}
    \caption{Effects on Public Health Spending per capita}\label{fig:9}
    \begin{subfigure}{0.48\textwidth}
        \caption{\scriptsize Health and Sanitation (Finbra)}\label{fig:9a}
        \centering
\includegraphics[width=\textwidth]{plots/finbra_desp_saude_san_pcapita_dist_ec29_baseline_dist_ec29_baseline_9.pdf}
    \end{subfigure}
    \begin{subfigure}{0.48\textwidth}
        \centering
        \caption{\scriptsize Total Health Spending (SIOPS)}\label{fig:9b}
        \includegraphics[width=\textwidth]{plots/siops_despsaude_pcapita_dist_ec29_baseline_dist_ec29_baseline_9.pdf}
    \end{subfigure}
    \begin{subfigure}{0.48\textwidth}
        \centering
        \caption{\scriptsize Health Spending - Own Resources (SIOPS)}\label{fig:9c}
        \includegraphics[width=\textwidth]{plots/siops_desprecpropriosaude_pcapita_dist_ec29_baseline_dist_ec29_baseline_9.pdf}
    \end{subfigure}
    \begin{subfigure}{0.48\textwidth}
        \centering
        \caption{\scriptsize Health Spending  - Transfers (SIOPS)}\label{fig:9d}
        \includegraphics[width=\textwidth]{plots/siops_despexrecproprio_pcapita_dist_ec29_baseline_dist_ec29_baseline_9.pdf}
    \end{subfigure}
    
    \end{center}
    
\end{figure}

Panel C presents our results on the public health sector human resources. We find no significant effect for the number of doctors and nurses, but Figures \ref{fig:10a} and \ref{fig:10b} suggests some effect taking place after 2005. On the other hand, we find significant effects in the number of nursing assistants and administrative professionals, a 13\% and 15\% increase, respectively, relative to the baseline. While Figure \ref{fig:10c} indicates a gradual effect trend after the EC/29, Figure \ref{fig:10d} suggests a sharp increase in the number of administrative personnel right after the EC/29. Not coincidentally, this pattern resembles the pattern found in other expenditures within health spending (Figure \ref{fig:7d}), that, as mentioned before, includes mainly administrative spending. 

\begin{figure}[h!]
    \begin{center}
    \caption{Effects on Public Health Spending per capita - By Type}\label{fig:10}
    \begin{subfigure}{0.48\textwidth}
        \centering
        \caption{\scriptsize Human Resources}\label{fig:10a}
        \includegraphics[width=\textwidth]{plots/spending/siops_desppessoal_pcapita_dist_ec29_baseline_dist_ec29_baseline_full.pdf}
    \end{subfigure}
    \begin{subfigure}{0.48\textwidth}
        \centering
        \caption{\scriptsize Investiment}\label{fig:10b}
        \includegraphics[width=\textwidth]{plots/spending/siops_despinvest_pcapita_dist_ec29_baseline_dist_ec29_baseline_full.pdf}
    \end{subfigure}
    \begin{subfigure}{0.48\textwidth}
        \centering
        \caption{\scriptsize 3rd parties services}\label{fig:10c}
        \includegraphics[width=\textwidth]{plots/spending/siops_despservicoster_pcapita_dist_ec29_baseline_dist_ec29_baseline_full.pdf}
    \end{subfigure}
    \begin{subfigure}{0.48\textwidth}
        \centering
        \caption{\scriptsize Other Expenditures}\label{fig:10d}
        \includegraphics[width=\textwidth]{plots/spending/siops_despoutros_pcapita_dist_ec29_baseline_dist_ec29_baseline_full.pdf}
    \end{subfigure}
    
    \end{center}
    
\end{figure}

Next, panel D shows the results for health infrastructure. The number of municipal hospital per 1000 inhabitants presented a significant increase of 0.16. This effect represents a sizable variation of 27\% relative to the baseline number of hospitals for the representative municipality. Moreover, Figure \ref{fig:11a} suggests that the effect follows a similar dynamic pattern as the increase in investment spending within health (Figure \ref{fig:7b}). In this analysis we leverage the variation in municipal health spending induced by the EC/29 within state governments, so one would not expect to see increases in the number of hospital from other governmental spheres or from the private sector. Our results validates that. Yet, the point estimates for the number of Federal, State and Private hospitals are negative, which could suggest some substitution effects in the supply of hospitals. These results would be worrisome if the effects were large, as mortality outcomes can be affected by the supply of hospitals, but that is not what the point estimates and Figure \ref{fig:11b} \ref{fig:11b} suggests. Our results also indicate some marginally significant reduction in the number of health facilities with ambulatory service, but this effect is rather small, representing a reduction of 1.5\% relative to the baseline for the representative municipality.

\begin{figure}[h!]
    \begin{center}
    \caption{Effects on Health Infrastructure}\label{fig:11}
    \begin{subfigure}{0.48\textwidth}
        \caption{\scriptsize N. of Municipal Hospitals (per capita*1000)}\label{fig:11a}
        \centering
\includegraphics[width=\textwidth]{plots/ams_hospital_mun_pcapita_dist_ec29_baseline_dist_ec29_baseline_11.pdf}
    \end{subfigure}
    \begin{subfigure}{0.48\textwidth}
        \centering
        \caption{\scriptsize N. of Federal and State Hospitals (per capita*1000)}\label{fig:11b}
        \includegraphics[width=\textwidth]{plots/ams_hospital_nmun_pcapita_dist_ec29_baseline_dist_ec29_baseline_11.pdf}
    \end{subfigure}
    \begin{subfigure}{0.48\textwidth}
        \centering
        \caption{\scriptsize N. of Private Hospitals (per capita*1000)}\label{fig:11c}
        \includegraphics[width=\textwidth]{plots/ams_hospital_pvt_pcapita_dist_ec29_baseline_dist_ec29_baseline_11.pdf}
    \end{subfigure}
    \begin{subfigure}{0.48\textwidth}
        \centering
        \caption{\scriptsize N. of Health Facilities with Ambulatory Service (per capita*1000)}\label{fig:11d}
        \includegraphics[width=\textwidth]{plots/sia_ncnes_amb_mun_pcapita_dist_ec29_baseline_dist_ec29_baseline_11.pdf}
    \end{subfigure}
    
    \end{center}
    \scriptsize{Notes: The number of observations is 19364 for Figure \ref{fig:11a}, \ref{fig:11b}, \ref{fig:11c} and 48916 for the remaining. DiD Estimates from Equation \ref{eq:2}. Independent variable is the distance to the EC/29 target in p.p. Square dots represent the baseline model with municipality and state-year fixed effects. Round dots represent fully saturated specification (Column 4 in regression Tables). Lines represent 95\% confidence intervals. Arrows, when present, indicate confidence intervals out of the plot bounds. Standard errors are clustered in the municipality level.}
    
\end{figure}

We also find significant effects on the number of health facilities with ambulatorial services and professionals related to primary care (Panel E of Table \ref{table:infra} and Figure \ref{fig:12}), with effects ranging from 5\% to 10\% relative to the baseline.

\begin{figure}[h!]
    \begin{center}
    \caption{Effects on Primary Care Coverage - Intensive Margin}\label{fig:12}
    \begin{subfigure}{0.32\textwidth}
        \caption{\scriptsize N. of People Visited}\label{fig:12a}
        \centering
        \includegraphics[width=\textwidth]{plots/siab_accomp_especif_pcapita_dist_ec29_baseline_dist_ec29_baseline_12.pdf}
    \end{subfigure}
    \begin{subfigure}{0.32\textwidth}
        \centering
        \caption{\scriptsize People Visited by CH Agents}\label{fig:12b}
        \includegraphics[width=\textwidth]{plots/siab_accomp_especif_pacs_pcapita_dist_ec29_baseline_dist_ec29_baseline_12.pdf}
    \end{subfigure}
    \begin{subfigure}{0.32\textwidth}
        \centering
        \caption{\scriptsize People Visited by FH Agents}\label{fig:12c}
        \includegraphics[width=\textwidth]{plots/siab_accomp_especif_psf_pcapita_dist_ec29_baseline_dist_ec29_baseline_12.pdf}
    \end{subfigure}
        \begin{subfigure}{0.32\textwidth}
        \caption{\scriptsize N. of Household Visits}\label{fig:12d}
        \centering
        \includegraphics[width=\textwidth]{plots/siab_visit_cha_pcapita_dist_ec29_baseline_dist_ec29_baseline_12.pdf}
    \end{subfigure}
    \begin{subfigure}{0.32\textwidth}
        \centering
        \caption{\scriptsize N. of Household Visits by CH Agents}\label{fig:12e}
        \includegraphics[width=\textwidth]{plots/siab_visit_cha_pacs_pcapita_dist_ec29_baseline_dist_ec29_baseline_12.pdf}
    \end{subfigure}
    \begin{subfigure}{0.32\textwidth}
        \centering
        \caption{\scriptsize N. of Household Visits by FH Agents}\label{fig:12f}
        \includegraphics[width=\textwidth]{plots/siab_visit_cha_psf_pcapita_dist_ec29_baseline_dist_ec29_baseline_12.pdf}
    \end{subfigure}
        \begin{subfigure}{0.32\textwidth}
        \caption{\scriptsize N. of Appointments}\label{fig:12g}
        \centering
        \includegraphics[width=\textwidth]{plots/siab_cons_especif_pcapita_dist_ec29_baseline_dist_ec29_baseline_12.pdf}
    \end{subfigure}
    \begin{subfigure}{0.32\textwidth}
        \centering
        \caption{\scriptsize N. of Appointments from CH Program}\label{fig:12h}
        \includegraphics[width=\textwidth]{plots/siab_cons_especif_pacs_pcapita_dist_ec29_baseline_dist_ec29_baseline_12.pdf}
    \end{subfigure}
    \begin{subfigure}{0.32\textwidth}
        \centering
        \caption{\scriptsize N. of Appointments from FH Program}\label{fig:12i}
        \includegraphics[width=\textwidth]{plots/siab_cons_especif_psf_pcapita_dist_ec29_baseline_dist_ec29_baseline_12.pdf}
    \end{subfigure}
    
    \end{center}
    
\end{figure}

Finally, in Table \ref{table:production} we assess the impacts on ambulatory production and on the access to health services. Panel A show significant effects on the number of outpatient procedures, primary care outpatient procedures and outpatient procedures of low and mid complexity. These point estimates represent a considerably small increase in production, between 2-3\%. We find no significant impact on outpatient procedures of high complexity. The dynamic effects for these outcomes are presented in Figure \ref{fig:14}. In Panel B we present our estimates for the access to health services, measured by prenatal visits. The results show a significant decrease of 0.093 in prenatal visits ignored, that measures under-registration of information on birth records, and a increase of 0.116 in 1 to 6 prenatal visits. These results suggests an improvement in data registration, and considerably small effect on prenatal 1-6. If we consider only the effect beyond the reduction in under-registration, the effect will be equivalent to only 0.4\% increase relative to the baseline for the representative municipality. Figure \ref{fig:13} present the dynamic effects for prenatal visits. Figure \ref{fig:13b} suggests that the EC/29 might have had some effect in reducing the share of births with mothers having no prenatal visits, which could explain this increase in prenatal 1-6 above the reduction in under-registration.

\begin{table}[H]
\begin{footnotesize}
\begin{center}
\scalebox{0.8}{
\begin{threeparttable}[b]

  \centering
  \caption{Ambulatorial Production and Access to Health Services}
     \begin{tabular}{rrrrrr}
          &       &       &       &       &  \\
    \midrule
    \midrule
          &       & \multicolumn{4}{c}{Distance to EC9 target} \\
\cmidrule{3-6}          &       & \multicolumn{1}{c}{(1)} & \multicolumn{1}{c}{(2)} & \multicolumn{1}{c}{(3)} & \multicolumn{1}{c}{(4)} \\
    \midrule
    \multicolumn{1}{p{26.355em}}{\textbf{A. Ambulatorial Production}} &       &       &       &       &  \\
    \multicolumn{1}{l}{\multirow{2}[0]{*}{Outpatient procedures per capita}} &       & \multicolumn{1}{c}{2.67***} & \multicolumn{1}{c}{2.473**} & \multicolumn{1}{c}{2.544**} & \multicolumn{1}{c}{ 2.528** } \\
          &       & \multicolumn{1}{c}{(1.027)} & \multicolumn{1}{c}{(1.025)} & \multicolumn{1}{c}{(1.021)} & \multicolumn{1}{c}{ (1.019) } \\
    \multicolumn{1}{l}{\multirow{2}[0]{*}{Primary Care Outpatient procedures per capita}} &       & \multicolumn{1}{c}{2.231**} & \multicolumn{1}{c}{2.228**} & \multicolumn{1}{c}{2.282**} & \multicolumn{1}{c}{ 2.266** } \\
          &       & \multicolumn{1}{c}{(0.944)} & \multicolumn{1}{c}{(0.934)} & \multicolumn{1}{c}{(0.931)} & \multicolumn{1}{c}{ (0.929) } \\
    \multicolumn{1}{l}{\multirow{2}[0]{*}{N. of Low \& Mid Complexity Outpatient Procedures (per capita)}} &       & \multicolumn{1}{c}{2.337**} & \multicolumn{1}{c}{2.225**} & \multicolumn{1}{c}{2.331***} & \multicolumn{1}{c}{ 2.337*** } \\
          &       & \multicolumn{1}{c}{(0.908)} & \multicolumn{1}{c}{(0.898)} & \multicolumn{1}{c}{(0.892)} & \multicolumn{1}{c}{ (0.891) } \\
    \multicolumn{1}{l}{\multirow{2}[0]{*}{N. of High Complexity Outpatient Procedures (per capita)}} &       & \multicolumn{1}{c}{-0.136} & \multicolumn{1}{c}{-0.191} & \multicolumn{1}{c}{-0.174} & \multicolumn{1}{c}{ -0.177 } \\
          &       & \multicolumn{1}{c}{(0.134)} & \multicolumn{1}{c}{(0.133)} & \multicolumn{1}{c}{(0.132)} & \multicolumn{1}{c}{ (0.132) } \\
    \multicolumn{1}{l}{\multirow{2}[0]{*}{N. of Outpatient Procedures by Low Skilled Workers (per capita)}} &       & \multicolumn{1}{c}{0.04} & \multicolumn{1}{c}{0.019} & \multicolumn{1}{c}{0.074} & \multicolumn{1}{c}{ 0.074 } \\
          &       & \multicolumn{1}{c}{(0.354)} & \multicolumn{1}{c}{(0.346)} & \multicolumn{1}{c}{(0.343)} & \multicolumn{1}{c}{ (0.342) } \\
    \multicolumn{1}{l}{\multirow{2}[0]{*}{N. of Outpatient procedures by Mid Skilled Workers (per capita)}} &       & \multicolumn{1}{c}{0.383} & \multicolumn{1}{c}{0.346} & \multicolumn{1}{c}{0.353} & \multicolumn{1}{c}{ 0.351 } \\
          &       & \multicolumn{1}{c}{(0.481)} & \multicolumn{1}{c}{(0.479)} & \multicolumn{1}{c}{(0.479)} & \multicolumn{1}{c}{ (0.479) } \\
          &       &       &       &       &  \\
    \midrule
    \multicolumn{1}{p{26.355em}}{\textbf{B. Access to Health Services}} &       &       &       &       &  \\
    \multicolumn{1}{l}{\multirow{2}[0]{*}{Prenatal Visits None}} &       & \multicolumn{1}{c}{0.025*} & \multicolumn{1}{c}{0.006} & \multicolumn{1}{c}{0.005} & \multicolumn{1}{c}{0.005} \\
          &       & \multicolumn{1}{c}{(0.015)} & \multicolumn{1}{c}{(0.011)} & \multicolumn{1}{c}{(0.011)} & \multicolumn{1}{c}{(0.011)} \\
    \multicolumn{1}{l}{\multirow{2}[0]{*}{Prenatal Visits 1-6}} &       & \multicolumn{1}{c}{0.13**} & \multicolumn{1}{c}{0.121**} & \multicolumn{1}{c}{0.118*} & \multicolumn{1}{c}{0.116*} \\
          &       & \multicolumn{1}{c}{(0.059)} & \multicolumn{1}{c}{(0.061)} & \multicolumn{1}{c}{(0.061)} & \multicolumn{1}{c}{(0.06)} \\
    \multicolumn{1}{l}{\multirow{2}[0]{*}{Prenatal Visits 7+}} &       & \multicolumn{1}{c}{-0.051} & \multicolumn{1}{c}{-0.034} & \multicolumn{1}{c}{-0.03} & \multicolumn{1}{c}{-0.029} \\
          &       & \multicolumn{1}{c}{(0.074)} & \multicolumn{1}{c}{(0.062)} & \multicolumn{1}{c}{(0.061)} & \multicolumn{1}{c}{(0.061)} \\
          &       &       &       &       &  \\
    \bottomrule
    \bottomrule
    \end{tabular}%
    
    
  \label{table:production}%

\end{threeparttable}
}
\end{center}
\end{footnotesize}
\end{table}

\begin{figure}[h!]
    \begin{center}
    \caption{Effects on Ambulatorial Production}\label{fig:14}
    \begin{subfigure}{0.48\textwidth}
        \centering
        \caption{\scriptsize Total}\label{fig:14a}
        \includegraphics[width=\textwidth]{plots/sia_pcapita_dist_ec29_baseline_dist_ec29_baseline_14.pdf}
    \end{subfigure}
    \begin{subfigure}{0.48\textwidth}
        \centering
        \caption{\scriptsize Primary Care}\label{fig:14b}
        \includegraphics[width=\textwidth]{plots/sia_ab_pcapita_dist_ec29_baseline_dist_ec29_baseline_14.pdf}
    \end{subfigure}
    \begin{subfigure}{0.48\textwidth}
        \centering
        \caption{\scriptsize Low and Mid Complexity}\label{fig:14c}
        \includegraphics[width=\textwidth]{plots/sia_nprod_amb_lc_mun_pcapita_dist_ec29_baseline_dist_ec29_baseline_14.pdf}
    \end{subfigure}
    \begin{subfigure}{0.48\textwidth}
        \centering
        \caption{\scriptsize High Complexity}\label{fig:14d}
        \includegraphics[width=\textwidth]{plots/sia_nprod_amb_hc_mun_pcapita_dist_ec29_baseline_dist_ec29_baseline_14.pdf}
    \end{subfigure}
    
    \end{center}
    
        \scriptsize{Notes: The number of observations is 64482 for \ref{fig:14a} and \ref{fig:14b}, 48916 for the remaining. DiD Estimates from Equation \ref{eq:2}. Independent variable is the distance to the EC/29 target in p.p. Square dots represent the baseline model with municipality and state-year fixed effects. Round dots represent fully saturated specification (Column 4 in regression Tables). Lines represent 95\% confidence intervals. Arrows, when present, indicate confidence intervals out of the plot bounds. Standard errors are clustered in the municipality level.}
    
\end{figure}

\begin{figure}[h!]
    \begin{center}
    \caption{Effects on Infrastructure and Human Resources: N. of Health Facilities with:}\label{fig:13}
    \begin{subfigure}{0.32\textwidth}
        \caption{\scriptsize Ambulatory Service}\label{fig:13a}
        \centering
        \includegraphics[width=\textwidth]{plots/sia_ncnes_amb_mun_pcapita_dist_ec29_baseline_dist_ec29_baseline_13.pdf}
    \end{subfigure}
    \begin{subfigure}{0.32\textwidth}
        \centering
        \caption{\scriptsize Ambulatory Service and PSF Teams}\label{fig:13b}
        \includegraphics[width=\textwidth]{plots/sia_ncnes_psf_pcapita_dist_ec29_baseline_dist_ec29_baseline_13.pdf}
    \end{subfigure}
    \begin{subfigure}{0.32\textwidth}
        \centering
        \caption{\scriptsize Ambulatory Service and ACS Teams}\label{fig:13c}
        \includegraphics[width=\textwidth]{plots/sia_ncnes_acs_pcapita_dist_ec29_baseline_dist_ec29_baseline_13.pdf}
    \end{subfigure}
        \begin{subfigure}{0.32\textwidth}
        \caption{\scriptsize Ambulatory Service and Community Doctors}\label{fig:13d}
        \centering
        \includegraphics[width=\textwidth]{plots/sia_ncnes_medcom_pcapita_dist_ec29_baseline_dist_ec29_baseline_13.pdf}
    \end{subfigure}
    \begin{subfigure}{0.32\textwidth}
        \centering
        \caption{\scriptsize Ambulatory Service and PSF Doctors}\label{fig:13e}
        \includegraphics[width=\textwidth]{plots/sia_ncnes_medpsf_pcapita_dist_ec29_baseline_dist_ec29_baseline_13.pdf}
    \end{subfigure}
    \begin{subfigure}{0.32\textwidth}
        \centering
        \caption{\scriptsize Ambulatory Service and PSF Nurses}\label{fig:13f}
        \includegraphics[width=\textwidth]{plots/sia_ncnes_enfpsf_pcapita_dist_ec29_baseline_dist_ec29_baseline_13.pdf}
    \end{subfigure}
        \begin{subfigure}{0.32\textwidth}
        \caption{\scriptsize Ambulatory Service and PSF Nursing Assistants}\label{fig:13g}
        \centering
        \includegraphics[width=\textwidth]{plots/sia_ncnes_outpsf_pcapita_dist_ec29_baseline_dist_ec29_baseline_13.pdf}
    \end{subfigure}
    \begin{subfigure}{0.32\textwidth}
        \centering
        \caption{\scriptsize Ambulatory Service and ACS Nurses}\label{fig:13h}
        \includegraphics[width=\textwidth]{plots/sia_ncnes_enfacs_pcapita_dist_ec29_baseline_dist_ec29_baseline_13.pdf}
    \end{subfigure}
    
    \end{center}
    
\end{figure}

With the data available we are not able to directly connect the increase in health spending with the increase in health inputs presented in this section. However, the evidence presented so far suggests that: (i) increases in human resource spending have been translated into greater primary care coverage at the intensive margin, a higher number of facilities with primary care personnel, and into a increase in the number of nursing assistants; (ii) increases in investment spending has been translated into a greater supply of municipal hospitals and a marginal increase ambulatory production; and (iii) increases in other expenditures, which consist mainly of administrative spending, might be associated with the increase in the number of administrative professionals.


\subsection{Effects on Infant Mortality}

Having provided meaningful evidence of the effects of EC/29 on health spending and how these effects translated into health inputs, we now present estimates of the effects on infant mortality. Differently from most of the literature linking health spending with infant mortality \citep{filmer1999,bokhari2007,moreno2015,nixon2006,gupta2002effectiveness,cremieux1999,bokhari2007}, we are able to assess the effects not only for total infant mortality rates, but also for infant mortality rates by timing of death and by cause of death. We are also able to analyse infant mortality rates by classifying them between amenable and non-amenable to primary care. These results are presented in Table \ref{table:imr}. In all specifications presented for this section, we added a trend of baseline ill-defined infant mortality with the goal of accounting for mortality under-reporting\footnote{Appendix Table \ref{app:imr} presents estimates with and without the baseline ill-defined infant mortality trend.}. During the period of analysis, the completeness of death counts improved considerably and it is strongly associated with the reduction in ill-defined causes of death \citep{lima2014evolution}.

\begin{table}[h!]
\begin{footnotesize}
\begin{center}
\scalebox{0.9}{
\begin{threeparttable}[b]

  \centering
  \caption{Infant Mortality Rates}
     \begin{tabular}{rrcccc}
          &       &       &       &       &  \\
          &       &       &       &       &  \\
    \midrule
    \midrule
          &       & (1)   & (2)   & (3)   & (4) \\
    \midrule
    \multicolumn{1}{p{15.145em}}{\textbf{A. Infant Mortality Rate}} &       &       &       &       &  \\
    \multicolumn{1}{p{15.145em}}{Total} &       & -5.015 & -3.772 & -3.831 & -3.889 \\
          &       & (3.435) & (2.853) & (2.836) & (2.828) \\
    \multicolumn{1}{p{15.145em}}{Amenable to Primary Care} &       & -0.361 & -0.866 & -0.893 & -0.905 \\
          &       & (0.603) & (0.553) & (0.553) & (0.554) \\
    \multicolumn{1}{p{15.145em}}{Non-Amenable to Primary Care} &       & -4.653 & -2.907 & -2.939 & -2.984 \\
          &       & (3.245) & (2.645) & (2.632) & (2.624) \\
          &       &       &       &       &  \\
    \midrule
    \multicolumn{1}{p{15.145em}}{\textbf{B. By timing}} &       &       &       &       &  \\
    \multicolumn{1}{p{15.145em}}{Fetal} &       & -0.008 & -0.007 & -0.008 & -0.008 \\
          &       & (0.008) & (0.008) & (0.008) & (0.008) \\
    \multicolumn{1}{p{15.145em}}{Within 24h} &       & -2.275* & -2.083** & -2.07** & -2.071** \\
          &       & (1.225) & (0.98) & (0.979) & (0.976) \\
    \multicolumn{1}{p{15.145em}}{1 to 27 days} &       & -4.228* & -2.883 & -2.911 & -2.922 \\
          &       & (2.555) & (2.064) & (2.052) & (2.046) \\
    \multicolumn{1}{p{15.145em}}{27 days to 1 year} &       & -0.787 & -0.89 & -0.92 & -0.967 \\
          &       & (1.435) & (1.248) & (1.246) & (1.243) \\
          &       &       &       &       &  \\
    \midrule
    \multicolumn{1}{p{15.145em}}{\textbf{C. By Cause of Death}} &       &       &       &       &  \\
    \multicolumn{1}{p{15.145em}}{Infectious} &       & -0.374 & -0.811 & -0.82 & -0.831 \\
          &       & (0.567) & (0.535) & (0.535) & (0.534) \\
    \multicolumn{1}{p{15.145em}}{Respiratory} &       & -0.494 & -0.507 & -0.511 & -0.517 \\
          &       & (0.474) & (0.411) & (0.409) & (0.409) \\
    \multicolumn{1}{p{15.145em}}{Perinatal} &       & -5.349** & -3.648* & -3.69* & -3.707* \\
          &       & (2.571) & (2.015) & (2.007) & (2.002) \\
    \multicolumn{1}{p{15.145em}}{Congenital} &       & -0.235 & -0.169 & -0.16 & -0.157 \\
          &       & (0.463) & (0.436) & (0.434) & (0.434) \\
    \multicolumn{1}{p{15.145em}}{External} &       & 0.024 & -0.049 & -0.037 & -0.034 \\
          &       & (0.183) & (0.165) & (0.165) & (0.166) \\
    \multicolumn{1}{p{15.145em}}{Nutritional} &       & -0.204 & -0.328 & -0.33 & -0.343 \\
          &       & (0.246) & (0.231) & (0.232) & (0.232) \\
    \multicolumn{1}{p{15.145em}}{Other} &       & -0.183 & -0.123 & -0.132 & -0.139 \\
          &       & (0.201) & (0.199) & (0.198) & (0.198) \\
    \multicolumn{1}{p{15.145em}}{Ill-Defined} &       & 1.8** & 1.862** & 1.849** & 1.84** \\
          &       & (0.849) & (0.776) & (0.779) & (0.779) \\
          &       &       &       &       &  \\
    \bottomrule
    \bottomrule
    \end{tabular}%
    
    
    \begin{tablenotes}
  \scriptsize{\underline{Notes}: The number of observations is 64482. DiD Estimates from Equation \ref{eq:1}. Independent variable is the distance to the EC/29 target in p.p. Column 1 presents the baseline model with municipality and state-year fixed effects. Column 2 adds baseline socioeconomic controls from the Census interacted with time. Column 3 adds controls for GDP per capita and \emph{Bolsa Familia} transfers per capita. Column 4 adds fiscal controls. Covariates omitted. Standard errors in brackets are clustered in the municipality level. ∗p < 0.10, ∗ ∗ p < 0.05, ∗ ∗ ∗p < 0.01}
  \end{tablenotes}
    
    
  \label{table:imr}%

\end{threeparttable}
}
\end{center}
\end{footnotesize}
\end{table}

Panel A present the estimates for total infant mortality rates (IMR) and IMR amenable and non-amenable to primary care. Though not significant, the estimates present the expected sign. Yet, the more flexible coefficients estimated with equation \ref{eq:2} provide useful information on the dynamics of the effects and suggest the presence of some significant reduction in IMR. Figure \ref{fig:15} plots the dynamic effects for the IMR presented in Panel A. IMR (Figure \ref{fig:15a}) and IMR amenable to primary care (Figure \ref{fig:15b})  present a clear trend of reduction, with estimates for 2007 onward being all statistically significant in our preferred specification. 

\begin{figure}[h!]
    \begin{center}
    \caption{Effects on Ambulatorial Production}\label{fig:15}
    \begin{subfigure}{0.32\textwidth}
        \centering
        \caption{\scriptsize Total}\label{fig:15a}
        \includegraphics[width=\textwidth]{plots/sia_pcapita_dist_ec29_baseline_dist_ec29_baseline_15.pdf}
    \end{subfigure}
    \begin{subfigure}{0.32\textwidth}
        \centering
        \caption{\scriptsize Primary Care}\label{fig:15b}
        \includegraphics[width=\textwidth]{plots/sia_ab_pcapita_dist_ec29_baseline_dist_ec29_baseline_15.pdf}
    \end{subfigure}
    \begin{subfigure}{0.32\textwidth}
        \centering
        \caption{\scriptsize Low and Mid Complexity}\label{fig:15c}
        \includegraphics[width=\textwidth]{plots/sia_nprod_amb_lc_mun_pcapita_dist_ec29_baseline_dist_ec29_baseline_15.pdf}
    \end{subfigure}
    \begin{subfigure}{0.32\textwidth}
        \centering
        \caption{\scriptsize High Complexity}\label{fig:15d}
        \includegraphics[width=\textwidth]{plots/sia_nprod_amb_hc_mun_pcapita_dist_ec29_baseline_dist_ec29_baseline_15.pdf}
    \end{subfigure}
    \begin{subfigure}{0.32\textwidth}
        \centering
        \caption{\scriptsize Performed by Low Skilled Worker}\label{fig:15e}
        \includegraphics[width=\textwidth]{plots/sia_nprod_low_skill_mun_pcapita_dist_ec29_baseline_dist_ec29_baseline_15.pdf}
    \end{subfigure}
    \begin{subfigure}{0.32\textwidth}
        \centering
        \caption{\scriptsize Performed by High Skilled Worker}\label{fig:15f}
        \includegraphics[width=\textwidth]{plots/sia_nprod_med_skill_mun_pcapita_dist_ec29_baseline_dist_ec29_baseline_15.pdf}
    \end{subfigure}
    
    \end{center}
    
        \scriptsize{Notes: The number of observations is 64710 for \ref{fig:15a} and \ref{fig:15b}, 49080 for the remaining. DiD Estimates from Equation \ref{eq:2}. Independent variable is the distance to the EC/29 target in p.p. Square dots represent the baseline model with municipality and state-year fixed effects. Round dots represent fully saturated specification (Column 4 in regression Tables). Lines represent 95\% confidence intervals. Arrows, when present, indicate confidence intervals out of the plot bounds. Standard errors are clustered in the municipality level.}
    
\end{figure}

Next, we present the estimates for IMR by timing of death (Panel B). IMR withing 24 hours shows a significant reduction of 2.071, which is equivalent to a 3.7\% decreased in baseline mortality for the representative municipality. The estimates for the remaining IMR of Panel B are not significant, but the dynamic effects suggests the presence of reduction trend in later years for IMR from 1 to 27 days (Figure \ref{fig:16c}) and IMR from 27 days to 1 year old (Figure \ref{fig:16d}).

\begin{figure}[h!]
    \begin{center}
    \caption{Effects on Infant Mortality Rates}\label{fig:16}
    \begin{subfigure}{0.32\textwidth}
        \caption{\scriptsize Total}\label{fig:16a}
        \centering
        \includegraphics[width=\textwidth]{plots/tx_mi_dist_ec29_baseline_dist_ec29_baseline_16.pdf}
    \end{subfigure}
    \begin{subfigure}{0.32\textwidth}
        \centering
        \caption{\scriptsize Amenable to Primary Care}\label{fig:16b}
        \includegraphics[width=\textwidth]{plots/tx_mi_icsap_dist_ec29_baseline_dist_ec29_baseline_16.pdf}
    \end{subfigure}
    \begin{subfigure}{0.32\textwidth}
        \centering
        \caption{\scriptsize Non-Amenable to Primary Care}\label{fig:16c}
        \includegraphics[width=\textwidth]{plots/tx_mi_nicsap_dist_ec29_baseline_dist_ec29_baseline_16.pdf}
    \end{subfigure}
    
    \end{center}
    
\end{figure}

The last part of Table \ref{table:imr} present the estimates for IMR by causes of death. We only find marginally significant reductions for IMR from Perinatal causes. This estimated represents a 3.4\% decrease relative to baseline mortality. We also find the unexpected effect of increases in IMR from ill-defined causes. Moreover, the dynamic effects presented in Figure \ref{fig:17} suggest the presence of some reduction in IMR by infectious and in IMR by respiratory causes.

In our preferred specification, with the exception of ill-defined infant mortality, all outcomes present negative point estimates, but only infant mortality rate within 24 hours and infant mortality rate by infectious diseases present marginally significant estimates. Relative to the baseline mortality rates, a 10\% distance to the target is associated with a reduction of around 3.5\% for mortality within 24 hours and a reduction of around 5\% for mortality due to infectious diseases. 

\begin{figure}[h!]
    \begin{center}
    \caption{Effects on Infant Mortality Rates - By Timing}\label{fig:17}
    \begin{subfigure}{0.48\textwidth}
        \caption{\scriptsize Fetal}\label{fig:17a}
        \centering
        \includegraphics[width=\textwidth]{plots/tx_mi_fet_dist_ec29_baseline_dist_ec29_baseline_17.pdf}
    \end{subfigure}
    \begin{subfigure}{0.48\textwidth}
        \centering
        \caption{\scriptsize Within 24h}\label{fig:17b}
        \includegraphics[width=\textwidth]{plots/tx_mi_24h_dist_ec29_baseline_dist_ec29_baseline_17.pdf}
    \end{subfigure}
    \begin{subfigure}{0.48\textwidth}
        \centering
        \caption{\scriptsize 1 to 27 days}\label{fig:17c}
        \includegraphics[width=\textwidth]{plots/tx_mi_27d_dist_ec29_baseline_dist_ec29_baseline_17.pdf}
    \end{subfigure}
    \begin{subfigure}{0.48\textwidth}
        \centering
        \caption{\scriptsize 27 days to 1 year}\label{fig:17d}
        \includegraphics[width=\textwidth]{plots/tx_mi_ano_dist_ec29_baseline_dist_ec29_baseline_17.pdf}
    \end{subfigure}
    
    \end{center}
    
            \scriptsize{Notes: The number of observations is 64701. DiD Estimates from Equation \ref{eq:2}. Independent variable is the distance to the EC/29 target in p.p. Square dots represent the baseline model with municipality and state-year fixed effects. Round dots represent fully saturated specification (Column 4 in regression Tables). Lines represent 95\% confidence intervals. Arrows, when present, indicate confidence intervals out of the plot bounds. Standard errors are clustered in the municipality level.}
    
\end{figure}

In general, articles estimating the causal relationship between health spending and mortality run log-log regressions and present estimates for the elasticity of mortality with respect to health spending. We explicitly choose not to apply transformations to our health outcomes variables due to the amount of observations with values equal to $0$, notably the ones related to birth and mortality. Our data comprises all the Brazilian municipalities with available data for the period of analyses, some with population size as little as $700$ inhabitants, and it is common to find infant mortality rates of $0$. Running log transformation would throw away some relevant information for several outcomes. Nonetheless, to relate our results to the literature on this topic we estimate "back of the envelope" elasticities for all IMR using the estimates of our regressions. Table \ref{table:elasticity} presents these elasticities. 

The elasticities presented in the literature vary greatly. Within cross-country studies, while \cite{filmer1999} finds a very small elasticity of $-0.08$, \cite{gupta2002effectiveness} finds an elasticity of $-0.31$ and \cite{bokhari2007} elasticities ranging between $-0.4$ and $-0.5$. In the micro studies \cite{cremieux1999} finds large elasticities between $-0.8$ and $-1.1$, \cite{sonia2007} finds an elasticity of $-0.24$ for rural regions, and \cite{castro2021effects} elasticities ranging between $-0.5$ and $-0.9$. Though not significant, our study finds much smaller elasticities for total infant mortality rates. Using SIOPS as the measure of health spending, we find IMR elasticities ranging between $-0.06$ and $-0.14$. When it comes to within 24 hours IMR and Perinatal IMR, the rates for which we found significant decreases, the back of the envelope calculations suggest an elasticity of $-0.136$ and $-0.124$, respectively.


\begin{table}[h!]
\begin{footnotesize}
\begin{center}
\scalebox{0.8}{
\begin{threeparttable}[b]

  \centering
  \caption{Fertility and Birth Outcomes}
     \begin{tabular}{rrcccr}
          &       &       &       &       &  \\
          &       &       &       &       &  \\
    \midrule
    \midrule
          &       & (1)   & (2)   & (3)   & \multicolumn{1}{c}{(4)} \\
    \midrule
    \multicolumn{1}{l}{\textbf{A. Fertility}} &       &       &       &       &  \\
    \multicolumn{1}{p{26.355em}}{Rates of Birth per Woman (10-49y)} &       & 0.009** & 0.008** & 0.009** & \multicolumn{1}{c}{ 0.009** } \\
          &       & (0.004) & (0.003) & (0.003) & \multicolumn{1}{c}{ (0.003) } \\
    \multicolumn{1}{p{26.355em}}{\textbf{B. Birth Outcomes}} &       &       &       &       &  \\
    \multicolumn{1}{p{26.355em}}{Apgar 1} &       & -0.056 & 0.063 & 0.053 & \multicolumn{1}{c}{ 0.051 } \\
          &       & (0.206) & (0.198) & (0.198) & \multicolumn{1}{c}{ (0.198) } \\
    \multicolumn{1}{p{26.355em}}{Apgar 5} &       & 0.009 & 0.107 & 0.104 & \multicolumn{1}{c}{ 0.101 } \\
          &       & (0.183) & (0.179) & (0.18) & \multicolumn{1}{c}{ (0.179) } \\
    \multicolumn{1}{p{26.355em}}{Low Birth Weight (<2.5k)} &       & -0.003 & -0.002 & -0.001 & \multicolumn{1}{c}{ -0.002 } \\
          &       & (0.003) & (0.003) & (0.003) & \multicolumn{1}{c}{ (0.003) } \\
    \multicolumn{1}{p{26.355em}}{Premature Birth} &       & -0.005 & -0.016 & -0.017 & \multicolumn{1}{c}{ -0.017 } \\
          &       & (0.026) & (0.023) & (0.023) & \multicolumn{1}{c}{ (0.023) } \\
    \multicolumn{1}{p{26.355em}}{Sex Ratio at Birth} &       & 0.014 & 0.016 & 0.017 & \multicolumn{1}{c}{ 0.017 } \\
          &       & (0.016) & (0.016) & (0.016) & \multicolumn{1}{c}{ (0.016) } \\
          &       &       &       &       &  \\
    \bottomrule
    \bottomrule
    \end{tabular}%
    
    
    \begin{tablenotes}
  \scriptsize{\underline{Notes}: The number of observations is 64482 for Panel A, 63705 for Apgar 1, 59524 for Apgar 5, 64481 for Low Birth Weight and Premature Birth, and 64470 for Sex Ratio at Birth. DiD Estimates from Equation \ref{eq:1}. Independent variable is the distance to the EC/29 target in p.p. Column 1 presents the baseline model with municipality and state-year fixed effects. Column 2 adds baseline socioeconomic controls from the Census interacted with time. Column 3 adds controls for GDP per capita and \emph{Bolsa Familia} transfers per capita. Column 4 adds fiscal controls. Covariates omitted. Standard errors in brackets are clustered in the municipality level. ∗p < 0.10, ∗ ∗ p < 0.05, ∗ ∗ ∗p < 0.011}
  \end{tablenotes}
    
    
  \label{table:birth}%

\end{threeparttable}
}
\end{center}
\end{footnotesize}
\end{table}

Lastly, table \ref{table:birth} presents the estimates for fertility and birth outcomes. In general, the point estimates are in the expected direction, substantially small,but and statistically insignificant. An exception is our measure of fertility that presents small but significant estimates. A 0.009 increase in the fertility rate, correspond to a 1.6\% variation relative to the baseline for the representative municipality. Figure \ref{fig:18} plots the dynamic effects for fertility and birth outcomes. Figure \ref{fig:18a} suggests the presence of some trend in our fertility measure, that seems to show a constant increase from the pre-treatment period until the last year of analysis.

\begin{figure}[h!]
    \begin{center}
    \caption{Effects on Infant Mortality Rates - By Cause}\label{fig:18}
    \begin{subfigure}{0.32\textwidth}
        \caption{\scriptsize Infectious}\label{fig:18a}
        \centering
        \includegraphics[width=\textwidth]{plots/tx_mi_infec_dist_ec29_baseline_dist_ec29_baseline_18.pdf}
    \end{subfigure}
    \begin{subfigure}{0.32\textwidth}
        \centering
        \caption{\scriptsize Respiratory}\label{fig:18b}
        \includegraphics[width=\textwidth]{plots/tx_mi_resp_dist_ec29_baseline_dist_ec29_baseline_18.pdf}
    \end{subfigure}
    \begin{subfigure}{0.32\textwidth}
        \centering
        \caption{\scriptsize Perinatal}\label{fig:18c}
        \includegraphics[width=\textwidth]{plots/tx_mi_perinat_dist_ec29_baseline_dist_ec29_baseline_18.pdf}
    \end{subfigure}
        \begin{subfigure}{0.32\textwidth}
        \centering
        \caption{\scriptsize Congenital}\label{fig:18d}
        \includegraphics[width=\textwidth]{plots/tx_mi_cong_dist_ec29_baseline_dist_ec29_baseline_18.pdf}
    \end{subfigure}
        \begin{subfigure}{0.32\textwidth}
        \centering
        \caption{\scriptsize External}\label{fig:18e}
        \includegraphics[width=\textwidth]{plots/tx_mi_ext_dist_ec29_baseline_dist_ec29_baseline_18.pdf}
    \end{subfigure}
        \begin{subfigure}{0.32\textwidth}
        \centering
        \caption{\scriptsize Nutritional}\label{fig:18f}
        \includegraphics[width=\textwidth]{plots/tx_mi_nut_dist_ec29_baseline_dist_ec29_baseline_18.pdf}
    \end{subfigure}
        \begin{subfigure}{0.32\textwidth}
        \centering
        \caption{\scriptsize Other}\label{fig:18g}
        \includegraphics[width=\textwidth]{plots/tx_mi_out_dist_ec29_baseline_dist_ec29_baseline_18.pdf}
    \end{subfigure}
        \begin{subfigure}{0.32\textwidth}
        \centering
        \caption{\scriptsize Ill-defined}\label{fig:18h}
        \includegraphics[width=\textwidth]{plots/tx_mi_illdef_dist_ec29_baseline_dist_ec29_baseline_18.pdf}
    \end{subfigure}
    \end{center}
    
            \scriptsize{Notes: The number of observations is 64701. DiD Estimates from Equation \ref{eq:2}. Independent variable is the distance to the EC/29 target in p.p. Square dots represent the baseline model with municipality and state-year fixed effects. Round dots represent fully saturated specification (Column 4 in regression Tables). Lines represent 95\% confidence intervals. Arrows, when present, indicate confidence intervals out of the plot bounds. Standard errors are clustered in the municipality level.}
    
\end{figure}


Overall, we found significant effects for infant moralities rates that are mainly associated with increases in access to primary care and community-based health interventions \citep{rocha2010evaluating,bhalotra2019can}, suggesting that the effects found for health inputs related to primary care might be the channel through which spending affected infant mortality.


\subsection{Robustness}

In Section \ref{sec:emp_val} we stressed that the validity of our research design relies on having parallel trends and homogeneity in treatment effect. However, we can only empirically test the first identification hypothesis. For all outcomes discussed in this article, we presented figures plotting the estimates of Equation \ref{eq:2} that captures pre-trends in the outcome variable. With only few exceptions, our estimates show no pre-trends in our outcomes.

Another concern may arise from unobserved events coinciding the approval of the EC/29. The inclusion of state-year fixed effects in all specification helps address this concern, as well as the concern on pre-trends. Moreover, we presented in all regression tables estimates for four specifications that gradually included different set of controls, with figures plotting estimates only for the baseline specification and the more complete specification. Our analysis shows that estimates for health spending and health inputs are highly robust to different specifications. The estimates for outcomes related to birth and mortality are a bit more sensible to the inclusion of baseline controls, but still very robust to different specifications.
