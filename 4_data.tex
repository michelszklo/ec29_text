\section{Data}\label{sec:data}
\setstretch{1.5}

Table \ref{table:stats} presents summary statistics at the baseline year for all the variables used in this analysis: variables related to the EC2/9, fiscal data, health inputs, infant mortality rates, birth outcomes, and control variables. Statistics are presented for the full sample, and for the bottom and top quartile of distribution of municipalities' own resource spending in public health.

\begin{sidewaystable}
\begin{table}[H]
\begin{footnotesize}
\begin{center}
\scalebox{0.6}{
\begin{threeparttable}[b]

  \centering
  \caption{Descriptive Statistics (at the baseline year)}
    
    \begin{tabular}{rrrrrrrrrrrrrrrrrrrr}
          &       &       &       &       &       &       &       &       &       &       &       &       &       &       &       &       &       &       &  \\
    \midrule
    \midrule
          & \multicolumn{5}{c}{Full Sample}       &       & \multicolumn{5}{c}{Bottom Quartile of OR Spent in Public Health} & \multicolumn{5}{c}{Top Quartile of OR Spent in Public Health} &       &       &  \\
\cmidrule{2-6}\cmidrule{8-17}          & \multicolumn{1}{c}{Mean} & \multicolumn{1}{c}{Std. Dev.} & \multicolumn{1}{c}{Min} & \multicolumn{1}{c}{Max} & \multicolumn{1}{c}{Obs.} &       & \multicolumn{1}{c}{Mean} & \multicolumn{1}{c}{Std. Dev.} & \multicolumn{1}{c}{Min} & \multicolumn{1}{c}{Max} & \multicolumn{1}{c}{Obs.} & \multicolumn{1}{c}{Mean} & \multicolumn{1}{c}{Std. Dev.} & \multicolumn{1}{c}{Min} & \multicolumn{1}{c}{Max} & \multicolumn{1}{c}{Obs.} &       & \multicolumn{1}{c}{Source of Data} & \multicolumn{1}{c}{Baseline Year} \\
\cmidrule{1-6}\cmidrule{8-17}\cmidrule{19-20}    \multicolumn{1}{l}{\textbf{EC 29 Variables}} &       &       &       &       &       &       &       &       &       &       &       &       &       &       &       &       &       &       &  \\
    \multicolumn{1}{l}{Share of Municipality's Own Resource Spent in Public Health} & \multicolumn{1}{c}{0.14} & \multicolumn{1}{c}{0.07} & \multicolumn{1}{c}{0.00} & \multicolumn{1}{c}{0.80} & \multicolumn{1}{c}{5224} &       & \multicolumn{1}{c}{0.06} & \multicolumn{1}{c}{0.02} & \multicolumn{1}{c}{0.00} & \multicolumn{1}{c}{0.09} & \multicolumn{1}{c}{1306} & \multicolumn{1}{c}{0.23} & \multicolumn{1}{c}{0.06} & \multicolumn{1}{c}{0.17} & \multicolumn{1}{c}{0.80} & \multicolumn{1}{c}{1306} &       & \multicolumn{1}{c}{SIOPS} & \multicolumn{1}{c}{2000} \\
    \multicolumn{1}{l}{Distance to the EC29 Target} & \multicolumn{1}{c}{0.01} & \multicolumn{1}{c}{0.07} & \multicolumn{1}{c}{-0.65} & \multicolumn{1}{c}{0.15} & \multicolumn{1}{c}{5224} &       & \multicolumn{1}{c}{0.09} & \multicolumn{1}{c}{0.02} & \multicolumn{1}{c}{0.06} & \multicolumn{1}{c}{0.15} & \multicolumn{1}{c}{1306} & \multicolumn{1}{c}{-0.08} & \multicolumn{1}{c}{0.06} & \multicolumn{1}{c}{-0.65} & \multicolumn{1}{c}{-0.02} & \multicolumn{1}{c}{1306} &       & \multicolumn{1}{c}{SIOPS} & \multicolumn{1}{c}{2000} \\
          &       &       &       &       &       &       &       &       &       &       &       &       &       &       &       &       &       &       &  \\
    \multicolumn{1}{l}{\textbf{Municipality Public Revenue}} &       &       &       &       &       &       &       &       &       &       &       &       &       &       &       &       &       &       &  \\
    \multicolumn{1}{l}{Total Revenue per capita} & \multicolumn{1}{c}{1225.27} & \multicolumn{1}{c}{2282.13} & \multicolumn{1}{c}{132.39} & \multicolumn{1}{c}{121105.02} & \multicolumn{1}{c}{5288} &       & \multicolumn{1}{c}{1162.33} & \multicolumn{1}{c}{3438.04} & \multicolumn{1}{c}{166.10} & \multicolumn{1}{c}{121105.02} & \multicolumn{1}{c}{1257} & \multicolumn{1}{c}{1225.99} & \multicolumn{1}{c}{710.26} & \multicolumn{1}{c}{282.39} & \multicolumn{1}{c}{8866.51} & \multicolumn{1}{c}{1269} &       & \multicolumn{1}{c}{Finbra} & \multicolumn{1}{c}{2000} \\
          &       &       &       &       &       &       &       &       &       &       &       &       &       &       &       &       &       &       &  \\
    \multicolumn{1}{l}{Revenue by Source - per capita} &       &       &       &       &       &       &       &       &       &       &       &       &       &       &       &       &       &       &  \\
    \multicolumn{1}{l}{Tax Revenue per capita} & \multicolumn{1}{c}{65.63} & \multicolumn{1}{c}{276.91} & \multicolumn{1}{c}{0.00} & \multicolumn{1}{c}{17459.07} & \multicolumn{1}{c}{5288} &       & \multicolumn{1}{c}{49.28} & \multicolumn{1}{c}{106.35} & \multicolumn{1}{c}{0.00} & \multicolumn{1}{c}{1607.99} & \multicolumn{1}{c}{1257} & \multicolumn{1}{c}{70.67} & \multicolumn{1}{c}{156.85} & \multicolumn{1}{c}{0.00} & \multicolumn{1}{c}{3145.52} & \multicolumn{1}{c}{1269} &       & \multicolumn{1}{c}{Finbra} & \multicolumn{1}{c}{2000} \\
    \multicolumn{1}{l}{Transfers Revenue per capita} & \multicolumn{1}{c}{1085.67} & \multicolumn{1}{c}{2055.60} & \multicolumn{1}{c}{131.93} & \multicolumn{1}{c}{118718.04} & \multicolumn{1}{c}{5288} &       & \multicolumn{1}{c}{1060.14} & \multicolumn{1}{c}{3364.99} & \multicolumn{1}{c}{164.99} & \multicolumn{1}{c}{118718.04} & \multicolumn{1}{c}{1257} & \multicolumn{1}{c}{1058.28} & \multicolumn{1}{c}{532.77} & \multicolumn{1}{c}{263.25} & \multicolumn{1}{c}{5210.39} & \multicolumn{1}{c}{1269} &       & \multicolumn{1}{c}{Finbra} & \multicolumn{1}{c}{2000} \\
    \multicolumn{1}{l}{Other Revenues per capita} & \multicolumn{1}{c}{73.97} & \multicolumn{1}{c}{194.92} & \multicolumn{1}{c}{0.00} & \multicolumn{1}{c}{5489.05} & \multicolumn{1}{c}{5288} &       & \multicolumn{1}{c}{52.91} & \multicolumn{1}{c}{94.54} & \multicolumn{1}{c}{0.00} & \multicolumn{1}{c}{1220.07} & \multicolumn{1}{c}{1257} & \multicolumn{1}{c}{97.03} & \multicolumn{1}{c}{293.02} & \multicolumn{1}{c}{0.00} & \multicolumn{1}{c}{5489.05} & \multicolumn{1}{c}{1269} &       & \multicolumn{1}{c}{Finbra} & \multicolumn{1}{c}{2000} \\
    \multicolumn{1}{l}{Revenue by Source - (\% Total Revenue)} &       &       &       &       &       &       &       &       &       &       &       &       &       &       &       &       &       &       &  \\
    \multicolumn{1}{l}{Tax Revenue} & \multicolumn{1}{c}{0.05} & \multicolumn{1}{c}{0.06} & \multicolumn{1}{c}{0.00} & \multicolumn{1}{c}{0.62} & \multicolumn{1}{c}{5288} &       & \multicolumn{1}{c}{0.04} & \multicolumn{1}{c}{0.06} & \multicolumn{1}{c}{0.00} & \multicolumn{1}{c}{0.48} & \multicolumn{1}{c}{1257} & \multicolumn{1}{c}{0.05} & \multicolumn{1}{c}{0.07} & \multicolumn{1}{c}{0.00} & \multicolumn{1}{c}{0.62} & \multicolumn{1}{c}{1269} &       & \multicolumn{1}{c}{Finbra} & \multicolumn{1}{c}{2000} \\
    \multicolumn{1}{l}{Transfers Revenue} & \multicolumn{1}{c}{0.90} & \multicolumn{1}{c}{0.11} & \multicolumn{1}{c}{0.19} & \multicolumn{1}{c}{1.00} & \multicolumn{1}{c}{5288} &       & \multicolumn{1}{c}{0.92} & \multicolumn{1}{c}{0.10} & \multicolumn{1}{c}{0.30} & \multicolumn{1}{c}{1.00} & \multicolumn{1}{c}{1257} & \multicolumn{1}{c}{0.89} & \multicolumn{1}{c}{0.12} & \multicolumn{1}{c}{0.19} & \multicolumn{1}{c}{1.00} & \multicolumn{1}{c}{1269} &       & \multicolumn{1}{c}{Finbra} & \multicolumn{1}{c}{2000} \\
    \multicolumn{1}{l}{Other Revenues} & \multicolumn{1}{c}{0.05} & \multicolumn{1}{c}{0.07} & \multicolumn{1}{c}{0.00} & \multicolumn{1}{c}{0.77} & \multicolumn{1}{c}{5288} &       & \multicolumn{1}{c}{0.04} & \multicolumn{1}{c}{0.06} & \multicolumn{1}{c}{0.00} & \multicolumn{1}{c}{0.70} & \multicolumn{1}{c}{1257} & \multicolumn{1}{c}{0.06} & \multicolumn{1}{c}{0.08} & \multicolumn{1}{c}{0.00} & \multicolumn{1}{c}{0.77} & \multicolumn{1}{c}{1269} &       & \multicolumn{1}{c}{Finbra} & \multicolumn{1}{c}{2000} \\
          &       &       &       &       &       &       &       &       &       &       &       &       &       &       &       &       &       &       &  \\
    \multicolumn{1}{l}{\textbf{Municipality Public Spending}} &       &       &       &       &       &       &       &       &       &       &       &       &       &       &       &       &       &       &  \\
    \multicolumn{1}{l}{Total Spending per capita} & \multicolumn{1}{c}{1284.77} & \multicolumn{1}{c}{2395.06} & \multicolumn{1}{c}{129.74} & \multicolumn{1}{c}{127974.26} & \multicolumn{1}{c}{5304} &       & \multicolumn{1}{c}{1209.09} & \multicolumn{1}{c}{3627.32} & \multicolumn{1}{c}{237.93} & \multicolumn{1}{c}{127974.26} & \multicolumn{1}{c}{1258} & \multicolumn{1}{c}{1298.79} & \multicolumn{1}{c}{727.39} & \multicolumn{1}{c}{282.02} & \multicolumn{1}{c}{8527.37} & \multicolumn{1}{c}{1277} &       & \multicolumn{1}{c}{Finbra} & \multicolumn{1}{c}{2000} \\
          &       &       &       &       &       &       &       &       &       &       &       &       &       &       &       &       &       &       &  \\
    \multicolumn{1}{l}{Spending by Type - per capita} &       &       &       &       &       &       &       &       &       &       &       &       &       &       &       &       &       &       &  \\
    \multicolumn{1}{l}{Human Resources} & \multicolumn{1}{c}{502.32} & \multicolumn{1}{c}{985.60} & \multicolumn{1}{c}{0.00} & \multicolumn{1}{c}{60697.09} & \multicolumn{1}{c}{5304} &       & \multicolumn{1}{c}{456.65} & \multicolumn{1}{c}{957.40} & \multicolumn{1}{c}{0.00} & \multicolumn{1}{c}{33164.37} & \multicolumn{1}{c}{1258} & \multicolumn{1}{c}{512.71} & \multicolumn{1}{c}{310.79} & \multicolumn{1}{c}{16.96} & \multicolumn{1}{c}{3028.27} & \multicolumn{1}{c}{1277} &       & \multicolumn{1}{c}{Finbra} & \multicolumn{1}{c}{2000} \\
    \multicolumn{1}{l}{Investment} & \multicolumn{1}{c}{153.70} & \multicolumn{1}{c}{277.13} & \multicolumn{1}{c}{0.00} & \multicolumn{1}{c}{14815.46} & \multicolumn{1}{c}{5304} &       & \multicolumn{1}{c}{148.48} & \multicolumn{1}{c}{436.17} & \multicolumn{1}{c}{0.00} & \multicolumn{1}{c}{14815.46} & \multicolumn{1}{c}{1258} & \multicolumn{1}{c}{158.51} & \multicolumn{1}{c}{186.73} & \multicolumn{1}{c}{0.00} & \multicolumn{1}{c}{2737.47} & \multicolumn{1}{c}{1277} &       & \multicolumn{1}{c}{Finbra} & \multicolumn{1}{c}{2000} \\
    \multicolumn{1}{l}{Other} & \multicolumn{1}{c}{628.75} & \multicolumn{1}{c}{1280.34} & \multicolumn{1}{c}{0.00} & \multicolumn{1}{c}{79994.44} & \multicolumn{1}{c}{5304} &       & \multicolumn{1}{c}{603.96} & \multicolumn{1}{c}{2263.11} & \multicolumn{1}{c}{0.00} & \multicolumn{1}{c}{79994.44} & \multicolumn{1}{c}{1258} & \multicolumn{1}{c}{627.58} & \multicolumn{1}{c}{354.68} & \multicolumn{1}{c}{10.60} & \multicolumn{1}{c}{3339.68} & \multicolumn{1}{c}{1277} &       & \multicolumn{1}{c}{Finbra} & \multicolumn{1}{c}{2000} \\
    \multicolumn{1}{l}{Spending by Type - (\% Total Spending)} &       &       &       &       &       &       &       &       &       &       &       &       &       &       &       &       &       &       &  \\
    \multicolumn{1}{l}{Human Resources} & \multicolumn{1}{c}{0.40} & \multicolumn{1}{c}{0.10} & \multicolumn{1}{c}{0.00} & \multicolumn{1}{c}{1.00} & \multicolumn{1}{c}{5304} &       & \multicolumn{1}{c}{0.39} & \multicolumn{1}{c}{0.11} & \multicolumn{1}{c}{0.00} & \multicolumn{1}{c}{1.00} & \multicolumn{1}{c}{1258} & \multicolumn{1}{c}{0.40} & \multicolumn{1}{c}{0.10} & \multicolumn{1}{c}{0.02} & \multicolumn{1}{c}{0.92} & \multicolumn{1}{c}{1277} &       & \multicolumn{1}{c}{Finbra} & \multicolumn{1}{c}{2000} \\
    \multicolumn{1}{l}{Investment} & \multicolumn{1}{c}{0.12} & \multicolumn{1}{c}{0.07} & \multicolumn{1}{c}{0.00} & \multicolumn{1}{c}{0.71} & \multicolumn{1}{c}{5304} &       & \multicolumn{1}{c}{0.12} & \multicolumn{1}{c}{0.08} & \multicolumn{1}{c}{0.00} & \multicolumn{1}{c}{0.71} & \multicolumn{1}{c}{1258} & \multicolumn{1}{c}{0.12} & \multicolumn{1}{c}{0.07} & \multicolumn{1}{c}{0.00} & \multicolumn{1}{c}{0.54} & \multicolumn{1}{c}{1277} &       & \multicolumn{1}{c}{Finbra} & \multicolumn{1}{c}{2000} \\
    \multicolumn{1}{l}{Other} & \multicolumn{1}{c}{0.49} & \multicolumn{1}{c}{0.11} & \multicolumn{1}{c}{0.00} & \multicolumn{1}{c}{1.00} & \multicolumn{1}{c}{5304} &       & \multicolumn{1}{c}{0.49} & \multicolumn{1}{c}{0.11} & \multicolumn{1}{c}{0.00} & \multicolumn{1}{c}{0.94} & \multicolumn{1}{c}{1258} & \multicolumn{1}{c}{0.49} & \multicolumn{1}{c}{0.10} & \multicolumn{1}{c}{0.00} & \multicolumn{1}{c}{0.89} & \multicolumn{1}{c}{1277} &       & \multicolumn{1}{c}{Finbra} & \multicolumn{1}{c}{2000} \\
          &       &       &       &       &       &       &       &       &       &       &       &       &       &       &       &       &       &       &  \\
    \multicolumn{1}{l}{Spending by Category - per capita} &       &       &       &       &       &       &       &       &       &       &       &       &       &       &       &       &       &       &  \\
    \multicolumn{1}{l}{Health and Sanitation} & \multicolumn{1}{c}{217.08} & \multicolumn{1}{c}{276.14} & \multicolumn{1}{c}{0.04} & \multicolumn{1}{c}{12559.61} & \multicolumn{1}{c}{5286} &       & \multicolumn{1}{c}{161.53} & \multicolumn{1}{c}{358.67} & \multicolumn{1}{c}{0.99} & \multicolumn{1}{c}{12471.29} & \multicolumn{1}{c}{1252} & \multicolumn{1}{c}{274.57} & \multicolumn{1}{c}{167.33} & \multicolumn{1}{c}{6.96} & \multicolumn{1}{c}{1736.83} & \multicolumn{1}{c}{1275} &       & \multicolumn{1}{c}{Finbra} & \multicolumn{1}{c}{2000} \\
    \multicolumn{1}{l}{Transport} & \multicolumn{1}{c}{91.55} & \multicolumn{1}{c}{138.28} & \multicolumn{1}{c}{0.00} & \multicolumn{1}{c}{5865.79} & \multicolumn{1}{c}{5304} &       & \multicolumn{1}{c}{89.77} & \multicolumn{1}{c}{123.32} & \multicolumn{1}{c}{0.00} & \multicolumn{1}{c}{1146.75} & \multicolumn{1}{c}{1258} & \multicolumn{1}{c}{77.51} & \multicolumn{1}{c}{95.58} & \multicolumn{1}{c}{0.00} & \multicolumn{1}{c}{1030.91} & \multicolumn{1}{c}{1277} &       & \multicolumn{1}{c}{Finbra} & \multicolumn{1}{c}{2000} \\
    \multicolumn{1}{l}{Education and Culture} & \multicolumn{1}{c}{419.95} & \multicolumn{1}{c}{640.29} & \multicolumn{1}{c}{0.00} & \multicolumn{1}{c}{36319.15} & \multicolumn{1}{c}{5304} &       & \multicolumn{1}{c}{399.03} & \multicolumn{1}{c}{1032.67} & \multicolumn{1}{c}{0.00} & \multicolumn{1}{c}{36319.15} & \multicolumn{1}{c}{1258} & \multicolumn{1}{c}{422.82} & \multicolumn{1}{c}{214.62} & \multicolumn{1}{c}{6.97} & \multicolumn{1}{c}{1811.43} & \multicolumn{1}{c}{1277} &       & \multicolumn{1}{c}{Finbra} & \multicolumn{1}{c}{2000} \\
    \multicolumn{1}{l}{Housing and Urban} & \multicolumn{1}{c}{116.05} & \multicolumn{1}{c}{301.25} & \multicolumn{1}{c}{0.00} & \multicolumn{1}{c}{19842.15} & \multicolumn{1}{c}{5304} &       & \multicolumn{1}{c}{118.96} & \multicolumn{1}{c}{565.01} & \multicolumn{1}{c}{0.00} & \multicolumn{1}{c}{19842.15} & \multicolumn{1}{c}{1258} & \multicolumn{1}{c}{120.68} & \multicolumn{1}{c}{115.52} & \multicolumn{1}{c}{0.00} & \multicolumn{1}{c}{1207.96} & \multicolumn{1}{c}{1277} &       & \multicolumn{1}{c}{Finbra} & \multicolumn{1}{c}{2000} \\
    \multicolumn{1}{l}{Social Assistance} & \multicolumn{1}{c}{84.05} & \multicolumn{1}{c}{253.84} & \multicolumn{1}{c}{0.00} & \multicolumn{1}{c}{13814.63} & \multicolumn{1}{c}{5304} &       & \multicolumn{1}{c}{83.29} & \multicolumn{1}{c}{393.78} & \multicolumn{1}{c}{0.00} & \multicolumn{1}{c}{13814.63} & \multicolumn{1}{c}{1258} & \multicolumn{1}{c}{79.22} & \multicolumn{1}{c}{71.73} & \multicolumn{1}{c}{0.00} & \multicolumn{1}{c}{870.44} & \multicolumn{1}{c}{1277} &       & \multicolumn{1}{c}{Finbra} & \multicolumn{1}{c}{2000} \\
    \multicolumn{1}{l}{Other Categories} & \multicolumn{1}{c}{472.88} & \multicolumn{1}{c}{1201.13} & \multicolumn{1}{c}{32.00} & \multicolumn{1}{c}{65369.18} & \multicolumn{1}{c}{5304} &       & \multicolumn{1}{c}{476.25} & \multicolumn{1}{c}{1851.75} & \multicolumn{1}{c}{85.06} & \multicolumn{1}{c}{65369.18} & \multicolumn{1}{c}{1258} & \multicolumn{1}{c}{445.10} & \multicolumn{1}{c}{323.86} & \multicolumn{1}{c}{86.34} & \multicolumn{1}{c}{3698.10} & \multicolumn{1}{c}{1277} &       & \multicolumn{1}{c}{Finbra} & \multicolumn{1}{c}{2000} \\
    \multicolumn{1}{l}{Spending by Category - (\% Total Spending)} &       &       &       &       &       &       &       &       &       &       &       &       &       &       &       &       &       &       &  \\
    \multicolumn{1}{l}{Health and Sanitation} & \multicolumn{1}{c}{0.17} & \multicolumn{1}{c}{0.07} & \multicolumn{1}{c}{0.00} & \multicolumn{1}{c}{0.53} & \multicolumn{1}{c}{5304} &       & \multicolumn{1}{c}{0.14} & \multicolumn{1}{c}{0.06} & \multicolumn{1}{c}{0.00} & \multicolumn{1}{c}{0.46} & \multicolumn{1}{c}{1258} & \multicolumn{1}{c}{0.21} & \multicolumn{1}{c}{0.06} & \multicolumn{1}{c}{0.00} & \multicolumn{1}{c}{0.47} & \multicolumn{1}{c}{1277} &       & \multicolumn{1}{c}{Finbra} & \multicolumn{1}{c}{2000} \\
    \multicolumn{1}{l}{Transport Spending} & \multicolumn{1}{c}{0.07} & \multicolumn{1}{c}{0.06} & \multicolumn{1}{c}{0.00} & \multicolumn{1}{c}{0.37} & \multicolumn{1}{c}{5304} &       & \multicolumn{1}{c}{0.07} & \multicolumn{1}{c}{0.07} & \multicolumn{1}{c}{0.00} & \multicolumn{1}{c}{0.37} & \multicolumn{1}{c}{1258} & \multicolumn{1}{c}{0.06} & \multicolumn{1}{c}{0.05} & \multicolumn{1}{c}{0.00} & \multicolumn{1}{c}{0.32} & \multicolumn{1}{c}{1277} &       & \multicolumn{1}{c}{Finbra} & \multicolumn{1}{c}{2000} \\
    \multicolumn{1}{l}{Education and Culture Spending} & \multicolumn{1}{c}{0.34} & \multicolumn{1}{c}{0.08} & \multicolumn{1}{c}{0.00} & \multicolumn{1}{c}{0.67} & \multicolumn{1}{c}{5304} &       & \multicolumn{1}{c}{0.34} & \multicolumn{1}{c}{0.08} & \multicolumn{1}{c}{0.00} & \multicolumn{1}{c}{0.64} & \multicolumn{1}{c}{1258} & \multicolumn{1}{c}{0.34} & \multicolumn{1}{c}{0.08} & \multicolumn{1}{c}{0.01} & \multicolumn{1}{c}{0.61} & \multicolumn{1}{c}{1277} &       & \multicolumn{1}{c}{Finbra} & \multicolumn{1}{c}{2000} \\
    \multicolumn{1}{l}{Housing and Urban Spending} & \multicolumn{1}{c}{0.09} & \multicolumn{1}{c}{0.06} & \multicolumn{1}{c}{0.00} & \multicolumn{1}{c}{0.63} & \multicolumn{1}{c}{5304} &       & \multicolumn{1}{c}{0.10} & \multicolumn{1}{c}{0.07} & \multicolumn{1}{c}{0.00} & \multicolumn{1}{c}{0.59} & \multicolumn{1}{c}{1258} & \multicolumn{1}{c}{0.09} & \multicolumn{1}{c}{0.06} & \multicolumn{1}{c}{0.00} & \multicolumn{1}{c}{0.33} & \multicolumn{1}{c}{1277} &       & \multicolumn{1}{c}{Finbra} & \multicolumn{1}{c}{2000} \\
    \multicolumn{1}{l}{Social Assistance Spending} & \multicolumn{1}{c}{0.06} & \multicolumn{1}{c}{0.04} & \multicolumn{1}{c}{0.00} & \multicolumn{1}{c}{0.33} & \multicolumn{1}{c}{5304} &       & \multicolumn{1}{c}{0.07} & \multicolumn{1}{c}{0.04} & \multicolumn{1}{c}{0.00} & \multicolumn{1}{c}{0.29} & \multicolumn{1}{c}{1258} & \multicolumn{1}{c}{0.06} & \multicolumn{1}{c}{0.04} & \multicolumn{1}{c}{0.00} & \multicolumn{1}{c}{0.33} & \multicolumn{1}{c}{1277} &       & \multicolumn{1}{c}{Finbra} & \multicolumn{1}{c}{2000} \\
    \multicolumn{1}{l}{Spending in Other Areas} & \multicolumn{1}{c}{0.36} & \multicolumn{1}{c}{0.09} & \multicolumn{1}{c}{0.12} & \multicolumn{1}{c}{1.00} & \multicolumn{1}{c}{5304} &       & \multicolumn{1}{c}{0.38} & \multicolumn{1}{c}{0.10} & \multicolumn{1}{c}{0.15} & \multicolumn{1}{c}{1.00} & \multicolumn{1}{c}{1258} & \multicolumn{1}{c}{0.33} & \multicolumn{1}{c}{0.08} & \multicolumn{1}{c}{0.12} & \multicolumn{1}{c}{0.79} & \multicolumn{1}{c}{1277} &       & \multicolumn{1}{c}{Finbra} & \multicolumn{1}{c}{2000} \\
          &       &       &       &       &       &       &       &       &       &       &       &       &       &       &       &       &       &       &  \\
    \midrule
    \midrule
          &       &       &       &       &       &       &       &       &       &       &       &       &       &       &       &       &       &       &  \\
    \end{tabular}%
    
  \label{table:stats}%

\end{threeparttable}
}
\end{center}
\end{footnotesize}
\end{table}
\end{sidewaystable}
\begin{table}[H]
\begin{footnotesize}
\begin{center}
\scalebox{0.75}{
\begin{threeparttable}[b]


  \centering
  \caption*{Table A.1: Descriptive Statistics (at the baseline year) -- \emph{Cont.}}
  
  \begin{tabular}{rrrrrrrr}
          &       &       &       &       &       &       &  \\
    \midrule
    \midrule
          & \multicolumn{1}{c}{Mean} & \multicolumn{1}{c}{Std. Dev.} & \multicolumn{1}{c}{Min} & \multicolumn{1}{c}{Max} & \multicolumn{1}{c}{Obs.} &       & \multicolumn{1}{c}{Source of Data} \\
\cmidrule{1-6}\cmidrule{8-8}    \multicolumn{1}{l}{\textbf{Primary Care Coverage}} &       &       &       &       &       &       &  \\
    \multicolumn{1}{l}{Extensive Margin} &       &       &       &       &       &       &  \\
    \multicolumn{1}{l}{Population covered (share) by Community Health Agents} & \multicolumn{1}{c}{0.635} & \multicolumn{1}{c}{0.409} & \multicolumn{1}{c}{0} & \multicolumn{1}{c}{1} & \multicolumn{1}{c}{5507} &       & \multicolumn{1}{c}{Datasus/SIAB} \\
    \multicolumn{1}{l}{Population covered (share) by Family Health Agents} & \multicolumn{1}{c}{0.311} & \multicolumn{1}{c}{0.383} & \multicolumn{1}{c}{0} & \multicolumn{1}{c}{1} & \multicolumn{1}{c}{5507} &       & \multicolumn{1}{c}{Datasus/SIAB} \\
          &       &       &       &       &       &       &  \\
    \multicolumn{1}{l}{Intensive Margin} &       &       &       &       &       &       &  \\
    \multicolumn{1}{l}{N. of People Visited by Primary Care Agents (per capita)} & \multicolumn{1}{c}{0.271} & \multicolumn{1}{c}{0.285} & \multicolumn{1}{c}{0} & \multicolumn{1}{c}{2.798} & \multicolumn{1}{c}{5507} &       & \multicolumn{1}{c}{Datasus/SIAB} \\
    \multicolumn{1}{l}{N. of People Visited by Community Health Agents (per capita)} & \multicolumn{1}{c}{0.121} & \multicolumn{1}{c}{0.18} & \multicolumn{1}{c}{0} & \multicolumn{1}{c}{1.518} & \multicolumn{1}{c}{5507} &       & \multicolumn{1}{c}{Datasus/SIAB} \\
    \multicolumn{1}{l}{N. of People Visited by Family Health Agents (per capita)} & \multicolumn{1}{c}{0.15} & \multicolumn{1}{c}{0.252} & \multicolumn{1}{c}{0} & \multicolumn{1}{c}{1.834} & \multicolumn{1}{c}{5507} &       & \multicolumn{1}{c}{Datasus/SIAB} \\
    \multicolumn{1}{l}{N. of Household Visits \& Appointments (per capita)} & \multicolumn{1}{c}{1.876} & \multicolumn{1}{c}{2.541} & \multicolumn{1}{c}{0} & \multicolumn{1}{c}{88.85} & \multicolumn{1}{c}{5507} &       & \multicolumn{1}{c}{Datasus/SIAB} \\
    \multicolumn{1}{l}{N. of Household Visits \& Appointments by Community Health Agents (per capita)} & \multicolumn{1}{c}{1.072} & \multicolumn{1}{c}{2.156} & \multicolumn{1}{c}{0} & \multicolumn{1}{c}{85.989} & \multicolumn{1}{c}{5507} &       & \multicolumn{1}{c}{Datasus/SIAB} \\
    \multicolumn{1}{l}{N. of Household Visits \& Appointments by Family Health Agents (per capita)} & \multicolumn{1}{c}{0.8} & \multicolumn{1}{c}{1.505} & \multicolumn{1}{c}{0} & \multicolumn{1}{c}{43.389} & \multicolumn{1}{c}{5507} &       & \multicolumn{1}{c}{Datasus/SIAB} \\
          &       &       &       &       &       &       &  \\
    \multicolumn{1}{l}{\textbf{Health Human Resources}} &       &       &       &       &       &       &  \\
    \multicolumn{1}{l}{N. of Health Professionals (per capita*1000)} & \multicolumn{1}{c}{5.104} & \multicolumn{1}{c}{4.825} & \multicolumn{1}{c}{0} & \multicolumn{1}{c}{187.904} & \multicolumn{1}{c}{5507} &       & \multicolumn{1}{c}{IBGE/AMS} \\
    \multicolumn{1}{l}{N. of Doctors (per capita*1000)} & \multicolumn{1}{c}{1.529} & \multicolumn{1}{c}{2.385} & \multicolumn{1}{c}{0} & \multicolumn{1}{c}{95.132} & \multicolumn{1}{c}{5507} &       & \multicolumn{1}{c}{IBGE/AMS} \\
    \multicolumn{1}{l}{N. of Nurses (per capita*1000)} & \multicolumn{1}{c}{1.159} & \multicolumn{1}{c}{1.636} & \multicolumn{1}{c}{0} & \multicolumn{1}{c}{95.097} & \multicolumn{1}{c}{5507} &       & \multicolumn{1}{c}{IBGE/AMS} \\
    \multicolumn{1}{l}{N. of Nursing Assistants (per capita*1000)} & \multicolumn{1}{c}{1.26} & \multicolumn{1}{c}{1.456} & \multicolumn{1}{c}{0} & \multicolumn{1}{c}{22.009} & \multicolumn{1}{c}{5507} &       & \multicolumn{1}{c}{IBGE/AMS} \\
    \multicolumn{1}{l}{N. of Administrative Professionals (per capita*1000)} & \multicolumn{1}{c}{1.155} & \multicolumn{1}{c}{1.251} & \multicolumn{1}{c}{0} & \multicolumn{1}{c}{36.599} & \multicolumn{1}{c}{5507} &       & \multicolumn{1}{c}{IBGE/AMS} \\
          &       &       &       &       &       &       &  \\
    \multicolumn{1}{l}{\textbf{Health Infrastructure}} &       &       &       &       &       &       &  \\
    \multicolumn{1}{l}{N. of Municipal Hospitals (per capita*1000)} & \multicolumn{1}{c}{0.06} & \multicolumn{1}{c}{0.138} & \multicolumn{1}{c}{0} & \multicolumn{1}{c}{1.357} & \multicolumn{1}{c}{5507} &       & \multicolumn{1}{c}{IBGE/AMS} \\
    \multicolumn{1}{l}{N. of Federal and State Hospitals (per capita*1000)} & \multicolumn{1}{c}{0.015} & \multicolumn{1}{c}{0.084} & \multicolumn{1}{c}{0} & \multicolumn{1}{c}{1.892} & \multicolumn{1}{c}{5507} &       & \multicolumn{1}{c}{IBGE/AMS} \\
    \multicolumn{1}{l}{N. of Private Hospitals (per capita*1000)} & \multicolumn{1}{c}{0.03} & \multicolumn{1}{c}{0.058} & \multicolumn{1}{c}{0} & \multicolumn{1}{c}{0.609} & \multicolumn{1}{c}{5507} &       & \multicolumn{1}{c}{IBGE/AMS} \\
    \multicolumn{1}{l}{N. of Health Facilities (per capita*1000) with Ambulatory Service} & \multicolumn{1}{c}{0.517} & \multicolumn{1}{c}{0.355} & \multicolumn{1}{c}{0} & \multicolumn{1}{c}{3.628} & \multicolumn{1}{c}{5493} &       & \multicolumn{1}{c}{Datasus/SIA} \\
          &       &       &       &       &       &       &  \\
    \multicolumn{1}{l}{\textbf{Primary Care Related Infrastructure and Human Resources}} &       &       &       &       &       &       &  \\
    \multicolumn{1}{l}{Number of Health Facilities (per capita * 1000) with} &       &       &       &       &       &       &  \\
    \multicolumn{1}{l}{Ambulatory Service and ACS Teams} & \multicolumn{1}{c}{0.14} & \multicolumn{1}{c}{0.197} & \multicolumn{1}{c}{0} & \multicolumn{1}{c}{2.41} & \multicolumn{1}{c}{5493} &       & \multicolumn{1}{c}{Datasus/SIA} \\
    \multicolumn{1}{l}{Ambulatory Service and Community Doctors} & \multicolumn{1}{c}{0.082} & \multicolumn{1}{c}{0.154} & \multicolumn{1}{c}{0} & \multicolumn{1}{c}{1.957} & \multicolumn{1}{c}{5493} &       & \multicolumn{1}{c}{Datasus/SIA} \\
    \multicolumn{1}{l}{Ambulatory Service and ACS Nurses} & \multicolumn{1}{c}{0.072} & \multicolumn{1}{c}{0.156} & \multicolumn{1}{c}{0} & \multicolumn{1}{c}{2.41} & \multicolumn{1}{c}{5493} &       & \multicolumn{1}{c}{Datasus/SIA} \\
    \multicolumn{1}{l}{Ambulatory Service and PSF Teams} & \multicolumn{1}{c}{0.083} & \multicolumn{1}{c}{0.159} & \multicolumn{1}{c}{0} & \multicolumn{1}{c}{2.41} & \multicolumn{1}{c}{5493} &       & \multicolumn{1}{c}{Datasus/SIA} \\
    \multicolumn{1}{l}{Ambulatory Service and PSF Doctors} & \multicolumn{1}{c}{0.077} & \multicolumn{1}{c}{0.149} & \multicolumn{1}{c}{0} & \multicolumn{1}{c}{1.957} & \multicolumn{1}{c}{5493} &       & \multicolumn{1}{c}{Datasus/SIA} \\
    \multicolumn{1}{l}{Ambulatory Service and PSF Nurses} & \multicolumn{1}{c}{0.075} & \multicolumn{1}{c}{0.149} & \multicolumn{1}{c}{0} & \multicolumn{1}{c}{2.41} & \multicolumn{1}{c}{5493} &       & \multicolumn{1}{c}{Datasus/SIA} \\
    \multicolumn{1}{l}{Ambulatory Service and PSF Nursing Assistants} & \multicolumn{1}{c}{0.05} & \multicolumn{1}{c}{0.123} & \multicolumn{1}{c}{0} & \multicolumn{1}{c}{1.957} & \multicolumn{1}{c}{5493} &       & \multicolumn{1}{c}{Datasus/SIA} \\
          &       &       &       &       &       &       &  \\
    \multicolumn{1}{l}{\textbf{Access to Health Services}} &       &       &       &       &       &       &  \\
    \multicolumn{1}{l}{Prenatal Ignored} & \multicolumn{1}{c}{0.044} & \multicolumn{1}{c}{0.094} & \multicolumn{1}{c}{0} & \multicolumn{1}{c}{1} & \multicolumn{1}{c}{5460} &       &  \\
    \multicolumn{1}{l}{Prenatal Visits None} & \multicolumn{1}{c}{0.053} & \multicolumn{1}{c}{0.077} & \multicolumn{1}{c}{0} & \multicolumn{1}{c}{0.921} & \multicolumn{1}{c}{5437} &       & \multicolumn{1}{c}{Datasus/SINASC} \\
    \multicolumn{1}{l}{Prenatal Visits 1-6} & \multicolumn{1}{c}{0.53} & \multicolumn{1}{c}{0.216} & \multicolumn{1}{c}{0} & \multicolumn{1}{c}{1} & \multicolumn{1}{c}{5507} &       & \multicolumn{1}{c}{Datasus/SINASC} \\
    \multicolumn{1}{l}{Prenatal Visits 7+} & \multicolumn{1}{c}{0.375} & \multicolumn{1}{c}{0.235} & \multicolumn{1}{c}{0} & \multicolumn{1}{c}{1} & \multicolumn{1}{c}{5507} &       & \multicolumn{1}{c}{Datasus/SINASC} \\
          &       &       &       &       &       &       &  \\
    \multicolumn{1}{l}{\textbf{Ambulatorial Production}} &       &       &       &       &       &       &  \\
    \multicolumn{1}{l}{N. Outpatient Procedures (per capita)} & \multicolumn{1}{c}{8.8} & \multicolumn{1}{c}{4.55} & \multicolumn{1}{c}{0} & \multicolumn{1}{c}{48.258} & \multicolumn{1}{c}{5507} &       & \multicolumn{1}{c}{Datasus/SIA} \\
    \multicolumn{1}{l}{N. Primary Care Outpatient Procedures (per capita)} & \multicolumn{1}{c}{7.415} & \multicolumn{1}{c}{3.974} & \multicolumn{1}{c}{0} & \multicolumn{1}{c}{39.367} & \multicolumn{1}{c}{5507} &       & \multicolumn{1}{c}{Datasus/SIA} \\
    \multicolumn{1}{l}{N. Low \& Mid Complexity Outpatient Procedures (per capita)} & \multicolumn{1}{c}{9.467} & \multicolumn{1}{c}{5.801} & \multicolumn{1}{c}{0} & \multicolumn{1}{c}{171.126} & \multicolumn{1}{c}{5493} &       & \multicolumn{1}{c}{Datasus/SIA} \\
    \multicolumn{1}{l}{N. High Complexity Outpatient Procedures (per capita)} & \multicolumn{1}{c}{0.005} & \multicolumn{1}{c}{0.052} & \multicolumn{1}{c}{0} & \multicolumn{1}{c}{2.58} & \multicolumn{1}{c}{5493} &       & \multicolumn{1}{c}{Datasus/SIA} \\
          &       &       &       &       &       &       &  \\
    \bottomrule
    \bottomrule
    \end{tabular}%
    



\end{threeparttable}
}
\end{center}
\end{footnotesize}
\end{table}
\begin{table}[H]
\begin{footnotesize}
\begin{center}
\scalebox{0.8}{
\begin{threeparttable}[b]


  \centering
  \caption*{Table A.1: Descriptive Statistics (at the baseline year) -- \emph{Cont.}}
  
  \begin{tabular}{rrrrrrrr}
          &       &       &       &       &       &       &  \\
    \midrule
    \midrule
          & \multicolumn{1}{c}{Mean} & \multicolumn{1}{c}{Std. Dev.} & \multicolumn{1}{c}{Min} & \multicolumn{1}{c}{Max} & \multicolumn{1}{c}{Obs.} &       & \multicolumn{1}{c}{Source of Data} \\
\cmidrule{1-6}\cmidrule{8-8}    \multicolumn{1}{l}{\textbf{Infant Mortality Rate}} &       &       &       &       &       &       &  \\
    \multicolumn{1}{l}{Total} & \multicolumn{1}{c}{23.07} & \multicolumn{1}{c}{26.16} & \multicolumn{1}{c}{0.00} & \multicolumn{1}{c}{1000.00} & \multicolumn{1}{c}{5507} &       & \multicolumn{1}{c}{Datasus/SIM} \\
    \multicolumn{1}{l}{APC} & \multicolumn{1}{c}{2.10} & \multicolumn{1}{c}{7.10} & \multicolumn{1}{c}{0.00} & \multicolumn{1}{c}{333.33} & \multicolumn{1}{c}{5507} &       & \multicolumn{1}{c}{Datasus/SIM} \\
    \multicolumn{1}{l}{non-APC} & \multicolumn{1}{c}{20.97} & \multicolumn{1}{c}{22.29} & \multicolumn{1}{c}{0.00} & \multicolumn{1}{c}{666.67} & \multicolumn{1}{c}{5507} &       & \multicolumn{1}{c}{Datasus/SIM} \\
    \multicolumn{1}{l}{Fetal} & \multicolumn{1}{c}{0.00} & \multicolumn{1}{c}{0.08} & \multicolumn{1}{c}{0.00} & \multicolumn{1}{c}{3.57} & \multicolumn{1}{c}{5507} &       & \multicolumn{1}{c}{Datasus/SIM} \\
    \multicolumn{1}{l}{Within 24h} & \multicolumn{1}{c}{5.55} & \multicolumn{1}{c}{10.15} & \multicolumn{1}{c}{0.00} & \multicolumn{1}{c}{333.33} & \multicolumn{1}{c}{5507} &       & \multicolumn{1}{c}{Datasus/SIM} \\
    \multicolumn{1}{l}{1 to 27 days} & \multicolumn{1}{c}{13.73} & \multicolumn{1}{c}{15.89} & \multicolumn{1}{c}{0.00} & \multicolumn{1}{c}{333.33} & \multicolumn{1}{c}{5507} &       & \multicolumn{1}{c}{Datasus/SIM} \\
    \multicolumn{1}{l}{27 days to 1 year} & \multicolumn{1}{c}{9.34} & \multicolumn{1}{c}{16.34} & \multicolumn{1}{c}{0.00} & \multicolumn{1}{c}{666.67} & \multicolumn{1}{c}{5507} &       & \multicolumn{1}{c}{Datasus/SIM} \\
    \multicolumn{1}{l}{Infectious} & \multicolumn{1}{c}{2.00} & \multicolumn{1}{c}{7.03} & \multicolumn{1}{c}{0.00} & \multicolumn{1}{c}{333.33} & \multicolumn{1}{c}{5507} &       & \multicolumn{1}{c}{Datasus/SIM} \\
    \multicolumn{1}{l}{Respiratory} & \multicolumn{1}{c}{1.52} & \multicolumn{1}{c}{4.45} & \multicolumn{1}{c}{0.00} & \multicolumn{1}{c}{142.86} & \multicolumn{1}{c}{5507} &       & \multicolumn{1}{c}{Datasus/SIM} \\
    \multicolumn{1}{l}{Perinatal} & \multicolumn{1}{c}{11.04} & \multicolumn{1}{c}{16.32} & \multicolumn{1}{c}{0.00} & \multicolumn{1}{c}{666.67} & \multicolumn{1}{c}{5507} &       & \multicolumn{1}{c}{Datasus/SIM} \\
    \multicolumn{1}{l}{Congenital} & \multicolumn{1}{c}{2.13} & \multicolumn{1}{c}{5.01} & \multicolumn{1}{c}{0.00} & \multicolumn{1}{c}{93.02} & \multicolumn{1}{c}{5507} &       & \multicolumn{1}{c}{Datasus/SIM} \\
    \multicolumn{1}{l}{External} & \multicolumn{1}{c}{0.37} & \multicolumn{1}{c}{1.91} & \multicolumn{1}{c}{0.00} & \multicolumn{1}{c}{43.48} & \multicolumn{1}{c}{5507} &       & \multicolumn{1}{c}{Datasus/SIM} \\
    \multicolumn{1}{l}{Nutritional} & \multicolumn{1}{c}{0.60} & \multicolumn{1}{c}{3.22} & \multicolumn{1}{c}{0.00} & \multicolumn{1}{c}{166.67} & \multicolumn{1}{c}{5507} &       & \multicolumn{1}{c}{Datasus/SIM} \\
    \multicolumn{1}{l}{Other} & \multicolumn{1}{c}{0.87} & \multicolumn{1}{c}{3.60} & \multicolumn{1}{c}{0.00} & \multicolumn{1}{c}{142.86} & \multicolumn{1}{c}{5507} &       & \multicolumn{1}{c}{Datasus/SIM} \\
    \multicolumn{1}{l}{Ill-Defined} & \multicolumn{1}{c}{4.55} & \multicolumn{1}{c}{10.68} & \multicolumn{1}{c}{0.00} & \multicolumn{1}{c}{142.86} & \multicolumn{1}{c}{5507} &       & \multicolumn{1}{c}{Datasus/SIM} \\
          &       &       &       &       &       &       &  \\
    \multicolumn{1}{l}{\textbf{Fertility}} &       &       &       &       &       &       &  \\
    \multicolumn{1}{l}{Rates of Birth per Woman (10-49y)} & \multicolumn{1}{c}{0.06} & \multicolumn{1}{c}{0.02} & \multicolumn{1}{c}{0.00} & \multicolumn{1}{c}{0.17} & \multicolumn{1}{c}{5507} &       & \multicolumn{1}{c}{Datasus/SINASC} \\
          &       &       &       &       &       &       &  \\
    \multicolumn{1}{l}{\textbf{Birth Oucomes}} &       &       &       &       &       &       &  \\
    \multicolumn{1}{l}{Apgar 1} & \multicolumn{1}{c}{8.18} & \multicolumn{1}{c}{0.90} & \multicolumn{1}{c}{1.00} & \multicolumn{1}{c}{9.00} & \multicolumn{1}{c}{5428} &       & \multicolumn{1}{c}{Datasus/SINASC} \\
    \multicolumn{1}{l}{Apgar 5} & \multicolumn{1}{c}{8.66} & \multicolumn{1}{c}{0.89} & \multicolumn{1}{c}{1.00} & \multicolumn{1}{c}{9.00} & \multicolumn{1}{c}{5082} &       & \multicolumn{1}{c}{Datasus/SINASC} \\
    \multicolumn{1}{l}{Low Birth Weight (<2.5k)} & \multicolumn{1}{c}{0.07} & \multicolumn{1}{c}{0.03} & \multicolumn{1}{c}{0.00} & \multicolumn{1}{c}{0.50} & \multicolumn{1}{c}{5507} &       & \multicolumn{1}{c}{Datasus/SINASC} \\
    \multicolumn{1}{l}{Premature Birth} & \multicolumn{1}{c}{0.09} & \multicolumn{1}{c}{0.11} & \multicolumn{1}{c}{0.00} & \multicolumn{1}{c}{1.00} & \multicolumn{1}{c}{5507} &       & \multicolumn{1}{c}{Datasus/SINASC} \\
    \multicolumn{1}{l}{Sex Ratio at Birth} & \multicolumn{1}{c}{1.07} & \multicolumn{1}{c}{0.25} & \multicolumn{1}{c}{0.15} & \multicolumn{1}{c}{5.00} & \multicolumn{1}{c}{5505} &       & \multicolumn{1}{c}{Datasus/SINASC} \\
          &       &       &       &       &       &       &  \\
    \midrule
    \midrule
          &       &       &       &       &       &       &  \\
    \end{tabular}%
    



\end{threeparttable}
}
\end{center}
\end{footnotesize}
\end{table}

\begin{sidewaystable}
\begin{table}[H]
\begin{footnotesize}
\begin{center}
\scalebox{0.6}{
\begin{threeparttable}[b]


  \centering
  \caption*{Table 1: Descriptive Statistics (at the baseline year) -- \emph{Cont.}}
  
  \begin{tabular}{rrrrrrrrrrrrrrrrrrrr}
          &       &       &       &       &       &       &       &       &       &       &       &       &       &       &       &       &       &       &  \\
    \midrule
    \midrule
          & \multicolumn{5}{c}{Full Sample}       &       & \multicolumn{5}{c}{Bottom Quartile of OR Spent in Public Health} & \multicolumn{5}{c}{Top Quartile of OR Spent in Public Health} &       &       &  \\
\cmidrule{2-6}\cmidrule{8-17}          & \multicolumn{1}{c}{Mean} & \multicolumn{1}{c}{Std. Dev.} & \multicolumn{1}{c}{Min} & \multicolumn{1}{c}{Max} & \multicolumn{1}{c}{Obs.} &       & \multicolumn{1}{c}{Mean} & \multicolumn{1}{c}{Std. Dev.} & \multicolumn{1}{c}{Min} & \multicolumn{1}{c}{Max} & \multicolumn{1}{c}{Obs.} & \multicolumn{1}{c}{Mean} & \multicolumn{1}{c}{Std. Dev.} & \multicolumn{1}{c}{Min} & \multicolumn{1}{c}{Max} & \multicolumn{1}{c}{Obs.} &       & \multicolumn{1}{c}{Source of Data} & \multicolumn{1}{c}{Baseline Year} \\
\cmidrule{1-6}\cmidrule{8-17}\cmidrule{19-20}    \multicolumn{1}{l}{\textbf{Infant Mortality Rate}} &       &       &       &       &       &       &       &       &       &       &       &       &       &       &       &       &       &       &  \\
    \multicolumn{1}{l}{Total} & \multicolumn{1}{c}{23.07} & \multicolumn{1}{c}{26.16} & \multicolumn{1}{c}{0.00} & \multicolumn{1}{c}{1000.00} & \multicolumn{1}{c}{5507} &       & \multicolumn{1}{c}{23.46} & \multicolumn{1}{c}{19.05} & \multicolumn{1}{c}{0.00} & \multicolumn{1}{c}{166.67} & \multicolumn{1}{c}{1306} & \multicolumn{1}{c}{22.84} & \multicolumn{1}{c}{32.80} & \multicolumn{1}{c}{0.00} & \multicolumn{1}{c}{1000.00} & \multicolumn{1}{c}{1306} &       & \multicolumn{1}{c}{Datasus/SIM} & \multicolumn{1}{c}{2000} \\
    \multicolumn{1}{l}{APC} & \multicolumn{1}{c}{2.10} & \multicolumn{1}{c}{7.10} & \multicolumn{1}{c}{0.00} & \multicolumn{1}{c}{333.33} & \multicolumn{1}{c}{5507} &       & \multicolumn{1}{c}{2.06} & \multicolumn{1}{c}{5.50} & \multicolumn{1}{c}{0.00} & \multicolumn{1}{c}{142.86} & \multicolumn{1}{c}{1306} & \multicolumn{1}{c}{2.18} & \multicolumn{1}{c}{10.30} & \multicolumn{1}{c}{0.00} & \multicolumn{1}{c}{333.33} & \multicolumn{1}{c}{1306} &       & \multicolumn{1}{c}{Datasus/SIM} & \multicolumn{1}{c}{2000} \\
    \multicolumn{1}{l}{non-APC} & \multicolumn{1}{c}{20.97} & \multicolumn{1}{c}{22.29} & \multicolumn{1}{c}{0.00} & \multicolumn{1}{c}{666.67} & \multicolumn{1}{c}{5507} &       & \multicolumn{1}{c}{21.40} & \multicolumn{1}{c}{17.83} & \multicolumn{1}{c}{0.00} & \multicolumn{1}{c}{166.67} & \multicolumn{1}{c}{1306} & \multicolumn{1}{c}{20.66} & \multicolumn{1}{c}{24.73} & \multicolumn{1}{c}{0.00} & \multicolumn{1}{c}{666.67} & \multicolumn{1}{c}{1306} &       & \multicolumn{1}{c}{Datasus/SIM} & \multicolumn{1}{c}{2000} \\
    \multicolumn{1}{l}{Fetal} & \multicolumn{1}{c}{0.00} & \multicolumn{1}{c}{0.08} & \multicolumn{1}{c}{0.00} & \multicolumn{1}{c}{3.57} & \multicolumn{1}{c}{5507} &       & \multicolumn{1}{c}{0.01} & \multicolumn{1}{c}{0.09} & \multicolumn{1}{c}{0.00} & \multicolumn{1}{c}{2.21} & \multicolumn{1}{c}{1306} & \multicolumn{1}{c}{0.00} & \multicolumn{1}{c}{0.08} & \multicolumn{1}{c}{0.00} & \multicolumn{1}{c}{2.87} & \multicolumn{1}{c}{1306} &       & \multicolumn{1}{c}{Datasus/SIM} & \multicolumn{1}{c}{2000} \\
    \multicolumn{1}{l}{Within 24h} & \multicolumn{1}{c}{5.55} & \multicolumn{1}{c}{10.15} & \multicolumn{1}{c}{0.00} & \multicolumn{1}{c}{333.33} & \multicolumn{1}{c}{5507} &       & \multicolumn{1}{c}{5.58} & \multicolumn{1}{c}{8.27} & \multicolumn{1}{c}{0.00} & \multicolumn{1}{c}{80.00} & \multicolumn{1}{c}{1306} & \multicolumn{1}{c}{5.53} & \multicolumn{1}{c}{12.02} & \multicolumn{1}{c}{0.00} & \multicolumn{1}{c}{333.33} & \multicolumn{1}{c}{1306} &       & \multicolumn{1}{c}{Datasus/SIM} & \multicolumn{1}{c}{2000} \\
    \multicolumn{1}{l}{1 to 27 days} & \multicolumn{1}{c}{13.73} & \multicolumn{1}{c}{15.89} & \multicolumn{1}{c}{0.00} & \multicolumn{1}{c}{333.33} & \multicolumn{1}{c}{5507} &       & \multicolumn{1}{c}{13.61} & \multicolumn{1}{c}{13.28} & \multicolumn{1}{c}{0.00} & \multicolumn{1}{c}{166.67} & \multicolumn{1}{c}{1306} & \multicolumn{1}{c}{13.46} & \multicolumn{1}{c}{15.55} & \multicolumn{1}{c}{0.00} & \multicolumn{1}{c}{333.33} & \multicolumn{1}{c}{1306} &       & \multicolumn{1}{c}{Datasus/SIM} & \multicolumn{1}{c}{2000} \\
    \multicolumn{1}{l}{27 days to 1 year} & \multicolumn{1}{c}{9.34} & \multicolumn{1}{c}{16.34} & \multicolumn{1}{c}{0.00} & \multicolumn{1}{c}{666.67} & \multicolumn{1}{c}{5507} &       & \multicolumn{1}{c}{9.85} & \multicolumn{1}{c}{12.67} & \multicolumn{1}{c}{0.00} & \multicolumn{1}{c}{142.86} & \multicolumn{1}{c}{1306} & \multicolumn{1}{c}{9.38} & \multicolumn{1}{c}{21.80} & \multicolumn{1}{c}{0.00} & \multicolumn{1}{c}{666.67} & \multicolumn{1}{c}{1306} &       & \multicolumn{1}{c}{Datasus/SIM} & \multicolumn{1}{c}{2000} \\
    \multicolumn{1}{l}{Infectious} & \multicolumn{1}{c}{2.00} & \multicolumn{1}{c}{7.03} & \multicolumn{1}{c}{0.00} & \multicolumn{1}{c}{333.33} & \multicolumn{1}{c}{5507} &       & \multicolumn{1}{c}{1.96} & \multicolumn{1}{c}{3.95} & \multicolumn{1}{c}{0.00} & \multicolumn{1}{c}{31.25} & \multicolumn{1}{c}{1306} & \multicolumn{1}{c}{2.17} & \multicolumn{1}{c}{10.56} & \multicolumn{1}{c}{0.00} & \multicolumn{1}{c}{333.33} & \multicolumn{1}{c}{1306} &       & \multicolumn{1}{c}{Datasus/SIM} & \multicolumn{1}{c}{2000} \\
    \multicolumn{1}{l}{Respiratory} & \multicolumn{1}{c}{1.52} & \multicolumn{1}{c}{4.45} & \multicolumn{1}{c}{0.00} & \multicolumn{1}{c}{142.86} & \multicolumn{1}{c}{5507} &       & \multicolumn{1}{c}{1.67} & \multicolumn{1}{c}{5.40} & \multicolumn{1}{c}{0.00} & \multicolumn{1}{c}{142.86} & \multicolumn{1}{c}{1306} & \multicolumn{1}{c}{1.58} & \multicolumn{1}{c}{4.20} & \multicolumn{1}{c}{0.00} & \multicolumn{1}{c}{52.63} & \multicolumn{1}{c}{1306} &       & \multicolumn{1}{c}{Datasus/SIM} & \multicolumn{1}{c}{2000} \\
    \multicolumn{1}{l}{Perinatal} & \multicolumn{1}{c}{11.04} & \multicolumn{1}{c}{16.32} & \multicolumn{1}{c}{0.00} & \multicolumn{1}{c}{666.67} & \multicolumn{1}{c}{5507} &       & \multicolumn{1}{c}{10.76} & \multicolumn{1}{c}{11.92} & \multicolumn{1}{c}{0.00} & \multicolumn{1}{c}{166.67} & \multicolumn{1}{c}{1306} & \multicolumn{1}{c}{10.99} & \multicolumn{1}{c}{21.33} & \multicolumn{1}{c}{0.00} & \multicolumn{1}{c}{666.67} & \multicolumn{1}{c}{1306} &       & \multicolumn{1}{c}{Datasus/SIM} & \multicolumn{1}{c}{2000} \\
    \multicolumn{1}{l}{Congenital} & \multicolumn{1}{c}{2.13} & \multicolumn{1}{c}{5.01} & \multicolumn{1}{c}{0.00} & \multicolumn{1}{c}{93.02} & \multicolumn{1}{c}{5507} &       & \multicolumn{1}{c}{2.19} & \multicolumn{1}{c}{5.11} & \multicolumn{1}{c}{0.00} & \multicolumn{1}{c}{55.56} & \multicolumn{1}{c}{1306} & \multicolumn{1}{c}{1.95} & \multicolumn{1}{c}{4.24} & \multicolumn{1}{c}{0.00} & \multicolumn{1}{c}{52.63} & \multicolumn{1}{c}{1306} &       & \multicolumn{1}{c}{Datasus/SIM} & \multicolumn{1}{c}{2000} \\
    \multicolumn{1}{l}{External} & \multicolumn{1}{c}{0.37} & \multicolumn{1}{c}{1.91} & \multicolumn{1}{c}{0.00} & \multicolumn{1}{c}{43.48} & \multicolumn{1}{c}{5507} &       & \multicolumn{1}{c}{0.34} & \multicolumn{1}{c}{2.03} & \multicolumn{1}{c}{0.00} & \multicolumn{1}{c}{41.67} & \multicolumn{1}{c}{1306} & \multicolumn{1}{c}{0.36} & \multicolumn{1}{c}{1.56} & \multicolumn{1}{c}{0.00} & \multicolumn{1}{c}{19.61} & \multicolumn{1}{c}{1306} &       & \multicolumn{1}{c}{Datasus/SIM} & \multicolumn{1}{c}{2000} \\
    \multicolumn{1}{l}{Nutritional} & \multicolumn{1}{c}{0.60} & \multicolumn{1}{c}{3.22} & \multicolumn{1}{c}{0.00} & \multicolumn{1}{c}{166.67} & \multicolumn{1}{c}{5507} &       & \multicolumn{1}{c}{0.56} & \multicolumn{1}{c}{1.92} & \multicolumn{1}{c}{0.00} & \multicolumn{1}{c}{23.26} & \multicolumn{1}{c}{1306} & \multicolumn{1}{c}{0.55} & \multicolumn{1}{c}{2.13} & \multicolumn{1}{c}{0.00} & \multicolumn{1}{c}{32.26} & \multicolumn{1}{c}{1306} &       & \multicolumn{1}{c}{Datasus/SIM} & \multicolumn{1}{c}{2000} \\
    \multicolumn{1}{l}{Other} & \multicolumn{1}{c}{0.87} & \multicolumn{1}{c}{3.60} & \multicolumn{1}{c}{0.00} & \multicolumn{1}{c}{142.86} & \multicolumn{1}{c}{5507} &       & \multicolumn{1}{c}{0.79} & \multicolumn{1}{c}{3.49} & \multicolumn{1}{c}{0.00} & \multicolumn{1}{c}{83.33} & \multicolumn{1}{c}{1306} & \multicolumn{1}{c}{0.89} & \multicolumn{1}{c}{4.75} & \multicolumn{1}{c}{0.00} & \multicolumn{1}{c}{142.86} & \multicolumn{1}{c}{1306} &       & \multicolumn{1}{c}{Datasus/SIM} & \multicolumn{1}{c}{2000} \\
    \multicolumn{1}{l}{Ill-Defined} & \multicolumn{1}{c}{4.55} & \multicolumn{1}{c}{10.68} & \multicolumn{1}{c}{0.00} & \multicolumn{1}{c}{142.86} & \multicolumn{1}{c}{5507} &       & \multicolumn{1}{c}{5.20} & \multicolumn{1}{c}{10.71} & \multicolumn{1}{c}{0.00} & \multicolumn{1}{c}{102.94} & \multicolumn{1}{c}{1306} & \multicolumn{1}{c}{4.35} & \multicolumn{1}{c}{9.19} & \multicolumn{1}{c}{0.00} & \multicolumn{1}{c}{80.65} & \multicolumn{1}{c}{1306} &       & \multicolumn{1}{c}{Datasus/SIM} & \multicolumn{1}{c}{2000} \\
          &       &       &       &       &       &       &       &       &       &       &       &       &       &       &       &       &       &       &  \\
    \multicolumn{1}{l}{\textbf{Adult Mortality Rate}} &       &       &       &       &       &       &       &       &       &       &       &       &       &       &       &       &       &       &  \\
    \multicolumn{1}{l}{Total} & \multicolumn{1}{c}{3.31} & \multicolumn{1}{c}{1.52} & \multicolumn{1}{c}{0.00} & \multicolumn{1}{c}{12.53} & \multicolumn{1}{c}{5507} &       & \multicolumn{1}{c}{3.19} & \multicolumn{1}{c}{1.43} & \multicolumn{1}{c}{0.00} & \multicolumn{1}{c}{9.10} & \multicolumn{1}{c}{1306} & \multicolumn{1}{c}{3.43} & \multicolumn{1}{c}{1.55} & \multicolumn{1}{c}{0.00} & \multicolumn{1}{c}{12.53} & \multicolumn{1}{c}{1306} &       & \multicolumn{1}{c}{Datasus/SIM} & \multicolumn{1}{c}{2000} \\
    \multicolumn{1}{l}{Amenable to Primary Care} & \multicolumn{1}{c}{0.53} & \multicolumn{1}{c}{0.48} & \multicolumn{1}{c}{0.00} & \multicolumn{1}{c}{4.62} & \multicolumn{1}{c}{5507} &       & \multicolumn{1}{c}{0.50} & \multicolumn{1}{c}{0.48} & \multicolumn{1}{c}{0.00} & \multicolumn{1}{c}{4.62} & \multicolumn{1}{c}{1306} & \multicolumn{1}{c}{0.56} & \multicolumn{1}{c}{0.48} & \multicolumn{1}{c}{0.00} & \multicolumn{1}{c}{4.30} & \multicolumn{1}{c}{1306} &       & \multicolumn{1}{c}{Datasus/SIM} & \multicolumn{1}{c}{2000} \\
    \multicolumn{1}{l}{non-Amenable to Primary Care} & \multicolumn{1}{c}{2.78} & \multicolumn{1}{c}{1.33} & \multicolumn{1}{c}{0.00} & \multicolumn{1}{c}{11.59} & \multicolumn{1}{c}{5507} &       & \multicolumn{1}{c}{2.69} & \multicolumn{1}{c}{1.27} & \multicolumn{1}{c}{0.00} & \multicolumn{1}{c}{8.53} & \multicolumn{1}{c}{1306} & \multicolumn{1}{c}{2.88} & \multicolumn{1}{c}{1.32} & \multicolumn{1}{c}{0.00} & \multicolumn{1}{c}{10.44} & \multicolumn{1}{c}{1306} &       & \multicolumn{1}{c}{Datasus/SIM} & \multicolumn{1}{c}{2000} \\
    \multicolumn{1}{l}{Circulatory} & \multicolumn{1}{c}{0.72} & \multicolumn{1}{c}{0.60} & \multicolumn{1}{c}{0.00} & \multicolumn{1}{c}{8.60} & \multicolumn{1}{c}{5507} &       & \multicolumn{1}{c}{0.67} & \multicolumn{1}{c}{0.55} & \multicolumn{1}{c}{0.00} & \multicolumn{1}{c}{4.62} & \multicolumn{1}{c}{1306} & \multicolumn{1}{c}{0.74} & \multicolumn{1}{c}{0.64} & \multicolumn{1}{c}{0.00} & \multicolumn{1}{c}{8.60} & \multicolumn{1}{c}{1306} &       & \multicolumn{1}{c}{Datasus/SIM} & \multicolumn{1}{c}{2000} \\
    \multicolumn{1}{l}{Neoplasm} & \multicolumn{1}{c}{0.42} & \multicolumn{1}{c}{0.45} & \multicolumn{1}{c}{0.00} & \multicolumn{1}{c}{3.95} & \multicolumn{1}{c}{5507} &       & \multicolumn{1}{c}{0.40} & \multicolumn{1}{c}{0.45} & \multicolumn{1}{c}{0.00} & \multicolumn{1}{c}{3.95} & \multicolumn{1}{c}{1306} & \multicolumn{1}{c}{0.42} & \multicolumn{1}{c}{0.42} & \multicolumn{1}{c}{0.00} & \multicolumn{1}{c}{3.19} & \multicolumn{1}{c}{1306} &       & \multicolumn{1}{c}{Datasus/SIM} & \multicolumn{1}{c}{2000} \\
    \multicolumn{1}{l}{Respiratory} & \multicolumn{1}{c}{0.17} & \multicolumn{1}{c}{0.25} & \multicolumn{1}{c}{0.00} & \multicolumn{1}{c}{2.89} & \multicolumn{1}{c}{5507} &       & \multicolumn{1}{c}{0.15} & \multicolumn{1}{c}{0.23} & \multicolumn{1}{c}{0.00} & \multicolumn{1}{c}{1.91} & \multicolumn{1}{c}{1306} & \multicolumn{1}{c}{0.19} & \multicolumn{1}{c}{0.27} & \multicolumn{1}{c}{0.00} & \multicolumn{1}{c}{2.89} & \multicolumn{1}{c}{1306} &       & \multicolumn{1}{c}{Datasus/SIM} & \multicolumn{1}{c}{2000} \\
    \multicolumn{1}{l}{Endocrine} & \multicolumn{1}{c}{0.17} & \multicolumn{1}{c}{0.26} & \multicolumn{1}{c}{0.00} & \multicolumn{1}{c}{3.37} & \multicolumn{1}{c}{5507} &       & \multicolumn{1}{c}{0.16} & \multicolumn{1}{c}{0.24} & \multicolumn{1}{c}{0.00} & \multicolumn{1}{c}{2.10} & \multicolumn{1}{c}{1306} & \multicolumn{1}{c}{0.18} & \multicolumn{1}{c}{0.26} & \multicolumn{1}{c}{0.00} & \multicolumn{1}{c}{2.56} & \multicolumn{1}{c}{1306} &       & \multicolumn{1}{c}{Datasus/SIM} & \multicolumn{1}{c}{2000} \\
    \multicolumn{1}{l}{External} & \multicolumn{1}{c}{0.67} & \multicolumn{1}{c}{0.60} & \multicolumn{1}{c}{0.00} & \multicolumn{1}{c}{5.57} & \multicolumn{1}{c}{5507} &       & \multicolumn{1}{c}{0.68} & \multicolumn{1}{c}{0.60} & \multicolumn{1}{c}{0.00} & \multicolumn{1}{c}{5.57} & \multicolumn{1}{c}{1306} & \multicolumn{1}{c}{0.68} & \multicolumn{1}{c}{0.60} & \multicolumn{1}{c}{0.00} & \multicolumn{1}{c}{5.32} & \multicolumn{1}{c}{1306} &       & \multicolumn{1}{c}{Datasus/SIM} & \multicolumn{1}{c}{2000} \\
    \multicolumn{1}{l}{Nutritional} & \multicolumn{1}{c}{0.22} & \multicolumn{1}{c}{0.30} & \multicolumn{1}{c}{0.00} & \multicolumn{1}{c}{6.92} & \multicolumn{1}{c}{5507} &       & \multicolumn{1}{c}{0.18} & \multicolumn{1}{c}{0.25} & \multicolumn{1}{c}{0.00} & \multicolumn{1}{c}{2.20} & \multicolumn{1}{c}{1306} & \multicolumn{1}{c}{0.24} & \multicolumn{1}{c}{0.35} & \multicolumn{1}{c}{0.00} & \multicolumn{1}{c}{6.92} & \multicolumn{1}{c}{1306} &       & \multicolumn{1}{c}{Datasus/SIM} & \multicolumn{1}{c}{2000} \\
    \multicolumn{1}{l}{Ill-Defined} & \multicolumn{1}{c}{0.64} & \multicolumn{1}{c}{0.72} & \multicolumn{1}{c}{0.00} & \multicolumn{1}{c}{7.89} & \multicolumn{1}{c}{5507} &       & \multicolumn{1}{c}{0.65} & \multicolumn{1}{c}{0.73} & \multicolumn{1}{c}{0.00} & \multicolumn{1}{c}{5.63} & \multicolumn{1}{c}{1306} & \multicolumn{1}{c}{0.65} & \multicolumn{1}{c}{0.71} & \multicolumn{1}{c}{0.00} & \multicolumn{1}{c}{5.96} & \multicolumn{1}{c}{1306} &       & \multicolumn{1}{c}{Datasus/SIM} & \multicolumn{1}{c}{2000} \\
    \multicolumn{1}{l}{Other} & \multicolumn{1}{c}{0.31} & \multicolumn{1}{c}{0.35} & \multicolumn{1}{c}{0.00} & \multicolumn{1}{c}{3.54} & \multicolumn{1}{c}{5507} &       & \multicolumn{1}{c}{0.29} & \multicolumn{1}{c}{0.32} & \multicolumn{1}{c}{0.00} & \multicolumn{1}{c}{2.04} & \multicolumn{1}{c}{1306} & \multicolumn{1}{c}{0.34} & \multicolumn{1}{c}{0.37} & \multicolumn{1}{c}{0.00} & \multicolumn{1}{c}{3.54} & \multicolumn{1}{c}{1306} &       & \multicolumn{1}{c}{Datasus/SIM} & \multicolumn{1}{c}{2000} \\
    \multicolumn{1}{l}{Diabetes} & \multicolumn{1}{c}{0.09} & \multicolumn{1}{c}{0.18} & \multicolumn{1}{c}{0.00} & \multicolumn{1}{c}{1.59} & \multicolumn{1}{c}{5507} &       & \multicolumn{1}{c}{0.09} & \multicolumn{1}{c}{0.17} & \multicolumn{1}{c}{0.00} & \multicolumn{1}{c}{1.33} & \multicolumn{1}{c}{1306} & \multicolumn{1}{c}{0.10} & \multicolumn{1}{c}{0.18} & \multicolumn{1}{c}{0.00} & \multicolumn{1}{c}{1.41} & \multicolumn{1}{c}{1306} &       & \multicolumn{1}{c}{Datasus/SIM} & \multicolumn{1}{c}{2000} \\
    \multicolumn{1}{l}{Hypertension} & \multicolumn{1}{c}{0.07} & \multicolumn{1}{c}{0.15} & \multicolumn{1}{c}{0.00} & \multicolumn{1}{c}{2.43} & \multicolumn{1}{c}{5507} &       & \multicolumn{1}{c}{0.06} & \multicolumn{1}{c}{0.15} & \multicolumn{1}{c}{0.00} & \multicolumn{1}{c}{2.43} & \multicolumn{1}{c}{1306} & \multicolumn{1}{c}{0.07} & \multicolumn{1}{c}{0.17} & \multicolumn{1}{c}{0.00} & \multicolumn{1}{c}{2.15} & \multicolumn{1}{c}{1306} &       & \multicolumn{1}{c}{Datasus/SIM} & \multicolumn{1}{c}{2000} \\
          &       &       &       &       &       &       &       &       &       &       &       &       &       &       &       &       &       &       &  \\
    \multicolumn{1}{l}{\textbf{Controls}} &       &       &       &       &       &       &       &       &       &       &       &       &       &       &       &       &       &       &  \\
    \multicolumn{1}{l}{Population (1,000)} & \multicolumn{1}{c}{29.67} & \multicolumn{1}{c}{181.18} & \multicolumn{1}{c}{0.71} & \multicolumn{1}{c}{9968.49} & \multicolumn{1}{c}{5224} &       & \multicolumn{1}{c}{29.15} & \multicolumn{1}{c}{104.17} & \multicolumn{1}{c}{0.85} & \multicolumn{1}{c}{2302.83} & \multicolumn{1}{c}{1306} & \multicolumn{1}{c}{30.49} & \multicolumn{1}{c}{101.71} & \multicolumn{1}{c}{1.16} & \multicolumn{1}{c}{2139.125} & \multicolumn{1}{c}{1306} &       & \multicolumn{1}{c}{2000 Census} & \multicolumn{1}{c}{2000} \\
    \multicolumn{1}{l}{Life Expectancy} & \multicolumn{1}{c}{68.54} & \multicolumn{1}{c}{3.93} & \multicolumn{1}{c}{57.46} & \multicolumn{1}{c}{77.24} & \multicolumn{1}{c}{5224} &       & \multicolumn{1}{c}{67.85} & \multicolumn{1}{c}{3.96} & \multicolumn{1}{c}{57.65} & \multicolumn{1}{c}{77.18} & \multicolumn{1}{c}{1306} & \multicolumn{1}{c}{68.37} & \multicolumn{1}{c}{3.99} & \multicolumn{1}{c}{58.02} & \multicolumn{1}{c}{76.11} & \multicolumn{1}{c}{1306} &       & \multicolumn{1}{c}{2000 Census} & \multicolumn{1}{c}{2000} \\
    \multicolumn{1}{l}{Expected Years of Study} & \multicolumn{1}{c}{8.42} & \multicolumn{1}{c}{1.77} & \multicolumn{1}{c}{2.29} & \multicolumn{1}{c}{13.02} & \multicolumn{1}{c}{5224} &       & \multicolumn{1}{c}{8.06} & \multicolumn{1}{c}{1.82} & \multicolumn{1}{c}{2.65} & \multicolumn{1}{c}{13.02} & \multicolumn{1}{c}{1306} & \multicolumn{1}{c}{8.39} & \multicolumn{1}{c}{1.70} & \multicolumn{1}{c}{2.91} & \multicolumn{1}{c}{12.27} & \multicolumn{1}{c}{1306} &       & \multicolumn{1}{c}{2000 Census} & \multicolumn{1}{c}{2000} \\
    \multicolumn{1}{l}{Iliteracy Rate (above 18y old)} & \multicolumn{1}{c}{23.18} & \multicolumn{1}{c}{13.44} & \multicolumn{1}{c}{1.00} & \multicolumn{1}{c}{63.01} & \multicolumn{1}{c}{5224} &       & \multicolumn{1}{c}{25.10} & \multicolumn{1}{c}{13.40} & \multicolumn{1}{c}{2.03} & \multicolumn{1}{c}{58.71} & \multicolumn{1}{c}{1306} & \multicolumn{1}{c}{23.65} & \multicolumn{1}{c}{13.73} & \multicolumn{1}{c}{1.00} & \multicolumn{1}{c}{60.79} & \multicolumn{1}{c}{1306} &       & \multicolumn{1}{c}{2000 Census} & \multicolumn{1}{c}{2000} \\
    \multicolumn{1}{l}{Income per capita} & \multicolumn{1}{c}{345.06} & \multicolumn{1}{c}{192.94} & \multicolumn{1}{c}{62.65} & \multicolumn{1}{c}{1759.76} & \multicolumn{1}{c}{5224} &       & \multicolumn{1}{c}{315.10} & \multicolumn{1}{c}{183.59} & \multicolumn{1}{c}{64.91} & \multicolumn{1}{c}{1639.93} & \multicolumn{1}{c}{1306} & \multicolumn{1}{c}{343.36} & \multicolumn{1}{c}{201.14} & \multicolumn{1}{c}{76.32} & \multicolumn{1}{c}{1596.51} & \multicolumn{1}{c}{1306} &       & \multicolumn{1}{c}{2000 Census} & \multicolumn{1}{c}{2000} \\
    \multicolumn{1}{l}{Share of Population Below Poverty Line} & \multicolumn{1}{c}{0.40} & \multicolumn{1}{c}{0.23} & \multicolumn{1}{c}{0.01} & \multicolumn{1}{c}{0.91} & \multicolumn{1}{c}{5224} &       & \multicolumn{1}{c}{0.45} & \multicolumn{1}{c}{0.22} & \multicolumn{1}{c}{0.01} & \multicolumn{1}{c}{0.86} & \multicolumn{1}{c}{1306} & \multicolumn{1}{c}{0.40} & \multicolumn{1}{c}{0.24} & \multicolumn{1}{c}{0.01} & \multicolumn{1}{c}{0.866} & \multicolumn{1}{c}{1306} &       & \multicolumn{1}{c}{2000 Census} & \multicolumn{1}{c}{2000} \\
    \multicolumn{1}{l}{Gini Coefficient} & \multicolumn{1}{c}{0.55} & \multicolumn{1}{c}{0.07} & \multicolumn{1}{c}{0.30} & \multicolumn{1}{c}{0.87} & \multicolumn{1}{c}{5224} &       & \multicolumn{1}{c}{0.55} & \multicolumn{1}{c}{0.07} & \multicolumn{1}{c}{0.31} & \multicolumn{1}{c}{0.81} & \multicolumn{1}{c}{1306} & \multicolumn{1}{c}{0.54} & \multicolumn{1}{c}{0.07} & \multicolumn{1}{c}{0.34} & \multicolumn{1}{c}{0.8} & \multicolumn{1}{c}{1306} &       & \multicolumn{1}{c}{2000 Census} & \multicolumn{1}{c}{2000} \\
    \multicolumn{1}{l}{Access to Sewage Network} & \multicolumn{1}{c}{0.26} & \multicolumn{1}{c}{0.31} & \multicolumn{1}{c}{0.00} & \multicolumn{1}{c}{0.99} & \multicolumn{1}{c}{5224} &       & \multicolumn{1}{c}{0.17} & \multicolumn{1}{c}{0.25} & \multicolumn{1}{c}{0.00} & \multicolumn{1}{c}{0.99} & \multicolumn{1}{c}{1306} & \multicolumn{1}{c}{0.32} & \multicolumn{1}{c}{0.33} & \multicolumn{1}{c}{0.00} & \multicolumn{1}{c}{0.981} & \multicolumn{1}{c}{1306} &       & \multicolumn{1}{c}{2000 Census} & \multicolumn{1}{c}{2000} \\
    \multicolumn{1}{l}{Access to Garbage Collection Service} & \multicolumn{1}{c}{0.55} & \multicolumn{1}{c}{0.27} & \multicolumn{1}{c}{0.00} & \multicolumn{1}{c}{1.00} & \multicolumn{1}{c}{5224} &       & \multicolumn{1}{c}{0.50} & \multicolumn{1}{c}{0.26} & \multicolumn{1}{c}{0.00} & \multicolumn{1}{c}{1.00} & \multicolumn{1}{c}{1306} & \multicolumn{1}{c}{0.57} & \multicolumn{1}{c}{0.28} & \multicolumn{1}{c}{0.00} & \multicolumn{1}{c}{0.999} & \multicolumn{1}{c}{1306} &       & \multicolumn{1}{c}{2000 Census} & \multicolumn{1}{c}{2000} \\
    \multicolumn{1}{l}{Access to Water Network} & \multicolumn{1}{c}{0.59} & \multicolumn{1}{c}{0.24} & \multicolumn{1}{c}{0.00} & \multicolumn{1}{c}{1.00} & \multicolumn{1}{c}{5224} &       & \multicolumn{1}{c}{0.56} & \multicolumn{1}{c}{0.23} & \multicolumn{1}{c}{0.00} & \multicolumn{1}{c}{1.00} & \multicolumn{1}{c}{1306} & \multicolumn{1}{c}{0.61} & \multicolumn{1}{c}{0.24} & \multicolumn{1}{c}{0.00} & \multicolumn{1}{c}{1} & \multicolumn{1}{c}{1306} &       & \multicolumn{1}{c}{2000 Census} & \multicolumn{1}{c}{2000} \\
    \multicolumn{1}{l}{Access to Electricity} & \multicolumn{1}{c}{0.88} & \multicolumn{1}{c}{0.16} & \multicolumn{1}{c}{0.08} & \multicolumn{1}{c}{1.00} & \multicolumn{1}{c}{5224} &       & \multicolumn{1}{c}{0.85} & \multicolumn{1}{c}{0.18} & \multicolumn{1}{c}{0.08} & \multicolumn{1}{c}{1.00} & \multicolumn{1}{c}{1306} & \multicolumn{1}{c}{0.88} & \multicolumn{1}{c}{0.16} & \multicolumn{1}{c}{0.19} & \multicolumn{1}{c}{1} & \multicolumn{1}{c}{1306} &       & \multicolumn{1}{c}{2000 Census} & \multicolumn{1}{c}{2000} \\
    \multicolumn{1}{l}{Urbanization Rate} & \multicolumn{1}{c}{0.61} & \multicolumn{1}{c}{0.23} & \multicolumn{1}{c}{0.00} & \multicolumn{1}{c}{1.00} & \multicolumn{1}{c}{5224} &       & \multicolumn{1}{c}{0.58} & \multicolumn{1}{c}{0.22} & \multicolumn{1}{c}{0.03} & \multicolumn{1}{c}{1.00} & \multicolumn{1}{c}{1306} & \multicolumn{1}{c}{0.62} & \multicolumn{1}{c}{0.23} & \multicolumn{1}{c}{0.05} & \multicolumn{1}{c}{1} & \multicolumn{1}{c}{1306} &       & \multicolumn{1}{c}{2000 Census} & \multicolumn{1}{c}{2000} \\
    \multicolumn{1}{l}{Average Neighbors Spending Health Spending per capita} & \multicolumn{1}{c}{208.47} & \multicolumn{1}{c}{124.00} & \multicolumn{1}{c}{1.74} & \multicolumn{1}{c}{3298.40} & \multicolumn{1}{c}{5222} &       & \multicolumn{1}{c}{181.06} & \multicolumn{1}{c}{105.13} & \multicolumn{1}{c}{42.02} & \multicolumn{1}{c}{2287.30} & \multicolumn{1}{c}{1306} & \multicolumn{1}{c}{228.64} & \multicolumn{1}{c}{148.82} & \multicolumn{1}{c}{40.65} & \multicolumn{1}{c}{3298.403} & \multicolumn{1}{c}{1305} &       & \multicolumn{1}{c}{Finbra} & \multicolumn{1}{c}{2000} \\
    \multicolumn{1}{l}{Noncompliance with Fiscal Responsibility Law (HR Spending)} & \multicolumn{1}{c}{0.03} & \multicolumn{1}{c}{0.17} & \multicolumn{1}{c}{0.00} & \multicolumn{1}{c}{1.00} & \multicolumn{1}{c}{5099} &       & \multicolumn{1}{c}{0.03} & \multicolumn{1}{c}{0.18} & \multicolumn{1}{c}{0.00} & \multicolumn{1}{c}{1.00} & \multicolumn{1}{c}{1258} & \multicolumn{1}{c}{0.03} & \multicolumn{1}{c}{0.17} & \multicolumn{1}{c}{0.00} & \multicolumn{1}{c}{1} & \multicolumn{1}{c}{1277} &       & \multicolumn{1}{c}{Finbra} & \multicolumn{1}{c}{2000} \\
          &       &       &       &       &       &       &       &       &       &       &       &       &       &       &       &       &       &       &  \\
    \midrule
    \midrule
          &       &       &       &       &       &       &       &       &       &       &       &       &       &       &       &       &       &       &  \\
    \end{tabular}%
    



\end{threeparttable}
}
\end{center}
\end{footnotesize}
\end{table}
\end{sidewaystable}

\subsection{EC/29 and Fiscal Data}

For evaluating the causal impact of the EC/29 on public spending, we combine public spending data from the Brazilian Finance System (FINBRA)\footnote{All spending data is presented in 2010 R\$. We used the General Price Index (IGP) to correct values}, which covers the period of 1998 to 2010, with data from the Brazilian National System of Public Health Budget (Datasus/SIOPS)\footnote{SIOPS was created right after the EC/29 to monitor revenues and expenditure in the provision of health care at the state and municipal levels, and to monitor compliance with the EC/29.} available from 2000 onward. FINBRA provides data on total public spending, spending by type, and spending by a few aggregated categories, such as Health and Sanitation, Education and Culture, etc, and SIOPS provides data on total health spending, health spending from own resources, health spending from intergovernmental transfers, spending by type, and the share of own resources spent in health.  Additionally, we also gather data on total public revenues and revenues by source (tax or intergovernmental transfers) from FINBRA.

Figure \ref{fig:4} displays the spatial variation in the share of own resources spent in health. Municipalities below the EC/29 are represented with colors in the red scale, while municipalities above the target are represented with the blue scale. The map shows significant differences in the share of own resources spent in health within the same state, providing the identifying variation of this study as we include state fixed-effects in our main specification. 

\begin{figure}[h]
\begin{center}
    \caption{EC/29 Compliance Geographic Variation}
    \scalebox{0.7}{
    \includegraphics{plots/ec29_map.pdf}
    % \label{fig:4}
    }
\end{center}
\end{figure}

\subsection{Infant Mortality and Birth Outcomes}

We use micro-data from Brazilian National System of Mortality Records (Datasus/SIM) and from the Brazilian National System of Birth Records (Datasus/SINASC) to construct Infant Mortality Rates. These micro-data allow us to construct Infant Mortality Rates by timing of death, and for the main causes of death. Moreover, following \cite{alfradique2009internaccoes} classification we are able to construct mortality rates amenable and non-amenable to primary care.

The SINASC micro-data records all births in Brazil and provides detailed information on these births, such as Apgar 1 and 5, birth weight and gestation weeks.

\subsection{Health Inputs}

There is no data available on municipalities' supply of public health infrastructure for the period of this analysis. However, we are able to indirectly construct proxies for public health infrastructure using the micro-data from the National System of Information on Ambulatory Care (Datasus/SIA). This database records every ambulatorial procedure funded by SUS, with information on the type and complexity of the procedure, the health professional responsible, and the corresponding health facility register number. Using this data we are able to calculate the number of health facilities with ambulatory service in a municipality, as well as the number of facilities with ambulatory services classified by the type of service and professional responsible for the service \footnote{We are able to construct these variables only for the period of 1998 to 2007, as changes in the SIA classification of ambulatorial procedures changes in 2008.}. Variables on the access to healthcare were constructed using data from the Brazilian National System of Information on Primary Care (Datasus/SIAB) and from SINASC. Primary care coverage at the intensive and extensive margin data comes from SIAB and data on the access to health services from SINASC. Lastly, we use SIA micro-data to build data on ambulatory production.

\subsection{Controls}

Our control variables can be classified into two different categories: socioeconomic controls and fiscal controls. The first, comes from the Census of 2000. The later, from FINBRA dataset. We use as fiscal controls the average health spending per capita in the bordering municipalities\footnote{cite Mattos article} and a dummy that indicates whether a municipalities spend more than 60\% of its revenue with personnel, non complying with the Fiscal Responsibility Law of 2000.

