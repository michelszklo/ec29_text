\section{Conclusions}\label{sec:conclusion}
\setstretch{1.5}

Our empirical analysis has demonstrated that when municipalities are induced to increase public health spending they do so by increasing mainly spending relative to the administrative structure of public health - roughly half of the increase -  followed by spending with investments and human resources. We also demonstrate that this increase is associated with a higher number of administrative professionals, greater supply of municipal hospitals, and greater primary care coverage at the intensive margin, with also a higher number of health facilities with primary care related professionals. The shifts in spending and health inputs are associated with small to moderate reductions in infant mortality rates related to improvements in primary care access, and long term reductions in total infant mortality rates. \cite{bhalotra2019can} have shown that the combination of access to primary and hospital care leads to better health outcomes relative to only primary care. This is a plausible channel through which the increase in the supply of municipal hospitals might be affecting infant mortality in our analysis.

These results are extremely relevant, specially in a context of a universal an decentralized health system, where provision of health care occurs mainly at the municipal level, and the majority of the resources spent locally comes from local tax incomes, in opposition to intergovernmental transfers. [Discuss transfers vs own resource spending]. We are not able to formally test exactly how health inputs and outcomes would react if municipalities allocated less resources on administrative structure and more resources into investments and personnel, but the evidence here present indicates it could lead to further improvements in health outcomes, and, thus, a more efficient use of resources within the public health sector. 
