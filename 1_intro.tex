\section{Introduction}\label{sec:intro}
\pagenumbering{arabic}
\setstretch{1.5}
\setcounter{page}{1}


-------------

%Understanding the production function of healthcare provision is complicated, given 
Potential opening paragraph structure
\begin{enumerate}
    \item Understanding the production function of health care is complicated, given that this production function is multi-faceted. Covers labour provision \citep{Custeretal1990}, capital and drugs \citep{Austeretal1969}, micro-level health seeking behaviour \citep{LlerasMuney2005}.  Overarching point is resources.
    \item Then paragraph segueing into first paragraph below, noting broad increases in public health spending.  Cite relevant big papers here.  Below first two paragraphs should be linked into one.
    \item Then go straight to p3 from below, making v clear what we do.
    \item \ [Could add results on centralization.  Recent NBER WP showing that centralization is good for health, at least when considering obstetric care: \citet{Fischeretal2022}]
\end{enumerate}
-------------

The last decades were marked by significant increases in public health expenditure around the globe. Data from the \cite{wb} reveals that per capita public health expenditure more than doubled since the turn of the century. A question that still remains unanswered is how effective this type of expenditure is in reducing mortality, specially among developing countries. 

Economist have been trying to answer this question for decades, but the evidence is still inconclusive. Moreover, most of the studies that attempt to establish this causal relationship are not able to analyse the links in the chain connecting health spending and health outcomes, and thus offer incomplete evidence on the effectiveness of public health spending.  

In this article, we aim to fill this gap by answering several questions along the chain connecting public health spending to health outcomes. How municipalities allocate resources when increasing health spending? How expenditures translate into health inputs, such as health infrastructure, health services, human resources and ambulatorial production? How all these affects infant mortality? To answer these questions we treat Brazil as case study, leveraging the variation in public health spending generated by Brazil's \nth{29} Constitutional Amendment of 2000 to document the causal effects of health spending on infant mortality using a difference-in-difference design with continuous treatment.

After a decade of public health underfinancing, in September of 2000, the Brazilian Congress enacted the \nth{29} Constitutional Amendment. It established the minimum share of resources that the federal, state and municipal governments need to spend on the provision of public health services. This institutional reform was responsible for increasing public health spending and for raising the direct participation of states and municipalities in the financing of health care \citep{Piola2013}.

Our econometric analysis ....

The remaining of the article is organized as follow: 


