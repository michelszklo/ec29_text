\section{Introduction}\label{sec:intro}
\pagenumbering{arabic}
\setstretch{1.5}
\setcounter{page}{1}


% -------------

%Understanding the production function of healthcare provision is complicated, given 
% Potential opening paragraph structure
% \begin{enumerate}
%     \item Understanding the production function of health care is complicated, given that this production function is multi-faceted. Covers labour provision \citep{Custeretal1990}, capital and drugs \citep{Austeretal1969}, micro-level health seeking behaviour \citep{LlerasMuney2005}.  Overarching point is resources.
%     \item Then paragraph segueing into first paragraph below, noting broad increases in public health spending.  Cite relevant big papers here.  Below first two paragraphs should be linked into one.
%     \item Then go straight to p3 from below, making v clear what we do.
%     \item \ [Could add results on centralization.  Recent NBER WP showing that centralization is good for health, at least when considering obstetric care: \citet{Fischeretal2022}]
% \end{enumerate}
% -------------

Understanding the production function of health care is complicated, given that this production function is multi-faced. It covers labour provision \citep{Custeretal1990}, capital and drugs \citep{Austeretal1969}, micro-level health seeking behaviour \citep{LlerasMuney2005}, among others. An overarching point, however, is the role of resources in the health care production function.

The last decades were marked by significant increases in public health expenditure around the globe. Data from the \cite{wb} reveals that per capita public health expenditure more than doubled since the turn of the century. A question that still remains unanswered is how effective this type of expenditure is in reducing mortality, specially among developing countries. Economist have been trying to answer this question for decades, but the evidence is still inconclusive. Moreover, most of the studies that attempt to establish this causal relationship are not able to analyse the links in the chain connecting health spending and health outcomes, and thus offer incomplete evidence on the effectiveness of public health spending.

In this article, we aim to fill this gap by answering several questions along the chain connecting public health spending to health outcomes. How municipalities allocate resources when increasing health spending? How expenditures translate into health inputs, such as health infrastructure, health services, human resources and ambulatorial production? How all these affects infant mortality? To answer these questions we treat Brazil as case study, leveraging the variation in municipal public health spending generated by Brazil's \nth{29} Constitutional Amendment of 2000 to document the causal effects of health spending on infant mortality using a difference-in-difference design with continuous treatment.

After a decade of public health underfinancing, in September of 2000, the Brazilian Congress enacted the \nth{29} Constitutional Amendment. It established the minimum share of resources that the federal, state and municipal governments need to spend on the provision of public health services. This institutional reform was responsible for increasing public health spending and for raising the direct participation of states and municipalities in the financing of health care \citep{Piola2013}.

Previous research has documented the relationship between health spending and mortality, but the great majority does not provide well identified parameters of the causal relationship between health spending and mortality. Most of the previous research relies on single sections of cross-country data and usually cannot account for unobserved heterogeneity and existing trends that could bias estimations. \cite{filmer1999} use an instrumental variable approach on a global panel data to find no significant impacts on infant and child mortality. \cite{bokhari2007}, using a similar approach on a cross-section for the year 2000, find small but significant effects on child and maternal mortality. More recently, \cite{moreno2015} find very similar effects, reinforcing this evidence. \cite{nixon2006}, using a 15 year panel data for 15 European Union members, finds that increases in health spending are associated with large reduction in infant mortality. \cite{gupta2002effectiveness} analyse this relationship for 50 developing and transition countries and find effects on infant and child mortality that are not very robust to different specifications. Working with larger and richer data set of developing and transition countries, \cite{Gupta2003} estimates suggests that the  effects of health spending on infant and child mortality are twice as large among the poor. Notwithstanding, a recent review of cross-country studies suggests that, in general, these cross-country results are very sensitive to robustness checks \citep{Nakamura2020}.

Some of the identification issues faced by the cross-country studies are partially addressed by the use of fixed effects in the micro-level literature. \cite{cremieux1999} findings suggest that increases in health expenditure are associated with decreases in the infant mortality rate and increases in life expectancy, on a panel of data for Canadian provinces. \cite{sonia2007}, working with a rich panel data at the individual level in India, presents probit models estimates of the impact of health expenditures on the risk of infant mortality that suggest no significant contemporaneous effect, but long term and small effects for rural residents. Research for Brazil suggests that increases in health spending are associated with increases in primary care coverage,  the number of mothers attending seven or more prenatal visits, and with decreases in infant mortality rates, especially for the poorer municipalities \citep{paixao2012,Castro2019}. 

A recent work provides much better identified parameters \citep{castro2021effects}. Using a panel of small Brazilian municipalities for the period of 2002-2012 and a regression discontinuity design approach, this study finds large and significant effects of health spending on infant mortality, with elasticities ranging from $-0.5$ to $-0.9$. Moreover, they show that health spending presents strong spatial externalities, with the population of neighboring municipalities also benefiting from increases in health spending. However, the issue with this approach is that it leverages exogenous transfers to municipalities from a federal grant that have been shown to also impact education outcomes and poverty reduction \citep{Litschig2013}, which in turn might be correlated with mortality outcomes. Disentangling the drivers of mortality reductions in this setting is rather difficult. 

It is hard to imagine scenarios in which increasing spending would not lead to improvements in outcomes, but the overall evidence on the impacts of health spending on health outcomes it is still quite mixed and weak. The main contributions of this paper lies not only in providing the one of the first well identified causal parameters on the relationship between health spending and infant mortality, but also in exploring the pathways through which health spending affects infant mortality. The richness of Brazilian health data allow us to construct a unique panel data set, covering fiscal data, health inputs and health outcomes.

Our econometric analysis suggests this constitutional reform had promoted substantial increase in local health spending. This increase took place mainly through administrative, investment and human resources spending, which in turn has been translated into greater primary care coverage, and greater supply of municipal hospitals and health care human resources. This shift in health inputs had led to important reductions in infant mortality rates within 24 hours of birth and in infant mortality rates caused by perinatal conditions, with elasticity ranging between $-0.12$ and $-0.27$ for these mortality rates. Moreover, we find some long term effect on total infant mortality, infant mortality amenable to primary care and infant mortality caused by infectious and respiratory diseases.

The remaining of the article is organized as follow: Section \ref{sec:inst} outlines the institutional background and the \nth{29} Constitutional Amendment. In Section \ref{sec:data} we detail our data. In Section \ref{sec:emp} we describe our empirical strategy. Our results are presented in Section \ref{sec:results}. Finally, Section concludes the paper. 


