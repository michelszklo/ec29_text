\documentclass[12pt, final]{article}

%_ PACKAGES __________________________________________________________________________ %

%__ INPUT/OUTPUT LANGUAGE _________________________________ %
\usepackage[portuguese,english]{babel}
\usepackage[utf8]{inputenc}
\usepackage[T1]{fontenc}
\usepackage{enumitem}
\usepackage{indentfirst}
\usepackage[super]{nth}
\usepackage{setspace}

%__ MATH __________________________________________________ %
\usepackage{amsfonts}
\usepackage{amssymb}
\usepackage{amsmath}
\usepackage{amsthm}
\usepackage{bbm}

%__ GRAPHS & TABLES________________________________________ %
\usepackage{graphicx}
\usepackage{booktabs}
\usepackage{multirow}
\usepackage{array}
\usepackage[flushleft,online,para]{threeparttable}
\usepackage{multirow}
\usepackage{svg}
\usepackage{pdfpages}
\usepackage{float}

% \usepackage[nolists,heads,tablesfirst,nomarkers]{endfloat}
% \renewcommand{\efloatseparator}{\mbox} % allows tables to share a page

\usepackage{sublabel}

\usepackage{caption}
\usepackage{subcaption}
\usepackage{chngcntr}

\usepackage{parskip}       

\usepackage[capposition=top]{floatrow}


\usepackage{tabularx}
\newcolumntype{Z}{>{\centering\arraybackslash}X}
\newcolumntype{L}{>{\raggedright\arraybackslash}X}

\usepackage{dcolumn}
\newcolumntype{d}[1]{D{.}{.}{#1}}

\usepackage{rotating}               
\usepackage{lscape}
\usepackage{pdflscape}
% \DeclareDelayedFloatFlavor{sidewaystable}{table}

%__ BIBLIOGRAPHY __________________________________________ %
\usepackage[round,longnamesfirst,nonamebreak]{natbib}
\usepackage{microtype}
%\usepackage{url}

%__ PDF, DISPLAY & PRODUCTIVITY ___________________________ %
\usepackage{xcolor}
\definecolor{darkblue}{rgb}{0,0,0.5}
\usepackage{hyperref}
\hypersetup{
	colorlinks = true,
	linkcolor = darkblue,
	citecolor = darkblue,
	pdfborder = 0 0 0,
	pdfdisplaydoctitle = true,
	pdfhighlight = /N,
	pdfpagelayout = OneColumn,
	pdfpagemode = UseNone,
	pdfstartview = {FitH},
	pdfauthor = {{MS, RR \& DC}},
	pdftitle = {{Health \& Constitutional Spending Reform}},
	pdfsubject = {{}}
}
\hypersetup{urlcolor=blue}

\usepackage[textsize=footnotesize, colorinlistoftodos, textwidth=4cm, obeyDraft]{todonotes}

\usepackage{geometry}
\geometry{verbose,tmargin=2.54cm,bmargin=2.54cm,lmargin=2.54cm,rmargin=2.54cm}
\usepackage{setspace}
\onehalfspacing

\usepackage[bottom, multiple]{footmisc}

\usepackage{verbatim}
\usepackage[normalem]{ulem}
\usepackage{mathpazo}
\usepackage{lscape}

\usepackage{fancyhdr}
\pagestyle{fancy}
\fancyhf{}
\fancyfoot[R]{\thepage}
\renewcommand{\headrulewidth}{0pt}


%__ APPENDIX _____________________________________________ %
\usepackage[,titletoc,toc,title,page]{appendix}

%__ COMMANDS _________________________________________________________________________ %
\newcommand{\mc}{\multicolumn}
\newcommand{\lbar}{\underline}
\newcommand{\ubar}{\overline}

\usepackage{etoolbox}
\usepackage{tocloft}






\usepackage{etoolbox}
\usepackage{tocloft}


%\newcommand\aauthorA{Michel Szklo}
\newcommand\aaffiliationA{FGV-SP}

\begin{document}


\title{\textbf{Public Health Spending and Infant Mortality: Evidence from a Constitutional Reform in Brazil}}
\pagenumbering{gobble}

\author{
\small{Michel Szklo}\footnote{FGV - S\~ao Paulo School of Business Administration. Contact email: \href{mailto:mszklo@gmail.com}{mszklo@gmail.com}.}
\and \small{Damian Clarke}\footnote{Department of Economics, Universidad de Chile, MIPP \& IZA. Address: Diagonal Paraguay, 257, Santiago, Chile.  Contact email: \href{mailto:dclarke@fen.uchile.cl}{dclarke@fen.uchile.cl}.}
\and \small{Rudi Rocha}\footnote{FGV - S\~ao Paulo School of Business Administration, rudi.rocha@fgv.br.}
}
\date{\today}
\maketitle
\begin{center}
    %First draft: July 2020 \\
    %\emph{Preliminary. Please do not circulate}
\end{center}
\singlespacing
\begin{abstract}
 \noindent This paper treats Brazil as a case study to document the causal effects of health spending on infant mortality. By leveraging the variation in health spending promoted by Brazil's \nth{29} Constitutional Amendment of 2000, we are able to document not only the effects of health spending on infant mortality, but also the links in the chain connecting spending to health outcomes. We show that increases in health spending translate into greater primary care coverage, higher supply of hospitals and low skilled professionals. Associated with this effects we document moderate reductions in infant mortality within 24 hours and due to perinatal conditions, as well as long term reductions in total infant mortality, infant mortality amenable to primary care, and infant mortality by infectious and respiratory causes.  Our results contribute to the literature on the impacts of health spending by providing on of the first well identified causal parameter of the relationship between spending and mortality.
\end{abstract}

\noindent \textbf{JEL Codes}: I1, I3, O5\\
\noindent \textbf{Keywords}: Health spending; health care reform; health care production; infant mortality.

\newpage
%\thispagestyle{empty} 
%\onehalfspacing 
\setcounter{page}{1}



\section{Introduction}\label{sec:intro}
\pagenumbering{arabic}
\setstretch{1.5}
\setcounter{page}{1}


% -------------

%Understanding the production function of healthcare provision is complicated, given 
% Potential opening paragraph structure
% \begin{enumerate}
%     \item Understanding the production function of health care is complicated, given that this production function is multi-faceted. Covers labour provision \citep{Custeretal1990}, capital and drugs \citep{Austeretal1969}, micro-level health seeking behaviour \citep{LlerasMuney2005}.  Overarching point is resources.
%     \item Then paragraph segueing into first paragraph below, noting broad increases in public health spending.  Cite relevant big papers here.  Below first two paragraphs should be linked into one.
%     \item Then go straight to p3 from below, making v clear what we do.
%     \item \ [Could add results on centralization.  Recent NBER WP showing that centralization is good for health, at least when considering obstetric care: \citet{Fischeretal2022}]
% \end{enumerate}
% -------------

Understanding the production function of health care is complicated, given that this production function is multi-faced. It covers labour provision \citep{Custeretal1990}, capital and drugs \citep{Austeretal1969}, micro-level health seeking behaviour \citep{LlerasMuney2005}, among others. An overarching point, however, is the role of resources in the health care production function.

The last decades were marked by significant increases in public health expenditure around the globe. Data from the \cite{wb} reveals that per capita public health expenditure more than doubled since the turn of the century. A question that still remains unanswered is how effective this type of expenditure is in reducing mortality, specially among developing countries. Economist have been trying to answer this question for decades, but the evidence is still inconclusive. Moreover, most of the studies that attempt to establish this causal relationship are not able to analyse the links in the chain connecting health spending and health outcomes, and thus offer incomplete evidence on the effectiveness of public health spending.

In this article, we aim to fill this gap by answering several questions along the chain connecting public health spending to health outcomes. How municipalities allocate resources when increasing health spending? How expenditures translate into health inputs, such as health infrastructure, health services, human resources and ambulatorial production? How all these affects infant mortality? To answer these questions we treat Brazil as case study, leveraging the variation in municipal public health spending generated by Brazil's \nth{29} Constitutional Amendment of 2000 to document the causal effects of health spending on infant mortality using a difference-in-difference design with continuous treatment.

After a decade of public health underfinancing, in September of 2000, the Brazilian Congress enacted the \nth{29} Constitutional Amendment. It established the minimum share of resources that the federal, state and municipal governments need to spend on the provision of public health services. This institutional reform was responsible for increasing public health spending and for raising the direct participation of states and municipalities in the financing of health care \citep{Piola2013}.

Previous research has documented the relationship between health spending and mortality, but the great majority does not provide well identified parameters of the causal relationship between health spending and mortality. Most of the previous research relies on single sections of cross-country data and usually cannot account for unobserved heterogeneity and existing trends that could bias estimations. \cite{filmer1999} use an instrumental variable approach on a global panel data to find no significant impacts on infant and child mortality. \cite{bokhari2007}, using a similar approach on a cross-section for the year 2000, find small but significant effects on child and maternal mortality. More recently, \cite{moreno2015} find very similar effects, reinforcing this evidence. \cite{nixon2006}, using a 15 year panel data for 15 European Union members, finds that increases in health spending are associated with large reduction in infant mortality. \cite{gupta2002effectiveness} analyse this relationship for 50 developing and transition countries and find effects on infant and child mortality that are not very robust to different specifications. Working with larger and richer data set of developing and transition countries, \cite{Gupta2003} estimates suggests that the  effects of health spending on infant and child mortality are twice as large among the poor. Notwithstanding, a recent review of cross-country studies suggests that, in general, these cross-country results are very sensitive to robustness checks \citep{Nakamura2020}.

Some of the identification issues faced by the cross-country studies are partially addressed by the use of fixed effects in the micro-level literature. \cite{cremieux1999} findings suggest that increases in health expenditure are associated with decreases in the infant mortality rate and increases in life expectancy, on a panel of data for Canadian provinces. \cite{sonia2007}, working with a rich panel data at the individual level in India, presents probit models estimates of the impact of health expenditures on the risk of infant mortality that suggest no significant contemporaneous effect, but long term and small effects for rural residents. Research for Brazil suggests that increases in health spending are associated with increases in primary care coverage,  the number of mothers attending seven or more prenatal visits, and with decreases in infant mortality rates, especially for the poorer municipalities \citep{paixao2012,Castro2019}. 

A recent work provides much better identified parameters \citep{castro2021effects}. Using a panel of small Brazilian municipalities for the period of 2002-2012 and a regression discontinuity design approach, this study finds large and significant effects of health spending on infant mortality, with elasticities ranging from $-0.5$ to $-0.9$. Moreover, they show that health spending presents strong spatial externalities, with the population of neighboring municipalities also benefiting from increases in health spending. However, the issue with this approach is that it leverages exogenous transfers to municipalities from a federal grant that have been shown to also impact education outcomes and poverty reduction \citep{Litschig2013}, which in turn might be correlated with mortality outcomes. Disentangling the drivers of mortality reductions in this setting is rather difficult. 

It is hard to imagine scenarios in which increasing spending would not lead to improvements in outcomes, but the overall evidence on the impacts of health spending on health outcomes it is still quite mixed and weak. The main contributions of this paper lies not only in providing the one of the first well identified causal parameters on the relationship between health spending and infant mortality, but also in exploring the pathways through which health spending affects infant mortality. The richness of Brazilian health data allow us to construct a unique panel data set, covering fiscal data, health inputs and health outcomes.

Our econometric analysis suggests this constitutional reform had promoted substantial increase in local health spending. This increase took place mainly through administrative, investment and human resources spending, which in turn has been translated into greater primary care coverage, and greater supply of municipal hospitals and health care human resources. This shift in health inputs had led to important reductions in infant mortality rates within 24 hours of birth and in infant mortality rates caused by perinatal conditions, with elasticity ranging between $-0.12$ and $-0.27$ for these mortality rates. Moreover, we find some long term effect on total infant mortality, infant mortality amenable to primary care and infant mortality caused by infectious and respiratory diseases.

The remaining of the article is organized as follow: Section \ref{sec:inst} outlines the institutional background and the \nth{29} Constitutional Amendment. In Section \ref{sec:data} we detail our data. In Section \ref{sec:emp} we describe our empirical strategy. Our results are presented in Section \ref{sec:results}. Finally, Section concludes the paper. 



\input{2_institutional}
\section{Data}\label{sec:data}
\setstretch{1.5}

Table \ref{table:stats} presents summary statistics at the baseline year for all the variables used in this analysis: variables related to the EC/29, fiscal data, health inputs, infant mortality rates, birth outcomes, and control variables.

\subsection{EC/29 and Fiscal Data}

To evaluate municipalities' fiscal reactions to the EC/29, we combine public spending data from the Brazilian Finance System (FINBRA)\footnote{All spending data is presented in 2010 R\$. We used the General Price Index (IGP) to correct values}, which covers the period of 1998 to 2010, with data from the Brazilian National System of Public Health Budget (Datasus/SIOPS)\footnote{SIOPS was created right after the EC/29 to monitor revenues and expenditure in the provision of health care at the state and municipal levels, and to monitor compliance with the EC/29.} available from 2000 onward. FINBRA provides data on total public spending, and spending by a few aggregated categories, such as Health and Sanitation, Education and Culture, etc, and data on public revenues. The SIOPS, on the other hand, provides more detailed information on public health spending, which allow us to evaluate how municipalities allocate resources within the public health sector. It gathers data on total health spending, health spending from own resources, health spending from intergovernmental transfers, and health spending by types of spending, including spending in human resources, investments, services from third parties, and others\footnote{Others expenditures includes mainly administrative spending}. Moreover, SIOPS calculates for each municipality the share of own resources spent in the provision of health care, that we use to build our independent variable.

Figure \ref{fig:5} displays the spatial variation in the share of own resources spent in health. Municipalities below the EC/29 are represented with colors in the red scale, while municipalities above the target are represented with the blue scale. The map shows significant differences in the share of own resources spent in health within the same state, providing the identifying variation of this study as we include state fixed-effects in our main specification. 

\begin{figure}[h!]
\begin{center}
    \caption{EC/29 Compliance Geographic Variation}
    \scalebox{0.7}{
    \includegraphics{plots/ec29_map.pdf}
    \label{fig:5}
    }
\end{center}
\end{figure}

\subsection{Health Inputs}

We combine data from several sources to build our health inputs data base. First we collect data on primary care coverage - extensive and intensive margin -  from Brazilian National System of Information on Primary Care (Datasus/SIAB) . Data on health human resources and hospital infrastructure comes from the 1999, 2002, 2005 and 2009 Medical-Sanitary Assistance Survey (AMS), a census of the health sector run by Brazilian Institute of Geography and Statistics (IBGE). 

The Brazilian National System of Information on Ambulatory Care (Datasus/SIA) every ambulatory procedure funded by SUS, with information on the type and complexity of the procedure, the health professional responsible, and the corresponding health facility register number. This data is used to create variables on ambulatory production, primary care ambulatory production, and ambulatory production by procedures complexity. We also use this data to indirectly create variables that measure the supply of health ambulatory facilities, as well as the supply of ambulatory facilities with health professionals related to primary care services. This is done by evaluating the number of facilities within a municipality that recorded ambulatory procedures of interests or ambulatory procedures executed by specific professionals of interest \footnote{We are able to construct these variables only for the period of 1998 to 2007, as changes in the SIA classification of ambulatorial procedures changes in 2008.}.

To measure access to health services, we used data from the from Brazilian National System of Birth Records (Datasus/SINASC), that records every birth in Brazil and provides detailed information on these births. Using this data we calculated the share of no prenatal visits, 1-6 prenatal visits and more than 7 prenatal visits. Importantly, in the first years of our sample, there is no information on prenatal visits for a considerable amount of births. To account for this under-registration issue, we also calculated the share of prenatal ignored. By estimating the impacts on this variable, we can separate the effects of access increasing from improvements in data registration.


\subsection{Infant Mortality and Birth Outcomes}

We use micro-data from Brazilian National System of Mortality Records (Datasus/SIM) and from SINASC to construct Infant Mortality Rates. These micro-data allow us to construct Infant Mortality Rates by the timing of death, and for the main causes of death. Moreover, following \cite{alfradique2009internaccoes} classification we are able to construct mortality rates that are amenable and non-amenable to primary care access. The SINASC also provides detailed information on Apgar 1 and 5, birth weight, and premature births. We also use data on population by age and sex from Datasus to calculate fertility rates.


\subsection{Controls}

Our control variables can be classified into three different categories: baseline socioeconomic controls, time varying socioeconomic controls, and time varying fiscal controls. The first, comes from IBGE's Census of 2000. Our time varying socioeconomic controls includes GDP per capita, from IBGE, and the \emph{Bolsa Família} program transfers per capita, from the Ministry of Social Development. The last set of controls comes from FINBRA dataset. We use as fiscal controls the average health spending per capita in the bordering municipalities\footnote{\cite{castro2021effects} show the importance spending of spillover effects in health, which highlights the importance of including this control.} and the share of total current public revenue spent with personnel\footnote{In the year of 2000, the Fiscal Responsibility Law \citep{lrf} was enacted. This law defined that municipalities cannot spend more than 60\% of its revenue in personnel. We include this control to account for the different incentives municipalities might face according to their compliance with the law.}. 

\input{4_empirical}
\section{Empirical Findings}\label{sec:results}
\setstretch{1.5}

The goal of this section is to understand the impacts of health spending on health outcomes, and the pathways through which the impact take place. For that, we first present the estimates of the impact of EC/29 on fiscal and spending outcomes. Later, we analyse how health expenditure increases translate into health inputs. Lastly, we examine the impacts on infant mortality rates and birth outcomes. All outcomes were analyzed as rates and that is how effects are presented in our regression tables and graphs. However, in our discussion of results, we will focus on the percent variation relative to baseline means of a representative municipality with a distance of 10\% to the EC/29 target of the share of own resource spent in health. This distance is equivalent to the distance to the target of the municipalities in bottom quartile of the distribution of the share of own resource spent in health, which is the group of municipalities that presented the most pronounced increase in health spending after the EC/29 was enacted.


\subsection{Municipalities' Fiscal Response to the EC/29}\label{sec:results_fiscal}

Table \ref{table:fiscal} shows the estimates for total public revenue and spending, public spending by category, and public health spending, total, by source and type. In column 1 we present our baseline estimates, a continuous DiD with municipality and state-year fixed effects. Column 2 adds to the baseline specification a set of baseline controls interacted with time. Column 3 adds socioeconomic time varying controls, and column 4, our preferred specification and most saturated, adds time varying fiscal controls. 

Panel A shows that the EC/29 had no significant impact on total revenues and total spending per capita. Though the points estimates are positive, Finbra data, specially for the pre-reform years, is in general quite noisy\footnote{Appendix Figure \ref{fig:b1} plots the dynamic effects for these outcomes estimated with Equation \ref{eq:2}.}. Next, we look at public health spending by category (Panel B). The only category that has been significantly impacted by the amendment is Health and Sanitation spending per capita, and the results are quite robust across different specifications. 

In our preferred specification (column 4), the estimate of around 300 suggests a increase of R\$30 in health spending per capita for our representative municipality, equivalent to a increase of around 14\% relative to the baseline health and sanitation spending per capita (see Table \ref{table:stats}). This distance is roughly the distance to the target of the bottom quartile of the distribution of the share of own resources spent in health in the baseline.

As discussed in Section \ref{sec:emp}, the validation of our research design relies partially on evaluating the presence of pre-trends. Even though SIOPS is a much better data source to study health spending, it is only available after the year 2000. Therefore, we will use Finbra data mainly to evaluate the presence of pre-trends in health spending and the move to analyse health spending and resource allocation within the public health sector using SIOPS data. Figure \ref{fig:6a} plots the dynamic effects estimated with Equation \ref{eq:2} for the equivalent of the specifications presented in Column 1 and 4, for Health and Sanitation per capita. We find no pre-trends and a clear and significant pattern of increase in spending, with each of the first years after the EC/29 presenting stronger effects, that stabilize after 2004. Appendix Figure \ref{fig:b2} plots the dynamic effects for all other categories of spending. Estimates are very imprecise for almost all categories and it is hard to extract much information. But in general, there seem to be no pre-trend, nor significant effects on other categories of spending besides Health and Sanitation. These results are extremely relevant because it will allow us to claim that any reductions we find in Infant Mortality rates are most certainly associated with increases in health expenditure and not increases in spending from other categories that could also affect mortality, such as social assistance and education. 


\begin{table}[h!]
\begin{footnotesize}
\begin{center}
\scalebox{0.8}{
\begin{threeparttable}[b]

  \centering
  \caption{Fiscal Reactions}
     \begin{tabular}{rrrrrr}
          &       &       &       &       &  \\
          &       &       &       &       &  \\
    \midrule
    \midrule
          &       & \multicolumn{1}{c}{(1)} & \multicolumn{1}{c}{(2)} & \multicolumn{1}{c}{(3)} & \multicolumn{1}{c}{(4)} \\
    \midrule
    \multicolumn{1}{p{23.285em}}{\textbf{A. Public Revenue and Spending per capita (Finbra)}} &       &       &       &       &  \\
    \multicolumn{1}{p{23.285em}}{Total Revenue} &       & \multicolumn{1}{c}{841.224} & \multicolumn{1}{c}{868.151} & \multicolumn{1}{c}{911.342} & \multicolumn{1}{c}{929.681} \\
          &       & \multicolumn{1}{c}{(1248.728)} & \multicolumn{1}{c}{(1264.873)} & \multicolumn{1}{c}{(1262.793)} & \multicolumn{1}{c}{(1264.449)} \\
    \multicolumn{1}{p{23.285em}}{Total Spending} &       & \multicolumn{1}{c}{1089.977} & \multicolumn{1}{c}{1116.507} & \multicolumn{1}{c}{1153.877} & \multicolumn{1}{c}{1120.384} \\
          &       & \multicolumn{1}{c}{(1450.455)} & \multicolumn{1}{c}{(1468.068)} & \multicolumn{1}{c}{(1466.367)} & \multicolumn{1}{c}{(1467.661)} \\
          &       &       &       &       &  \\
    \midrule
    \multicolumn{1}{p{23.285em}}{\textbf{B. Public Spending By Category (Finbra)}} &       &       &       &       &  \\
    \multicolumn{1}{p{23.285em}}{Health and Sanitation Spending} &       & \multicolumn{1}{c}{302.751***} & \multicolumn{1}{c}{307.022***} & \multicolumn{1}{c}{314.312***} & \multicolumn{1}{c}{308.104***} \\
          &       & \multicolumn{1}{c}{(94.499)} & \multicolumn{1}{c}{(95.853)} & \multicolumn{1}{c}{(94.851)} & \multicolumn{1}{c}{(94.901)} \\
    \multicolumn{1}{p{23.285em}}{Transport Spending} &       & \multicolumn{1}{c}{53.404} & \multicolumn{1}{c}{55.428} & \multicolumn{1}{c}{57.487} & \multicolumn{1}{c}{58.332} \\
          &       & \multicolumn{1}{c}{(64.588)} & \multicolumn{1}{c}{(65.717)} & \multicolumn{1}{c}{(65.673)} & \multicolumn{1}{c}{(65.738)} \\
    \multicolumn{1}{p{23.285em}}{Education and Culture Spending} &       & \multicolumn{1}{c}{181.786} & \multicolumn{1}{c}{193.281} & \multicolumn{1}{c}{203.347} & \multicolumn{1}{c}{195.657} \\
          &       & \multicolumn{1}{c}{(391.088)} & \multicolumn{1}{c}{(396.875)} & \multicolumn{1}{c}{(396.579)} & \multicolumn{1}{c}{(396.904)} \\
    \multicolumn{1}{p{23.285em}}{Housing and Urban Spending} &       & \multicolumn{1}{c}{106.441} & \multicolumn{1}{c}{103.719} & \multicolumn{1}{c}{107.997} & \multicolumn{1}{c}{105.891} \\
          &       & \multicolumn{1}{c}{(151.807)} & \multicolumn{1}{c}{(153.609)} & \multicolumn{1}{c}{(153.292)} & \multicolumn{1}{c}{(153.428)} \\
    \multicolumn{1}{p{23.285em}}{Social Assistance Spending per capita} &       & \multicolumn{1}{c}{189.327} & \multicolumn{1}{c}{197.649} & \multicolumn{1}{c}{200.966} & \multicolumn{1}{c}{200.701} \\
          &       & \multicolumn{1}{c}{(251.479)} & \multicolumn{1}{c}{(254.819)} & \multicolumn{1}{c}{(254.729)} & \multicolumn{1}{c}{(254.927)} \\
    \multicolumn{1}{p{23.285em}}{Spending in Other Categories per capita} &       & \multicolumn{1}{c}{362.594} & \multicolumn{1}{c}{365.701} & \multicolumn{1}{c}{381.213} & \multicolumn{1}{c}{365.486} \\
          &       & \multicolumn{1}{c}{(668.655)} & \multicolumn{1}{c}{(676.98)} & \multicolumn{1}{c}{(676.072)} & \multicolumn{1}{c}{(676.672)} \\
          &       &       &       &       &  \\
    \midrule
    \multicolumn{1}{p{23.285em}}{\textbf{C. Public Health Spending (SIOPS)}} &       &       &       &       &  \\
    \multicolumn{1}{p{23.285em}}{Total} &       & \multicolumn{1}{c}{529.375***} & \multicolumn{1}{c}{530.301***} & \multicolumn{1}{c}{530.936***} & \multicolumn{1}{c}{530.317***} \\
          &       & \multicolumn{1}{c}{(18.16)} & \multicolumn{1}{c}{(17.876)} & \multicolumn{1}{c}{(17.507)} & \multicolumn{1}{c}{(17.485)} \\
    \multicolumn{1}{p{23.285em}}{\textbf{By Source}} &       &       &       &       &  \\
    \multicolumn{1}{p{23.285em}}{Own Resources} &       & \multicolumn{1}{c}{580.644***} & \multicolumn{1}{c}{581.011***} & \multicolumn{1}{c}{581.215***} & \multicolumn{1}{c}{580.792***} \\
          &       & \multicolumn{1}{c}{(13.943)} & \multicolumn{1}{c}{(13.725)} & \multicolumn{1}{c}{(13.421)} & \multicolumn{1}{c}{(13.431)} \\
    \multicolumn{1}{p{23.285em}}{Transfers} &       & \multicolumn{1}{c}{-53.096***} & \multicolumn{1}{c}{-52.482***} & \multicolumn{1}{c}{-51.826***} & \multicolumn{1}{c}{-52.036***} \\
          &       & \multicolumn{1}{c}{(11.269)} & \multicolumn{1}{c}{(11.157)} & \multicolumn{1}{c}{(11.123)} & \multicolumn{1}{c}{(11.107)} \\
    \multicolumn{1}{p{23.285em}}{\textbf{By Type}} &       &       &       &       &  \\
    \multicolumn{1}{p{23.285em}}{Human Resources} &       & \multicolumn{1}{c}{96.997***} & \multicolumn{1}{c}{94.917***} & \multicolumn{1}{c}{95.188***} & \multicolumn{1}{c}{93.164***} \\
          &       & \multicolumn{1}{c}{(11.202)} & \multicolumn{1}{c}{(11.12)} & \multicolumn{1}{c}{(11)} & \multicolumn{1}{c}{(10.914)} \\
    \multicolumn{1}{p{23.285em}}{Investiment} &       & \multicolumn{1}{c}{132.947***} & \multicolumn{1}{c}{133.38***} & \multicolumn{1}{c}{133.42***} & \multicolumn{1}{c}{133.64***} \\
          &       & \multicolumn{1}{c}{(9.654)} & \multicolumn{1}{c}{(9.683)} & \multicolumn{1}{c}{(9.667)} & \multicolumn{1}{c}{(9.672)} \\
    \multicolumn{1}{p{23.285em}}{3rd parties services} &       & \multicolumn{1}{c}{55.911***} & \multicolumn{1}{c}{54.863***} & \multicolumn{1}{c}{54.756***} & \multicolumn{1}{c}{55.165***} \\
          &       & \multicolumn{1}{c}{(11.687)} & \multicolumn{1}{c}{(11.545)} & \multicolumn{1}{c}{(11.471)} & \multicolumn{1}{c}{(11.49)} \\
    \multicolumn{1}{p{23.285em}}{Other Expenditures} &       & \multicolumn{1}{c}{247.246***} & \multicolumn{1}{c}{250.325***} & \multicolumn{1}{c}{250.742***} & \multicolumn{1}{c}{251.5***} \\
          &       & \multicolumn{1}{c}{(11.365)} & \multicolumn{1}{c}{(11.417)} & \multicolumn{1}{c}{(11.403)} & \multicolumn{1}{c}{(11.37)} \\
          &       &       &       &       &  \\
    \bottomrule
    \bottomrule
    \end{tabular}%
    
    
  \label{table:fiscal}%
  
  \begin{tablenotes}
  \scriptsize{\underline{Notes}: The number of observations is 64470 for Finbra variables and 55810 for SIOPS variables.  DiD Estimates from Equation \ref{eq:1}. Independent variable is the distance to the EC/29 target in p.p. Column 1 presents the baseline model with municipality and state-year fixed effects. Column 2 adds baseline socioeconomic controls from the Census interacted with time. Column 3 adds controls for GDP per capita and \emph{Bolsa Familia} transfers per capita. Column 4 adds fiscal controls. Covariates omitted. Standard errors in brackets are clustered in the municipality level. ∗p < 0.10, ∗ ∗ p < 0.05, ∗ ∗ ∗p < 0.01}
  \end{tablenotes}

\end{threeparttable}
}
\end{center}
\end{footnotesize}
\end{table}

\begin{figure}[h!]
    \begin{center}
    \caption{Fiscal Reactions}\label{fig:6}
    \begin{subfigure}{0.49\textwidth}
        \caption{\scriptsize Total Revenue}\label{fig:6a}
        \centering
        \includegraphics[width=\textwidth]{plots/finbra_reccorr_pcapita_dist_ec29_baseline_dist_ec29_baseline_6.pdf}
    \end{subfigure}
    \begin{subfigure}{0.49\textwidth}
        \centering
        \caption{\scriptsize Total Public Spending}\label{fig:6b}
        \includegraphics[width=\textwidth]{plots/finbra_desp_o_pcapita_dist_ec29_baseline_dist_ec29_baseline_6.pdf}
    \end{subfigure}
    
    \end{center}
    
\end{figure}

Panel C in Table \ref{table:fiscal} presents the results for total health spending, health spending by source and health spending by type. Our estimations suggests an effect of R\$ 530 for total Health Spending per capita, which is equivalent to a 27\% increase in spending relative to the baseline for our representative municipality, almost twice the effect on Health and Sanitation per capita. Additionally, this effect comes almost entirely from increases in spending from own resources, a 50\% increase relative to own resource spending in the baseline. We also find some substitution effects, with municipalities reducing some of its spending from intergovernmental transfers in health. All types of health spending were responsible for this increase in total health spending, but the increase in investment is the noteworthy, specially in relative terms. This estimate is associated with a 90\% increase in health investments. Baseline statistics show very little resources allocated in investments within total public health spending, the great majority of resources were allocated in human resources and in other administrative expenses. Considering the importance of capital investments to the supply of medical resources and the quality of medical services, and the little amount of investments in the baseline, an effect of this size is really relevant. Other expenditures, that includes mainly administrative spending, presents the strongest effect in per capita terms, almost half of the total increase in health spending per capita, equivalent to a 34\% increase relative to the baseline for the representative municipality. In opposition to the relevance investment in a health production function, administrative expenditure plays a much minor role in affecting health outcomes.

\begin{figure}[h!]
    \begin{center}
    \caption{Effects Public Spending per capita - By Type}\label{fig:7}
    \begin{subfigure}{0.32\textwidth}
        \centering
        \caption{\scriptsize Human Resources}\label{fig:7a}
        \includegraphics[width=\textwidth]{plots/finbra_desp_pessoal_pcapita_dist_ec29_baseline_dist_ec29_baseline_7.pdf}
    \end{subfigure}
    \begin{subfigure}{0.32\textwidth}
        \centering
        \caption{\scriptsize Investments}\label{fig:7b}
        \includegraphics[width=\textwidth]{plots/finbra_desp_investimento_pcapita_dist_ec29_baseline_dist_ec29_baseline_7.pdf}
    \end{subfigure}
    \begin{subfigure}{0.32\textwidth}
        \centering
        \caption{\scriptsize Other}\label{fig:7c}
        \includegraphics[width=\textwidth]{plots/finbra_desp_outros_nature_pcapita_dist_ec29_baseline_dist_ec29_baseline_7.pdf}
    \end{subfigure}
    
    \end{center}
    
\end{figure}

The clear pattern of increases in health and sanitation spending depicted in Figure \ref{fig:6a} can also be seen for SIOPS total health spending in Figure \ref{fig:6b}, but at different levels. Figure \ref{fig:7} suggests that this pattern is mostly influenced by dynamic of the effects on human resources spending (Figure \ref{fig:7a}). Investments, 3rd parties and other expenditures present a sharp increase in spending in the first one or two years after the EC/29 an than stabilize.


\subsection{Effects on Health Inputs}

In this subsection we aim to explore the pathways that mediate the relationship between health spending and health outcomes. For that, we explore the impacts of the EC/29 on several health inputs: primary care coverage, human resources, hospital infrastructure, primary care related infrastructure, ambulatorial production, and access to health services. 

\begin{table}[H]
\begin{footnotesize}
\begin{center}
\scalebox{0.65}{
\begin{threeparttable}[b]

  \centering
  \caption{Primary Care Coverage, Health Infrastructure and Human Resources}
     \begin{tabular}{rrcccr}
          &       &       &       &       &  \\
    \midrule
    \midrule
          &       & \multicolumn{4}{c}{Distance to EC9 target} \\
\cmidrule{3-6}          &       & (1)   & (2)   & (3)   & \multicolumn{1}{c}{(4)} \\
    \midrule
    \multicolumn{1}{p{23.645em}}{\textbf{A. Primary Care Coverage -  Extensive Margin}} &       &       &       &       &  \\
    \multicolumn{1}{l}{\multirow{2}[0]{*}{Population covered (share) by Community Health Agents}} &       & 0.25*** & 0.245*** & 0.245*** & \multicolumn{1}{c}{0.245***} \\
          &       & (0.056) & (0.055) & (0.055) & \multicolumn{1}{c}{(0.055)} \\
    \multicolumn{1}{l}{\multirow{2}[0]{*}{Population covered (share) by Family Health Agents}} &       & 0.187*** & 0.197*** & 0.201*** & \multicolumn{1}{c}{0.2***} \\
          &       & (0.059) & (0.058) & (0.058) & \multicolumn{1}{c}{(0.058)} \\
          &       &       &       &       &  \\
    \midrule
    \multicolumn{1}{p{23.645em}}{\textbf{B. Primary Care Coverage -  Intensive Margin}} &       &       &       &       &  \\
    \multicolumn{1}{l}{\multirow{2}[0]{*}{N. of People Visited by Primary Care Agents (per capita)}} &       & 0.294*** & 0.287*** & 0.298*** & \multicolumn{1}{c}{0.297***} \\
          &       & (0.101) & (0.097) & (0.097) & \multicolumn{1}{c}{(0.097)} \\
    \multicolumn{1}{l}{\multirow{2}[0]{*}{N. of People Visited by Community Health Agents (per capita)}} &       & -0.028 & -0.025 & -0.026 & \multicolumn{1}{c}{-0.026} \\
          &       & (0.053) & (0.053) & (0.053) & \multicolumn{1}{c}{(0.053)} \\
    \multicolumn{1}{l}{\multirow{2}[0]{*}{N. of People Visited by Family Health Agents (per capita)}} &       & 0.321*** & 0.311*** & 0.323*** & \multicolumn{1}{c}{0.322***} \\
          &       & (0.098) & (0.093) & (0.093) & \multicolumn{1}{c}{(0.093)} \\
    \multicolumn{1}{l}{\multirow{2}[0]{*}{N. of Household Visits (per capita)}} &       & 1.059*** & 1.057*** & 1.085*** & \multicolumn{1}{c}{1.085***} \\
          &       & (0.325) & (0.325) & (0.324) & \multicolumn{1}{c}{(0.324)} \\
    \multicolumn{1}{l}{\multirow{2}[0]{*}{N. of Household Visits by Community Health Agents (per capita)}} &       & 0.396 & 0.375 & 0.368 & \multicolumn{1}{c}{0.37} \\
          &       & (0.277) & (0.279) & (0.279) & \multicolumn{1}{c}{(0.278)} \\
    \multicolumn{1}{l}{\multirow{2}[0]{*}{N. of Household Visits by Family Health Agents (per capita)}} &       & 0.653** & 0.676*** & 0.711*** & \multicolumn{1}{c}{0.71***} \\
          &       & (0.256) & (0.247) & (0.246) & \multicolumn{1}{c}{(0.246)} \\
    \multicolumn{1}{l}{\multirow{2}[0]{*}{N. of Appointments (per capita)}} &       & 0.181* & 0.181* & 0.186* & \multicolumn{1}{c}{0.187*} \\
          &       & (0.108) & (0.108) & (0.109) & \multicolumn{1}{c}{(0.109)} \\
    \multicolumn{1}{l}{\multirow{2}[0]{*}{N. of Appointments from Community Health Program (per capita)}} &       & -0.015 & -0.013 & -0.013 & \multicolumn{1}{c}{-0.013} \\
          &       & (0.02) & (0.021) & (0.021) & \multicolumn{1}{c}{(0.021)} \\
    \multicolumn{1}{l}{\multirow{2}[0]{*}{N. of Appointments from Family Health Program (per capita)}} &       & 0.192* & 0.188* & 0.193* & \multicolumn{1}{c}{0.194*} \\
          &       & (0.108) & (0.107) & (0.108) & \multicolumn{1}{c}{(0.108)} \\
          &       &       &       &       &  \\
    \midrule
    \multicolumn{1}{p{23.645em}}{\textbf{C. Number of Health Facilities (per capita * 1000) with}} &       &       &       &       &  \\
    \multicolumn{1}{l}{\multirow{2}[0]{*}{Ambulatory Service}} &       & -0.094** & -0.085** & -0.08* & \multicolumn{1}{c}{ -0.08* } \\
          &       & (0.042) & (0.043) & (0.042) & \multicolumn{1}{c}{ (0.042) } \\
    \multicolumn{1}{l}{\multirow{2}[0]{*}{Ambulatory Service and PSF Teams}} &       & 0.061** & 0.059** & 0.063** & \multicolumn{1}{c}{ 0.063** } \\
          &       & (0.029) & (0.028) & (0.028) & \multicolumn{1}{c}{ (0.028) } \\
    \multicolumn{1}{l}{\multirow{2}[0]{*}{Ambulatory Service and ACS Teams}} &       & 0.046 & 0.052 & 0.056* & \multicolumn{1}{c}{ 0.056* } \\
          &       & (0.033) & (0.032) & (0.032) & \multicolumn{1}{c}{ (0.032) } \\
    \multicolumn{1}{l}{\multirow{2}[0]{*}{Ambulatory Service and Community Doctors}} &       & 0.054* & 0.056* & 0.061** & \multicolumn{1}{c}{ 0.061** } \\
          &       & (0.032) & (0.031) & (0.031) & \multicolumn{1}{c}{ (0.031) } \\
    \multicolumn{1}{l}{\multirow{2}[0]{*}{Ambulatory Service and PSF Doctors}} &       & 0.047 & 0.051* & 0.056* & \multicolumn{1}{c}{ 0.056* } \\
          &       & (0.032) & (0.03) & (0.03) & \multicolumn{1}{c}{ (0.03) } \\
    \multicolumn{1}{l}{\multirow{2}[0]{*}{Ambulatory Service and PSF Nurses}} &       & 0.033* & 0.032 & 0.034* & \multicolumn{1}{c}{ 0.034* } \\
          &       & (0.02) & (0.02) & (0.02) & \multicolumn{1}{c}{ (0.02) } \\
    \multicolumn{1}{l}{\multirow{2}[0]{*}{Ambulatory Service and PSF Nursing Assistants}} &       & 0.061** & 0.066** & 0.07** & \multicolumn{1}{c}{ 0.07** } \\
          &       & (0.031) & (0.03) & (0.029) & \multicolumn{1}{c}{ (0.029) } \\
    \multicolumn{1}{l}{\multirow{2}[0]{*}{Ambulatory Service and ACS Nurses}} &       & 0.02  & 0.023 & 0.028 & \multicolumn{1}{c}{ 0.028 } \\
          &       & (0.03) & (0.029) & (0.029) & \multicolumn{1}{c}{ (0.029) } \\
          &       &       &       &       &  \\
    \midrule
    \multicolumn{1}{p{23.645em}}{\textbf{D. Hospital and Beds}} &       &       &       &       &  \\
    \multicolumn{1}{l}{\multirow{2}[0]{*}{N. of Hospital Beds (per capita * 1000)}} &       & -0.933** & -0.803** & -0.779** & \multicolumn{1}{c}{ -0.78** } \\
          &       & (0.373) & (0.373) & (0.372) & \multicolumn{1}{c}{ (0.372) } \\
    \multicolumn{1}{l}{\multirow{2}[0]{*}{Presence of Hospital}} &       & -0.11*** & -0.082** & -0.082** & \multicolumn{1}{c}{ -0.082** } \\
          &       & (0.041) & (0.039) & (0.039) & \multicolumn{1}{c}{ (0.039) } \\
          &       &       &       &       &  \\
    \midrule
    \midrule
          &       &       &       &       &  \\
    \end{tabular}%
    
    
  \label{table:infra}%

\end{threeparttable}
}
\end{center}
\end{footnotesize}
\end{table}

First we analyse the effects on primary care coverage at the extensive and intensive margin (Table \ref{table:infra}, Panel A and B). We find significant effects on the share of population covered by the Community Health program and by Family Health Program. Though significant and positive, these effects are quite small. The representative municipality only increased by 2 percentage points the share of the population covered by these primary care programs. On the other hand, the effects on the intensive margin are much more pronounced and our estimates suggests that they come mainly from the Family Health Program. We find significant increases in the number of people visited and in the number of household visits and appointments by Family Health Agents. They are equivalent to a 21\% and 11\% increase relative to the baseline, respectively, for our representative municipality. Figure \ref{fig:8} shows the dynamic effects for the extensive margin and Figure \ref{fig:9} the dynamic effects for the intensive margin of primary care coverage. The temporal patterns of this effects resemble the pattern in health spending increase, where the effect is increasing in the first years after the EC/29 and becomes steady after 2004. 

\begin{figure}[h!]
    \begin{center}
    \caption{Effects on Public Spending per capita - By Category}\label{fig:8}
    \begin{subfigure}{0.48\textwidth}
        \caption{\scriptsize Health and Sanitation}\label{fig:8a}
        \centering
        \includegraphics[width=\textwidth]{plots/finbra_desp_saude_san_pcapita_dist_ec29_baseline_dist_ec29_baseline_8.pdf}
    \end{subfigure}
    \begin{subfigure}{0.48\textwidth}
        \centering
        \caption{\scriptsize Education and Culture}\label{fig:8b}
        \includegraphics[width=\textwidth]{plots/finbra_desp_educ_cultura_pcapita_dist_ec29_baseline_dist_ec29_baseline_8.pdf}
    \end{subfigure}
    \begin{subfigure}{0.48\textwidth}
        \centering
        \caption{\scriptsize Social Assistance}\label{fig:8c}
        \includegraphics[width=\textwidth]{plots/finbra_desp_assist_prev_pcapita_dist_ec29_baseline_dist_ec29_baseline_8.pdf}
    \end{subfigure}
    \begin{subfigure}{0.48\textwidth}
        \centering
        \caption{\scriptsize Transport}\label{fig:8d}
        \includegraphics[width=\textwidth]{plots/finbra_desp_transporte_pcapita_dist_ec29_baseline_dist_ec29_baseline_8.pdf}
    \end{subfigure}
    \begin{subfigure}{0.48\textwidth}
        \centering
        \caption{\scriptsize Housing and Urban}\label{fig:8e}
        \includegraphics[width=\textwidth]{plots/finbra_desp_hab_urb_pcapita_dist_ec29_baseline_dist_ec29_baseline_8.pdf}
    \end{subfigure}
    \begin{subfigure}{0.48\textwidth}
        \centering
        \caption{\scriptsize Spending in Other Categories}\label{fig:8f}
        \includegraphics[width=\textwidth]{plots/finbra_desp_outros_area_pcapita_dist_ec29_baseline_dist_ec29_baseline_8.pdf}
    \end{subfigure}
    
    \end{center}
    
\end{figure}

\begin{figure}[h!]
    \begin{center}
    \caption{Effects on Public Health Spending per capita}\label{fig:9}
    \begin{subfigure}{0.48\textwidth}
        \caption{\scriptsize Health and Sanitation (Finbra)}\label{fig:9a}
        \centering
\includegraphics[width=\textwidth]{plots/finbra_desp_saude_san_pcapita_dist_ec29_baseline_dist_ec29_baseline_9.pdf}
    \end{subfigure}
    \begin{subfigure}{0.48\textwidth}
        \centering
        \caption{\scriptsize Total Health Spending (SIOPS)}\label{fig:9b}
        \includegraphics[width=\textwidth]{plots/siops_despsaude_pcapita_dist_ec29_baseline_dist_ec29_baseline_9.pdf}
    \end{subfigure}
    \begin{subfigure}{0.48\textwidth}
        \centering
        \caption{\scriptsize Health Spending - Own Resources (SIOPS)}\label{fig:9c}
        \includegraphics[width=\textwidth]{plots/siops_desprecpropriosaude_pcapita_dist_ec29_baseline_dist_ec29_baseline_9.pdf}
    \end{subfigure}
    \begin{subfigure}{0.48\textwidth}
        \centering
        \caption{\scriptsize Health Spending  - Transfers (SIOPS)}\label{fig:9d}
        \includegraphics[width=\textwidth]{plots/siops_despexrecproprio_pcapita_dist_ec29_baseline_dist_ec29_baseline_9.pdf}
    \end{subfigure}
    
    \end{center}
    
\end{figure}

Panel C presents our results on the public health sector human resources. We find no significant effect for the number of doctors and nurses, but Figures \ref{fig:10a} and \ref{fig:10b} suggests some effect taking place after 2005. On the other hand, we find significant effects in the number of nursing assistants and administrative professionals, a 13\% and 15\% increase, respectively, relative to the baseline. While Figure \ref{fig:10c} indicates a gradual effect trend after the EC/29, Figure \ref{fig:10d} suggests a sharp increase in the number of administrative personnel right after the EC/29. Not coincidentally, this pattern resembles the pattern found in other expenditures within health spending (Figure \ref{fig:7d}), that, as mentioned before, includes mainly administrative spending. 

\begin{figure}[h!]
    \begin{center}
    \caption{Effects on Public Health Spending per capita - By Type}\label{fig:10}
    \begin{subfigure}{0.48\textwidth}
        \centering
        \caption{\scriptsize Human Resources}\label{fig:10a}
        \includegraphics[width=\textwidth]{plots/spending/siops_desppessoal_pcapita_dist_ec29_baseline_dist_ec29_baseline_full.pdf}
    \end{subfigure}
    \begin{subfigure}{0.48\textwidth}
        \centering
        \caption{\scriptsize Investiment}\label{fig:10b}
        \includegraphics[width=\textwidth]{plots/spending/siops_despinvest_pcapita_dist_ec29_baseline_dist_ec29_baseline_full.pdf}
    \end{subfigure}
    \begin{subfigure}{0.48\textwidth}
        \centering
        \caption{\scriptsize 3rd parties services}\label{fig:10c}
        \includegraphics[width=\textwidth]{plots/spending/siops_despservicoster_pcapita_dist_ec29_baseline_dist_ec29_baseline_full.pdf}
    \end{subfigure}
    \begin{subfigure}{0.48\textwidth}
        \centering
        \caption{\scriptsize Other Expenditures}\label{fig:10d}
        \includegraphics[width=\textwidth]{plots/spending/siops_despoutros_pcapita_dist_ec29_baseline_dist_ec29_baseline_full.pdf}
    \end{subfigure}
    
    \end{center}
    
\end{figure}

Next, panel D shows the results for health infrastructure. The number of municipal hospital per 1000 inhabitants presented a significant increase of 0.16. This effect represents a sizable variation of 27\% relative to the baseline number of hospitals for the representative municipality. Moreover, Figure \ref{fig:11a} suggests that the effect follows a similar dynamic pattern as the increase in investment spending within health (Figure \ref{fig:7b}). In this analysis we leverage the variation in municipal health spending induced by the EC/29 within state governments, so one would not expect to see increases in the number of hospital from other governmental spheres or from the private sector. Our results validates that. Yet, the point estimates for the number of Federal, State and Private hospitals are negative, which could suggest some substitution effects in the supply of hospitals. These results would be worrisome if the effects were large, as mortality outcomes can be affected by the supply of hospitals, but that is not what the point estimates and Figure \ref{fig:11b} \ref{fig:11b} suggests. Our results also indicate some marginally significant reduction in the number of health facilities with ambulatory service, but this effect is rather small, representing a reduction of 1.5\% relative to the baseline for the representative municipality.

\begin{figure}[h!]
    \begin{center}
    \caption{Effects on Health Infrastructure}\label{fig:11}
    \begin{subfigure}{0.48\textwidth}
        \caption{\scriptsize N. of Municipal Hospitals (per capita*1000)}\label{fig:11a}
        \centering
\includegraphics[width=\textwidth]{plots/ams_hospital_mun_pcapita_dist_ec29_baseline_dist_ec29_baseline_11.pdf}
    \end{subfigure}
    \begin{subfigure}{0.48\textwidth}
        \centering
        \caption{\scriptsize N. of Federal and State Hospitals (per capita*1000)}\label{fig:11b}
        \includegraphics[width=\textwidth]{plots/ams_hospital_nmun_pcapita_dist_ec29_baseline_dist_ec29_baseline_11.pdf}
    \end{subfigure}
    \begin{subfigure}{0.48\textwidth}
        \centering
        \caption{\scriptsize N. of Private Hospitals (per capita*1000)}\label{fig:11c}
        \includegraphics[width=\textwidth]{plots/ams_hospital_pvt_pcapita_dist_ec29_baseline_dist_ec29_baseline_11.pdf}
    \end{subfigure}
    \begin{subfigure}{0.48\textwidth}
        \centering
        \caption{\scriptsize N. of Health Facilities with Ambulatory Service (per capita*1000)}\label{fig:11d}
        \includegraphics[width=\textwidth]{plots/sia_ncnes_amb_mun_pcapita_dist_ec29_baseline_dist_ec29_baseline_11.pdf}
    \end{subfigure}
    
    \end{center}
    \scriptsize{Notes: The number of observations is 19364 for Figure \ref{fig:11a}, \ref{fig:11b}, \ref{fig:11c} and 48916 for the remaining. DiD Estimates from Equation \ref{eq:2}. Independent variable is the distance to the EC/29 target in p.p. Square dots represent the baseline model with municipality and state-year fixed effects. Round dots represent fully saturated specification (Column 4 in regression Tables). Lines represent 95\% confidence intervals. Arrows, when present, indicate confidence intervals out of the plot bounds. Standard errors are clustered in the municipality level.}
    
\end{figure}

We also find significant effects on the number of health facilities with ambulatorial services and professionals related to primary care (Panel E of Table \ref{table:infra} and Figure \ref{fig:12}), with effects ranging from 5\% to 10\% relative to the baseline.

\begin{figure}[h!]
    \begin{center}
    \caption{Effects on Primary Care Coverage - Intensive Margin}\label{fig:12}
    \begin{subfigure}{0.32\textwidth}
        \caption{\scriptsize N. of People Visited}\label{fig:12a}
        \centering
        \includegraphics[width=\textwidth]{plots/siab_accomp_especif_pcapita_dist_ec29_baseline_dist_ec29_baseline_12.pdf}
    \end{subfigure}
    \begin{subfigure}{0.32\textwidth}
        \centering
        \caption{\scriptsize People Visited by CH Agents}\label{fig:12b}
        \includegraphics[width=\textwidth]{plots/siab_accomp_especif_pacs_pcapita_dist_ec29_baseline_dist_ec29_baseline_12.pdf}
    \end{subfigure}
    \begin{subfigure}{0.32\textwidth}
        \centering
        \caption{\scriptsize People Visited by FH Agents}\label{fig:12c}
        \includegraphics[width=\textwidth]{plots/siab_accomp_especif_psf_pcapita_dist_ec29_baseline_dist_ec29_baseline_12.pdf}
    \end{subfigure}
        \begin{subfigure}{0.32\textwidth}
        \caption{\scriptsize N. of Household Visits}\label{fig:12d}
        \centering
        \includegraphics[width=\textwidth]{plots/siab_visit_cha_pcapita_dist_ec29_baseline_dist_ec29_baseline_12.pdf}
    \end{subfigure}
    \begin{subfigure}{0.32\textwidth}
        \centering
        \caption{\scriptsize N. of Household Visits by CH Agents}\label{fig:12e}
        \includegraphics[width=\textwidth]{plots/siab_visit_cha_pacs_pcapita_dist_ec29_baseline_dist_ec29_baseline_12.pdf}
    \end{subfigure}
    \begin{subfigure}{0.32\textwidth}
        \centering
        \caption{\scriptsize N. of Household Visits by FH Agents}\label{fig:12f}
        \includegraphics[width=\textwidth]{plots/siab_visit_cha_psf_pcapita_dist_ec29_baseline_dist_ec29_baseline_12.pdf}
    \end{subfigure}
        \begin{subfigure}{0.32\textwidth}
        \caption{\scriptsize N. of Appointments}\label{fig:12g}
        \centering
        \includegraphics[width=\textwidth]{plots/siab_cons_especif_pcapita_dist_ec29_baseline_dist_ec29_baseline_12.pdf}
    \end{subfigure}
    \begin{subfigure}{0.32\textwidth}
        \centering
        \caption{\scriptsize N. of Appointments from CH Program}\label{fig:12h}
        \includegraphics[width=\textwidth]{plots/siab_cons_especif_pacs_pcapita_dist_ec29_baseline_dist_ec29_baseline_12.pdf}
    \end{subfigure}
    \begin{subfigure}{0.32\textwidth}
        \centering
        \caption{\scriptsize N. of Appointments from FH Program}\label{fig:12i}
        \includegraphics[width=\textwidth]{plots/siab_cons_especif_psf_pcapita_dist_ec29_baseline_dist_ec29_baseline_12.pdf}
    \end{subfigure}
    
    \end{center}
    
\end{figure}

Finally, in Table \ref{table:production} we assess the impacts on ambulatory production and on the access to health services. Panel A show significant effects on the number of outpatient procedures, primary care outpatient procedures and outpatient procedures of low and mid complexity. These point estimates represent a considerably small increase in production, between 2-3\%. We find no significant impact on outpatient procedures of high complexity. The dynamic effects for these outcomes are presented in Figure \ref{fig:14}. In Panel B we present our estimates for the access to health services, measured by prenatal visits. The results show a significant decrease of 0.093 in prenatal visits ignored, that measures under-registration of information on birth records, and a increase of 0.116 in 1 to 6 prenatal visits. These results suggests an improvement in data registration, and considerably small effect on prenatal 1-6. If we consider only the effect beyond the reduction in under-registration, the effect will be equivalent to only 0.4\% increase relative to the baseline for the representative municipality. Figure \ref{fig:13} present the dynamic effects for prenatal visits. Figure \ref{fig:13b} suggests that the EC/29 might have had some effect in reducing the share of births with mothers having no prenatal visits, which could explain this increase in prenatal 1-6 above the reduction in under-registration.

\begin{table}[H]
\begin{footnotesize}
\begin{center}
\scalebox{0.8}{
\begin{threeparttable}[b]

  \centering
  \caption{Ambulatorial Production and Access to Health Services}
     \begin{tabular}{rrrrrr}
          &       &       &       &       &  \\
    \midrule
    \midrule
          &       & \multicolumn{4}{c}{Distance to EC9 target} \\
\cmidrule{3-6}          &       & \multicolumn{1}{c}{(1)} & \multicolumn{1}{c}{(2)} & \multicolumn{1}{c}{(3)} & \multicolumn{1}{c}{(4)} \\
    \midrule
    \multicolumn{1}{p{26.355em}}{\textbf{A. Ambulatorial Production}} &       &       &       &       &  \\
    \multicolumn{1}{l}{\multirow{2}[0]{*}{Outpatient procedures per capita}} &       & \multicolumn{1}{c}{2.67***} & \multicolumn{1}{c}{2.473**} & \multicolumn{1}{c}{2.544**} & \multicolumn{1}{c}{ 2.528** } \\
          &       & \multicolumn{1}{c}{(1.027)} & \multicolumn{1}{c}{(1.025)} & \multicolumn{1}{c}{(1.021)} & \multicolumn{1}{c}{ (1.019) } \\
    \multicolumn{1}{l}{\multirow{2}[0]{*}{Primary Care Outpatient procedures per capita}} &       & \multicolumn{1}{c}{2.231**} & \multicolumn{1}{c}{2.228**} & \multicolumn{1}{c}{2.282**} & \multicolumn{1}{c}{ 2.266** } \\
          &       & \multicolumn{1}{c}{(0.944)} & \multicolumn{1}{c}{(0.934)} & \multicolumn{1}{c}{(0.931)} & \multicolumn{1}{c}{ (0.929) } \\
    \multicolumn{1}{l}{\multirow{2}[0]{*}{N. of Low \& Mid Complexity Outpatient Procedures (per capita)}} &       & \multicolumn{1}{c}{2.337**} & \multicolumn{1}{c}{2.225**} & \multicolumn{1}{c}{2.331***} & \multicolumn{1}{c}{ 2.337*** } \\
          &       & \multicolumn{1}{c}{(0.908)} & \multicolumn{1}{c}{(0.898)} & \multicolumn{1}{c}{(0.892)} & \multicolumn{1}{c}{ (0.891) } \\
    \multicolumn{1}{l}{\multirow{2}[0]{*}{N. of High Complexity Outpatient Procedures (per capita)}} &       & \multicolumn{1}{c}{-0.136} & \multicolumn{1}{c}{-0.191} & \multicolumn{1}{c}{-0.174} & \multicolumn{1}{c}{ -0.177 } \\
          &       & \multicolumn{1}{c}{(0.134)} & \multicolumn{1}{c}{(0.133)} & \multicolumn{1}{c}{(0.132)} & \multicolumn{1}{c}{ (0.132) } \\
    \multicolumn{1}{l}{\multirow{2}[0]{*}{N. of Outpatient Procedures by Low Skilled Workers (per capita)}} &       & \multicolumn{1}{c}{0.04} & \multicolumn{1}{c}{0.019} & \multicolumn{1}{c}{0.074} & \multicolumn{1}{c}{ 0.074 } \\
          &       & \multicolumn{1}{c}{(0.354)} & \multicolumn{1}{c}{(0.346)} & \multicolumn{1}{c}{(0.343)} & \multicolumn{1}{c}{ (0.342) } \\
    \multicolumn{1}{l}{\multirow{2}[0]{*}{N. of Outpatient procedures by Mid Skilled Workers (per capita)}} &       & \multicolumn{1}{c}{0.383} & \multicolumn{1}{c}{0.346} & \multicolumn{1}{c}{0.353} & \multicolumn{1}{c}{ 0.351 } \\
          &       & \multicolumn{1}{c}{(0.481)} & \multicolumn{1}{c}{(0.479)} & \multicolumn{1}{c}{(0.479)} & \multicolumn{1}{c}{ (0.479) } \\
          &       &       &       &       &  \\
    \midrule
    \multicolumn{1}{p{26.355em}}{\textbf{B. Access to Health Services}} &       &       &       &       &  \\
    \multicolumn{1}{l}{\multirow{2}[0]{*}{Prenatal Visits None}} &       & \multicolumn{1}{c}{0.025*} & \multicolumn{1}{c}{0.006} & \multicolumn{1}{c}{0.005} & \multicolumn{1}{c}{0.005} \\
          &       & \multicolumn{1}{c}{(0.015)} & \multicolumn{1}{c}{(0.011)} & \multicolumn{1}{c}{(0.011)} & \multicolumn{1}{c}{(0.011)} \\
    \multicolumn{1}{l}{\multirow{2}[0]{*}{Prenatal Visits 1-6}} &       & \multicolumn{1}{c}{0.13**} & \multicolumn{1}{c}{0.121**} & \multicolumn{1}{c}{0.118*} & \multicolumn{1}{c}{0.116*} \\
          &       & \multicolumn{1}{c}{(0.059)} & \multicolumn{1}{c}{(0.061)} & \multicolumn{1}{c}{(0.061)} & \multicolumn{1}{c}{(0.06)} \\
    \multicolumn{1}{l}{\multirow{2}[0]{*}{Prenatal Visits 7+}} &       & \multicolumn{1}{c}{-0.051} & \multicolumn{1}{c}{-0.034} & \multicolumn{1}{c}{-0.03} & \multicolumn{1}{c}{-0.029} \\
          &       & \multicolumn{1}{c}{(0.074)} & \multicolumn{1}{c}{(0.062)} & \multicolumn{1}{c}{(0.061)} & \multicolumn{1}{c}{(0.061)} \\
          &       &       &       &       &  \\
    \bottomrule
    \bottomrule
    \end{tabular}%
    
    
  \label{table:production}%

\end{threeparttable}
}
\end{center}
\end{footnotesize}
\end{table}

\begin{figure}[h!]
    \begin{center}
    \caption{Effects on Ambulatorial Production}\label{fig:14}
    \begin{subfigure}{0.48\textwidth}
        \centering
        \caption{\scriptsize Total}\label{fig:14a}
        \includegraphics[width=\textwidth]{plots/sia_pcapita_dist_ec29_baseline_dist_ec29_baseline_14.pdf}
    \end{subfigure}
    \begin{subfigure}{0.48\textwidth}
        \centering
        \caption{\scriptsize Primary Care}\label{fig:14b}
        \includegraphics[width=\textwidth]{plots/sia_ab_pcapita_dist_ec29_baseline_dist_ec29_baseline_14.pdf}
    \end{subfigure}
    \begin{subfigure}{0.48\textwidth}
        \centering
        \caption{\scriptsize Low and Mid Complexity}\label{fig:14c}
        \includegraphics[width=\textwidth]{plots/sia_nprod_amb_lc_mun_pcapita_dist_ec29_baseline_dist_ec29_baseline_14.pdf}
    \end{subfigure}
    \begin{subfigure}{0.48\textwidth}
        \centering
        \caption{\scriptsize High Complexity}\label{fig:14d}
        \includegraphics[width=\textwidth]{plots/sia_nprod_amb_hc_mun_pcapita_dist_ec29_baseline_dist_ec29_baseline_14.pdf}
    \end{subfigure}
    
    \end{center}
    
        \scriptsize{Notes: The number of observations is 64482 for \ref{fig:14a} and \ref{fig:14b}, 48916 for the remaining. DiD Estimates from Equation \ref{eq:2}. Independent variable is the distance to the EC/29 target in p.p. Square dots represent the baseline model with municipality and state-year fixed effects. Round dots represent fully saturated specification (Column 4 in regression Tables). Lines represent 95\% confidence intervals. Arrows, when present, indicate confidence intervals out of the plot bounds. Standard errors are clustered in the municipality level.}
    
\end{figure}

\begin{figure}[h!]
    \begin{center}
    \caption{Effects on Infrastructure and Human Resources: N. of Health Facilities with:}\label{fig:13}
    \begin{subfigure}{0.32\textwidth}
        \caption{\scriptsize Ambulatory Service}\label{fig:13a}
        \centering
        \includegraphics[width=\textwidth]{plots/sia_ncnes_amb_mun_pcapita_dist_ec29_baseline_dist_ec29_baseline_13.pdf}
    \end{subfigure}
    \begin{subfigure}{0.32\textwidth}
        \centering
        \caption{\scriptsize Ambulatory Service and PSF Teams}\label{fig:13b}
        \includegraphics[width=\textwidth]{plots/sia_ncnes_psf_pcapita_dist_ec29_baseline_dist_ec29_baseline_13.pdf}
    \end{subfigure}
    \begin{subfigure}{0.32\textwidth}
        \centering
        \caption{\scriptsize Ambulatory Service and ACS Teams}\label{fig:13c}
        \includegraphics[width=\textwidth]{plots/sia_ncnes_acs_pcapita_dist_ec29_baseline_dist_ec29_baseline_13.pdf}
    \end{subfigure}
        \begin{subfigure}{0.32\textwidth}
        \caption{\scriptsize Ambulatory Service and Community Doctors}\label{fig:13d}
        \centering
        \includegraphics[width=\textwidth]{plots/sia_ncnes_medcom_pcapita_dist_ec29_baseline_dist_ec29_baseline_13.pdf}
    \end{subfigure}
    \begin{subfigure}{0.32\textwidth}
        \centering
        \caption{\scriptsize Ambulatory Service and PSF Doctors}\label{fig:13e}
        \includegraphics[width=\textwidth]{plots/sia_ncnes_medpsf_pcapita_dist_ec29_baseline_dist_ec29_baseline_13.pdf}
    \end{subfigure}
    \begin{subfigure}{0.32\textwidth}
        \centering
        \caption{\scriptsize Ambulatory Service and PSF Nurses}\label{fig:13f}
        \includegraphics[width=\textwidth]{plots/sia_ncnes_enfpsf_pcapita_dist_ec29_baseline_dist_ec29_baseline_13.pdf}
    \end{subfigure}
        \begin{subfigure}{0.32\textwidth}
        \caption{\scriptsize Ambulatory Service and PSF Nursing Assistants}\label{fig:13g}
        \centering
        \includegraphics[width=\textwidth]{plots/sia_ncnes_outpsf_pcapita_dist_ec29_baseline_dist_ec29_baseline_13.pdf}
    \end{subfigure}
    \begin{subfigure}{0.32\textwidth}
        \centering
        \caption{\scriptsize Ambulatory Service and ACS Nurses}\label{fig:13h}
        \includegraphics[width=\textwidth]{plots/sia_ncnes_enfacs_pcapita_dist_ec29_baseline_dist_ec29_baseline_13.pdf}
    \end{subfigure}
    
    \end{center}
    
\end{figure}

With the data available we are not able to directly connect the increase in health spending with the increase in health inputs presented in this section. However, the evidence presented so far suggests that: (i) increases in human resource spending have been translated into greater primary care coverage at the intensive margin, a higher number of facilities with primary care personnel, and into a increase in the number of nursing assistants; (ii) increases in investment spending has been translated into a greater supply of municipal hospitals and a marginal increase ambulatory production; and (iii) increases in other expenditures, which consist mainly of administrative spending, might be associated with the increase in the number of administrative professionals.


\subsection{Effects on Infant Mortality}

Having provided meaningful evidence of the effects of EC/29 on health spending and how these effects translated into health inputs, we now present estimates of the effects on infant mortality. Differently from most of the literature linking health spending with infant mortality \citep{filmer1999,bokhari2007,moreno2015,nixon2006,gupta2002effectiveness,cremieux1999,bokhari2007}, we are able to assess the effects not only for total infant mortality rates, but also for infant mortality rates by timing of death and by cause of death. We are also able to analyse infant mortality rates by classifying them between amenable and non-amenable to primary care. These results are presented in Table \ref{table:imr}. In all specifications presented for this section, we added a trend of baseline ill-defined infant mortality with the goal of accounting for mortality under-reporting\footnote{Appendix Table \ref{app:imr} presents estimates with and without the baseline ill-defined infant mortality trend.}. During the period of analysis, the completeness of death counts improved considerably and it is strongly associated with the reduction in ill-defined causes of death \citep{lima2014evolution}.

\begin{table}[h!]
\begin{footnotesize}
\begin{center}
\scalebox{0.9}{
\begin{threeparttable}[b]

  \centering
  \caption{Infant Mortality Rates}
     \begin{tabular}{rrcccc}
          &       &       &       &       &  \\
          &       &       &       &       &  \\
    \midrule
    \midrule
          &       & (1)   & (2)   & (3)   & (4) \\
    \midrule
    \multicolumn{1}{p{15.145em}}{\textbf{A. Infant Mortality Rate}} &       &       &       &       &  \\
    \multicolumn{1}{p{15.145em}}{Total} &       & -5.015 & -3.772 & -3.831 & -3.889 \\
          &       & (3.435) & (2.853) & (2.836) & (2.828) \\
    \multicolumn{1}{p{15.145em}}{Amenable to Primary Care} &       & -0.361 & -0.866 & -0.893 & -0.905 \\
          &       & (0.603) & (0.553) & (0.553) & (0.554) \\
    \multicolumn{1}{p{15.145em}}{Non-Amenable to Primary Care} &       & -4.653 & -2.907 & -2.939 & -2.984 \\
          &       & (3.245) & (2.645) & (2.632) & (2.624) \\
          &       &       &       &       &  \\
    \midrule
    \multicolumn{1}{p{15.145em}}{\textbf{B. By timing}} &       &       &       &       &  \\
    \multicolumn{1}{p{15.145em}}{Fetal} &       & -0.008 & -0.007 & -0.008 & -0.008 \\
          &       & (0.008) & (0.008) & (0.008) & (0.008) \\
    \multicolumn{1}{p{15.145em}}{Within 24h} &       & -2.275* & -2.083** & -2.07** & -2.071** \\
          &       & (1.225) & (0.98) & (0.979) & (0.976) \\
    \multicolumn{1}{p{15.145em}}{1 to 27 days} &       & -4.228* & -2.883 & -2.911 & -2.922 \\
          &       & (2.555) & (2.064) & (2.052) & (2.046) \\
    \multicolumn{1}{p{15.145em}}{27 days to 1 year} &       & -0.787 & -0.89 & -0.92 & -0.967 \\
          &       & (1.435) & (1.248) & (1.246) & (1.243) \\
          &       &       &       &       &  \\
    \midrule
    \multicolumn{1}{p{15.145em}}{\textbf{C. By Cause of Death}} &       &       &       &       &  \\
    \multicolumn{1}{p{15.145em}}{Infectious} &       & -0.374 & -0.811 & -0.82 & -0.831 \\
          &       & (0.567) & (0.535) & (0.535) & (0.534) \\
    \multicolumn{1}{p{15.145em}}{Respiratory} &       & -0.494 & -0.507 & -0.511 & -0.517 \\
          &       & (0.474) & (0.411) & (0.409) & (0.409) \\
    \multicolumn{1}{p{15.145em}}{Perinatal} &       & -5.349** & -3.648* & -3.69* & -3.707* \\
          &       & (2.571) & (2.015) & (2.007) & (2.002) \\
    \multicolumn{1}{p{15.145em}}{Congenital} &       & -0.235 & -0.169 & -0.16 & -0.157 \\
          &       & (0.463) & (0.436) & (0.434) & (0.434) \\
    \multicolumn{1}{p{15.145em}}{External} &       & 0.024 & -0.049 & -0.037 & -0.034 \\
          &       & (0.183) & (0.165) & (0.165) & (0.166) \\
    \multicolumn{1}{p{15.145em}}{Nutritional} &       & -0.204 & -0.328 & -0.33 & -0.343 \\
          &       & (0.246) & (0.231) & (0.232) & (0.232) \\
    \multicolumn{1}{p{15.145em}}{Other} &       & -0.183 & -0.123 & -0.132 & -0.139 \\
          &       & (0.201) & (0.199) & (0.198) & (0.198) \\
    \multicolumn{1}{p{15.145em}}{Ill-Defined} &       & 1.8** & 1.862** & 1.849** & 1.84** \\
          &       & (0.849) & (0.776) & (0.779) & (0.779) \\
          &       &       &       &       &  \\
    \bottomrule
    \bottomrule
    \end{tabular}%
    
    
    \begin{tablenotes}
  \scriptsize{\underline{Notes}: The number of observations is 64482. DiD Estimates from Equation \ref{eq:1}. Independent variable is the distance to the EC/29 target in p.p. Column 1 presents the baseline model with municipality and state-year fixed effects. Column 2 adds baseline socioeconomic controls from the Census interacted with time. Column 3 adds controls for GDP per capita and \emph{Bolsa Familia} transfers per capita. Column 4 adds fiscal controls. Covariates omitted. Standard errors in brackets are clustered in the municipality level. ∗p < 0.10, ∗ ∗ p < 0.05, ∗ ∗ ∗p < 0.01}
  \end{tablenotes}
    
    
  \label{table:imr}%

\end{threeparttable}
}
\end{center}
\end{footnotesize}
\end{table}

Panel A present the estimates for total infant mortality rates (IMR) and IMR amenable and non-amenable to primary care. Though not significant, the estimates present the expected sign. Yet, the more flexible coefficients estimated with equation \ref{eq:2} provide useful information on the dynamics of the effects and suggest the presence of some significant reduction in IMR. Figure \ref{fig:15} plots the dynamic effects for the IMR presented in Panel A. IMR (Figure \ref{fig:15a}) and IMR amenable to primary care (Figure \ref{fig:15b})  present a clear trend of reduction, with estimates for 2007 onward being all statistically significant in our preferred specification. 

\begin{figure}[h!]
    \begin{center}
    \caption{Effects on Ambulatorial Production}\label{fig:15}
    \begin{subfigure}{0.32\textwidth}
        \centering
        \caption{\scriptsize Total}\label{fig:15a}
        \includegraphics[width=\textwidth]{plots/sia_pcapita_dist_ec29_baseline_dist_ec29_baseline_15.pdf}
    \end{subfigure}
    \begin{subfigure}{0.32\textwidth}
        \centering
        \caption{\scriptsize Primary Care}\label{fig:15b}
        \includegraphics[width=\textwidth]{plots/sia_ab_pcapita_dist_ec29_baseline_dist_ec29_baseline_15.pdf}
    \end{subfigure}
    \begin{subfigure}{0.32\textwidth}
        \centering
        \caption{\scriptsize Low and Mid Complexity}\label{fig:15c}
        \includegraphics[width=\textwidth]{plots/sia_nprod_amb_lc_mun_pcapita_dist_ec29_baseline_dist_ec29_baseline_15.pdf}
    \end{subfigure}
    \begin{subfigure}{0.32\textwidth}
        \centering
        \caption{\scriptsize High Complexity}\label{fig:15d}
        \includegraphics[width=\textwidth]{plots/sia_nprod_amb_hc_mun_pcapita_dist_ec29_baseline_dist_ec29_baseline_15.pdf}
    \end{subfigure}
    \begin{subfigure}{0.32\textwidth}
        \centering
        \caption{\scriptsize Performed by Low Skilled Worker}\label{fig:15e}
        \includegraphics[width=\textwidth]{plots/sia_nprod_low_skill_mun_pcapita_dist_ec29_baseline_dist_ec29_baseline_15.pdf}
    \end{subfigure}
    \begin{subfigure}{0.32\textwidth}
        \centering
        \caption{\scriptsize Performed by High Skilled Worker}\label{fig:15f}
        \includegraphics[width=\textwidth]{plots/sia_nprod_med_skill_mun_pcapita_dist_ec29_baseline_dist_ec29_baseline_15.pdf}
    \end{subfigure}
    
    \end{center}
    
        \scriptsize{Notes: The number of observations is 64710 for \ref{fig:15a} and \ref{fig:15b}, 49080 for the remaining. DiD Estimates from Equation \ref{eq:2}. Independent variable is the distance to the EC/29 target in p.p. Square dots represent the baseline model with municipality and state-year fixed effects. Round dots represent fully saturated specification (Column 4 in regression Tables). Lines represent 95\% confidence intervals. Arrows, when present, indicate confidence intervals out of the plot bounds. Standard errors are clustered in the municipality level.}
    
\end{figure}

Next, we present the estimates for IMR by timing of death (Panel B). IMR withing 24 hours shows a significant reduction of 2.071, which is equivalent to a 3.7\% decreased in baseline mortality for the representative municipality. The estimates for the remaining IMR of Panel B are not significant, but the dynamic effects suggests the presence of reduction trend in later years for IMR from 1 to 27 days (Figure \ref{fig:16c}) and IMR from 27 days to 1 year old (Figure \ref{fig:16d}).

\begin{figure}[h!]
    \begin{center}
    \caption{Effects on Infant Mortality Rates}\label{fig:16}
    \begin{subfigure}{0.32\textwidth}
        \caption{\scriptsize Total}\label{fig:16a}
        \centering
        \includegraphics[width=\textwidth]{plots/tx_mi_dist_ec29_baseline_dist_ec29_baseline_16.pdf}
    \end{subfigure}
    \begin{subfigure}{0.32\textwidth}
        \centering
        \caption{\scriptsize Amenable to Primary Care}\label{fig:16b}
        \includegraphics[width=\textwidth]{plots/tx_mi_icsap_dist_ec29_baseline_dist_ec29_baseline_16.pdf}
    \end{subfigure}
    \begin{subfigure}{0.32\textwidth}
        \centering
        \caption{\scriptsize Non-Amenable to Primary Care}\label{fig:16c}
        \includegraphics[width=\textwidth]{plots/tx_mi_nicsap_dist_ec29_baseline_dist_ec29_baseline_16.pdf}
    \end{subfigure}
    
    \end{center}
    
\end{figure}

The last part of Table \ref{table:imr} present the estimates for IMR by causes of death. We only find marginally significant reductions for IMR from Perinatal causes. This estimated represents a 3.4\% decrease relative to baseline mortality. We also find the unexpected effect of increases in IMR from ill-defined causes. Moreover, the dynamic effects presented in Figure \ref{fig:17} suggest the presence of some reduction in IMR by infectious and in IMR by respiratory causes.

In our preferred specification, with the exception of ill-defined infant mortality, all outcomes present negative point estimates, but only infant mortality rate within 24 hours and infant mortality rate by infectious diseases present marginally significant estimates. Relative to the baseline mortality rates, a 10\% distance to the target is associated with a reduction of around 3.5\% for mortality within 24 hours and a reduction of around 5\% for mortality due to infectious diseases. 

\begin{figure}[h!]
    \begin{center}
    \caption{Effects on Infant Mortality Rates - By Timing}\label{fig:17}
    \begin{subfigure}{0.48\textwidth}
        \caption{\scriptsize Fetal}\label{fig:17a}
        \centering
        \includegraphics[width=\textwidth]{plots/tx_mi_fet_dist_ec29_baseline_dist_ec29_baseline_17.pdf}
    \end{subfigure}
    \begin{subfigure}{0.48\textwidth}
        \centering
        \caption{\scriptsize Within 24h}\label{fig:17b}
        \includegraphics[width=\textwidth]{plots/tx_mi_24h_dist_ec29_baseline_dist_ec29_baseline_17.pdf}
    \end{subfigure}
    \begin{subfigure}{0.48\textwidth}
        \centering
        \caption{\scriptsize 1 to 27 days}\label{fig:17c}
        \includegraphics[width=\textwidth]{plots/tx_mi_27d_dist_ec29_baseline_dist_ec29_baseline_17.pdf}
    \end{subfigure}
    \begin{subfigure}{0.48\textwidth}
        \centering
        \caption{\scriptsize 27 days to 1 year}\label{fig:17d}
        \includegraphics[width=\textwidth]{plots/tx_mi_ano_dist_ec29_baseline_dist_ec29_baseline_17.pdf}
    \end{subfigure}
    
    \end{center}
    
            \scriptsize{Notes: The number of observations is 64701. DiD Estimates from Equation \ref{eq:2}. Independent variable is the distance to the EC/29 target in p.p. Square dots represent the baseline model with municipality and state-year fixed effects. Round dots represent fully saturated specification (Column 4 in regression Tables). Lines represent 95\% confidence intervals. Arrows, when present, indicate confidence intervals out of the plot bounds. Standard errors are clustered in the municipality level.}
    
\end{figure}

In general, articles estimating the causal relationship between health spending and mortality run log-log regressions and present estimates for the elasticity of mortality with respect to health spending. We explicitly choose not to apply transformations to our health outcomes variables due to the amount of observations with values equal to $0$, notably the ones related to birth and mortality. Our data comprises all the Brazilian municipalities with available data for the period of analyses, some with population size as little as $700$ inhabitants, and it is common to find infant mortality rates of $0$. Running log transformation would throw away some relevant information for several outcomes. Nonetheless, to relate our results to the literature on this topic we estimate "back of the envelope" elasticities for all IMR using the estimates of our regressions. Table \ref{table:elasticity} presents these elasticities. 

The elasticities presented in the literature vary greatly. Within cross-country studies, while \cite{filmer1999} finds a very small elasticity of $-0.08$, \cite{gupta2002effectiveness} finds an elasticity of $-0.31$ and \cite{bokhari2007} elasticities ranging between $-0.4$ and $-0.5$. In the micro studies \cite{cremieux1999} finds large elasticities between $-0.8$ and $-1.1$, \cite{sonia2007} finds an elasticity of $-0.24$ for rural regions, and \cite{castro2021effects} elasticities ranging between $-0.5$ and $-0.9$. Though not significant, our study finds much smaller elasticities for total infant mortality rates. Using SIOPS as the measure of health spending, we find IMR elasticities ranging between $-0.06$ and $-0.14$. When it comes to within 24 hours IMR and Perinatal IMR, the rates for which we found significant decreases, the back of the envelope calculations suggest an elasticity of $-0.136$ and $-0.124$, respectively.


\begin{table}[h!]
\begin{footnotesize}
\begin{center}
\scalebox{0.8}{
\begin{threeparttable}[b]

  \centering
  \caption{Fertility and Birth Outcomes}
     \begin{tabular}{rrcccr}
          &       &       &       &       &  \\
          &       &       &       &       &  \\
    \midrule
    \midrule
          &       & (1)   & (2)   & (3)   & \multicolumn{1}{c}{(4)} \\
    \midrule
    \multicolumn{1}{l}{\textbf{A. Fertility}} &       &       &       &       &  \\
    \multicolumn{1}{p{26.355em}}{Rates of Birth per Woman (10-49y)} &       & 0.009** & 0.008** & 0.009** & \multicolumn{1}{c}{ 0.009** } \\
          &       & (0.004) & (0.003) & (0.003) & \multicolumn{1}{c}{ (0.003) } \\
    \multicolumn{1}{p{26.355em}}{\textbf{B. Birth Outcomes}} &       &       &       &       &  \\
    \multicolumn{1}{p{26.355em}}{Apgar 1} &       & -0.056 & 0.063 & 0.053 & \multicolumn{1}{c}{ 0.051 } \\
          &       & (0.206) & (0.198) & (0.198) & \multicolumn{1}{c}{ (0.198) } \\
    \multicolumn{1}{p{26.355em}}{Apgar 5} &       & 0.009 & 0.107 & 0.104 & \multicolumn{1}{c}{ 0.101 } \\
          &       & (0.183) & (0.179) & (0.18) & \multicolumn{1}{c}{ (0.179) } \\
    \multicolumn{1}{p{26.355em}}{Low Birth Weight (<2.5k)} &       & -0.003 & -0.002 & -0.001 & \multicolumn{1}{c}{ -0.002 } \\
          &       & (0.003) & (0.003) & (0.003) & \multicolumn{1}{c}{ (0.003) } \\
    \multicolumn{1}{p{26.355em}}{Premature Birth} &       & -0.005 & -0.016 & -0.017 & \multicolumn{1}{c}{ -0.017 } \\
          &       & (0.026) & (0.023) & (0.023) & \multicolumn{1}{c}{ (0.023) } \\
    \multicolumn{1}{p{26.355em}}{Sex Ratio at Birth} &       & 0.014 & 0.016 & 0.017 & \multicolumn{1}{c}{ 0.017 } \\
          &       & (0.016) & (0.016) & (0.016) & \multicolumn{1}{c}{ (0.016) } \\
          &       &       &       &       &  \\
    \bottomrule
    \bottomrule
    \end{tabular}%
    
    
    \begin{tablenotes}
  \scriptsize{\underline{Notes}: The number of observations is 64482 for Panel A, 63705 for Apgar 1, 59524 for Apgar 5, 64481 for Low Birth Weight and Premature Birth, and 64470 for Sex Ratio at Birth. DiD Estimates from Equation \ref{eq:1}. Independent variable is the distance to the EC/29 target in p.p. Column 1 presents the baseline model with municipality and state-year fixed effects. Column 2 adds baseline socioeconomic controls from the Census interacted with time. Column 3 adds controls for GDP per capita and \emph{Bolsa Familia} transfers per capita. Column 4 adds fiscal controls. Covariates omitted. Standard errors in brackets are clustered in the municipality level. ∗p < 0.10, ∗ ∗ p < 0.05, ∗ ∗ ∗p < 0.011}
  \end{tablenotes}
    
    
  \label{table:birth}%

\end{threeparttable}
}
\end{center}
\end{footnotesize}
\end{table}

Lastly, table \ref{table:birth} presents the estimates for fertility and birth outcomes. In general, the point estimates are in the expected direction, substantially small,but and statistically insignificant. An exception is our measure of fertility that presents small but significant estimates. A 0.009 increase in the fertility rate, correspond to a 1.6\% variation relative to the baseline for the representative municipality. Figure \ref{fig:18} plots the dynamic effects for fertility and birth outcomes. Figure \ref{fig:18a} suggests the presence of some trend in our fertility measure, that seems to show a constant increase from the pre-treatment period until the last year of analysis.

\begin{figure}[h!]
    \begin{center}
    \caption{Effects on Infant Mortality Rates - By Cause}\label{fig:18}
    \begin{subfigure}{0.32\textwidth}
        \caption{\scriptsize Infectious}\label{fig:18a}
        \centering
        \includegraphics[width=\textwidth]{plots/tx_mi_infec_dist_ec29_baseline_dist_ec29_baseline_18.pdf}
    \end{subfigure}
    \begin{subfigure}{0.32\textwidth}
        \centering
        \caption{\scriptsize Respiratory}\label{fig:18b}
        \includegraphics[width=\textwidth]{plots/tx_mi_resp_dist_ec29_baseline_dist_ec29_baseline_18.pdf}
    \end{subfigure}
    \begin{subfigure}{0.32\textwidth}
        \centering
        \caption{\scriptsize Perinatal}\label{fig:18c}
        \includegraphics[width=\textwidth]{plots/tx_mi_perinat_dist_ec29_baseline_dist_ec29_baseline_18.pdf}
    \end{subfigure}
        \begin{subfigure}{0.32\textwidth}
        \centering
        \caption{\scriptsize Congenital}\label{fig:18d}
        \includegraphics[width=\textwidth]{plots/tx_mi_cong_dist_ec29_baseline_dist_ec29_baseline_18.pdf}
    \end{subfigure}
        \begin{subfigure}{0.32\textwidth}
        \centering
        \caption{\scriptsize External}\label{fig:18e}
        \includegraphics[width=\textwidth]{plots/tx_mi_ext_dist_ec29_baseline_dist_ec29_baseline_18.pdf}
    \end{subfigure}
        \begin{subfigure}{0.32\textwidth}
        \centering
        \caption{\scriptsize Nutritional}\label{fig:18f}
        \includegraphics[width=\textwidth]{plots/tx_mi_nut_dist_ec29_baseline_dist_ec29_baseline_18.pdf}
    \end{subfigure}
        \begin{subfigure}{0.32\textwidth}
        \centering
        \caption{\scriptsize Other}\label{fig:18g}
        \includegraphics[width=\textwidth]{plots/tx_mi_out_dist_ec29_baseline_dist_ec29_baseline_18.pdf}
    \end{subfigure}
        \begin{subfigure}{0.32\textwidth}
        \centering
        \caption{\scriptsize Ill-defined}\label{fig:18h}
        \includegraphics[width=\textwidth]{plots/tx_mi_illdef_dist_ec29_baseline_dist_ec29_baseline_18.pdf}
    \end{subfigure}
    \end{center}
    
            \scriptsize{Notes: The number of observations is 64701. DiD Estimates from Equation \ref{eq:2}. Independent variable is the distance to the EC/29 target in p.p. Square dots represent the baseline model with municipality and state-year fixed effects. Round dots represent fully saturated specification (Column 4 in regression Tables). Lines represent 95\% confidence intervals. Arrows, when present, indicate confidence intervals out of the plot bounds. Standard errors are clustered in the municipality level.}
    
\end{figure}


Overall, we found significant effects for infant moralities rates that are mainly associated with increases in access to primary care and community-based health interventions \citep{rocha2010evaluating,bhalotra2019can}, suggesting that the effects found for health inputs related to primary care might be the channel through which spending affected infant mortality.


\subsection{Robustness}

In Section \ref{sec:emp_val} we stressed that the validity of our research design relies on having parallel trends and homogeneity in treatment effect. However, we can only empirically test the first identification hypothesis. For all outcomes discussed in this article, we presented figures plotting the estimates of Equation \ref{eq:2} that captures pre-trends in the outcome variable. With only few exceptions, our estimates show no pre-trends in our outcomes.

Another concern may arise from unobserved events coinciding the approval of the EC/29. The inclusion of state-year fixed effects in all specification helps address this concern, as well as the concern on pre-trends. Moreover, we presented in all regression tables estimates for four specifications that gradually included different set of controls, with figures plotting estimates only for the baseline specification and the more complete specification. Our analysis shows that estimates for health spending and health inputs are highly robust to different specifications. The estimates for outcomes related to birth and mortality are a bit more sensible to the inclusion of baseline controls, but still very robust to different specifications.

\section{Conclusions}\label{sec:conclusion}
\setstretch{1.5}

Our empirical analysis has demonstrated that when municipalities are induced to increase public health spending they do so by increasing mainly spending relative to the administrative structure of public health - roughly half of the increase -  followed by spending with investments and human resources. We also demonstrate that this increase is associated with a higher number of administrative professionals, greater supply of municipal hospitals, and greater primary care coverage at the intensive margin, with also a higher number of health facilities with primary care related professionals. The shifts in spending and health inputs are associated with small to moderate reductions in infant mortality rates related to improvements in primary care access, and long term reductions in total infant mortality rates. \cite{bhalotra2019can} have shown that the combination of access to primary and hospital care leads to better health outcomes relative to only primary care. This is a plausible channel through which the increase in the supply of municipal hospitals might be affecting infant mortality in our analysis.

These results are extremely relevant, specially in a context of a universal an decentralized health system, where provision of health care occurs mainly at the municipal level, and the majority of the resources spent locally comes from local tax incomes, in opposition to intergovernmental transfers. [Discuss transfers vs own resource spending]. We are not able to formally test exactly how health inputs and outcomes would react if municipalities allocated less resources on administrative structure and more resources into investments and personnel, but the evidence here present indicates it could lead to further improvements in health outcomes, and, thus, a more efficient use of resources within the public health sector. 


\clearpage
\pagebreak
\singlespacing  
\bibliographystyle{apalike}
\bibliography{SRC}
\nocite{}




\pagebreak
\appendix
\singlespacing	

\LARGE{\textbf{Appendix}}


\counterwithin{figure}{section}
\counterwithin{table}{section}



\section{Descriptive Statistics}\label{app:stats}

\begin{sidewaystable}
\begin{table}[H]
\begin{footnotesize}
\begin{center}
\scalebox{0.6}{
\begin{threeparttable}[b]

  \centering
  \caption{Descriptive Statistics (at the baseline year)}
    
    \begin{tabular}{rrrrrrrrrrrrrrrrrrrr}
          &       &       &       &       &       &       &       &       &       &       &       &       &       &       &       &       &       &       &  \\
    \midrule
    \midrule
          & \multicolumn{5}{c}{Full Sample}       &       & \multicolumn{5}{c}{Bottom Quartile of OR Spent in Public Health} & \multicolumn{5}{c}{Top Quartile of OR Spent in Public Health} &       &       &  \\
\cmidrule{2-6}\cmidrule{8-17}          & \multicolumn{1}{c}{Mean} & \multicolumn{1}{c}{Std. Dev.} & \multicolumn{1}{c}{Min} & \multicolumn{1}{c}{Max} & \multicolumn{1}{c}{Obs.} &       & \multicolumn{1}{c}{Mean} & \multicolumn{1}{c}{Std. Dev.} & \multicolumn{1}{c}{Min} & \multicolumn{1}{c}{Max} & \multicolumn{1}{c}{Obs.} & \multicolumn{1}{c}{Mean} & \multicolumn{1}{c}{Std. Dev.} & \multicolumn{1}{c}{Min} & \multicolumn{1}{c}{Max} & \multicolumn{1}{c}{Obs.} &       & \multicolumn{1}{c}{Source of Data} & \multicolumn{1}{c}{Baseline Year} \\
\cmidrule{1-6}\cmidrule{8-17}\cmidrule{19-20}    \multicolumn{1}{l}{\textbf{EC 29 Variables}} &       &       &       &       &       &       &       &       &       &       &       &       &       &       &       &       &       &       &  \\
    \multicolumn{1}{l}{Share of Municipality's Own Resource Spent in Public Health} & \multicolumn{1}{c}{0.14} & \multicolumn{1}{c}{0.07} & \multicolumn{1}{c}{0.00} & \multicolumn{1}{c}{0.80} & \multicolumn{1}{c}{5224} &       & \multicolumn{1}{c}{0.06} & \multicolumn{1}{c}{0.02} & \multicolumn{1}{c}{0.00} & \multicolumn{1}{c}{0.09} & \multicolumn{1}{c}{1306} & \multicolumn{1}{c}{0.23} & \multicolumn{1}{c}{0.06} & \multicolumn{1}{c}{0.17} & \multicolumn{1}{c}{0.80} & \multicolumn{1}{c}{1306} &       & \multicolumn{1}{c}{SIOPS} & \multicolumn{1}{c}{2000} \\
    \multicolumn{1}{l}{Distance to the EC29 Target} & \multicolumn{1}{c}{0.01} & \multicolumn{1}{c}{0.07} & \multicolumn{1}{c}{-0.65} & \multicolumn{1}{c}{0.15} & \multicolumn{1}{c}{5224} &       & \multicolumn{1}{c}{0.09} & \multicolumn{1}{c}{0.02} & \multicolumn{1}{c}{0.06} & \multicolumn{1}{c}{0.15} & \multicolumn{1}{c}{1306} & \multicolumn{1}{c}{-0.08} & \multicolumn{1}{c}{0.06} & \multicolumn{1}{c}{-0.65} & \multicolumn{1}{c}{-0.02} & \multicolumn{1}{c}{1306} &       & \multicolumn{1}{c}{SIOPS} & \multicolumn{1}{c}{2000} \\
          &       &       &       &       &       &       &       &       &       &       &       &       &       &       &       &       &       &       &  \\
    \multicolumn{1}{l}{\textbf{Municipality Public Revenue}} &       &       &       &       &       &       &       &       &       &       &       &       &       &       &       &       &       &       &  \\
    \multicolumn{1}{l}{Total Revenue per capita} & \multicolumn{1}{c}{1225.27} & \multicolumn{1}{c}{2282.13} & \multicolumn{1}{c}{132.39} & \multicolumn{1}{c}{121105.02} & \multicolumn{1}{c}{5288} &       & \multicolumn{1}{c}{1162.33} & \multicolumn{1}{c}{3438.04} & \multicolumn{1}{c}{166.10} & \multicolumn{1}{c}{121105.02} & \multicolumn{1}{c}{1257} & \multicolumn{1}{c}{1225.99} & \multicolumn{1}{c}{710.26} & \multicolumn{1}{c}{282.39} & \multicolumn{1}{c}{8866.51} & \multicolumn{1}{c}{1269} &       & \multicolumn{1}{c}{Finbra} & \multicolumn{1}{c}{2000} \\
          &       &       &       &       &       &       &       &       &       &       &       &       &       &       &       &       &       &       &  \\
    \multicolumn{1}{l}{Revenue by Source - per capita} &       &       &       &       &       &       &       &       &       &       &       &       &       &       &       &       &       &       &  \\
    \multicolumn{1}{l}{Tax Revenue per capita} & \multicolumn{1}{c}{65.63} & \multicolumn{1}{c}{276.91} & \multicolumn{1}{c}{0.00} & \multicolumn{1}{c}{17459.07} & \multicolumn{1}{c}{5288} &       & \multicolumn{1}{c}{49.28} & \multicolumn{1}{c}{106.35} & \multicolumn{1}{c}{0.00} & \multicolumn{1}{c}{1607.99} & \multicolumn{1}{c}{1257} & \multicolumn{1}{c}{70.67} & \multicolumn{1}{c}{156.85} & \multicolumn{1}{c}{0.00} & \multicolumn{1}{c}{3145.52} & \multicolumn{1}{c}{1269} &       & \multicolumn{1}{c}{Finbra} & \multicolumn{1}{c}{2000} \\
    \multicolumn{1}{l}{Transfers Revenue per capita} & \multicolumn{1}{c}{1085.67} & \multicolumn{1}{c}{2055.60} & \multicolumn{1}{c}{131.93} & \multicolumn{1}{c}{118718.04} & \multicolumn{1}{c}{5288} &       & \multicolumn{1}{c}{1060.14} & \multicolumn{1}{c}{3364.99} & \multicolumn{1}{c}{164.99} & \multicolumn{1}{c}{118718.04} & \multicolumn{1}{c}{1257} & \multicolumn{1}{c}{1058.28} & \multicolumn{1}{c}{532.77} & \multicolumn{1}{c}{263.25} & \multicolumn{1}{c}{5210.39} & \multicolumn{1}{c}{1269} &       & \multicolumn{1}{c}{Finbra} & \multicolumn{1}{c}{2000} \\
    \multicolumn{1}{l}{Other Revenues per capita} & \multicolumn{1}{c}{73.97} & \multicolumn{1}{c}{194.92} & \multicolumn{1}{c}{0.00} & \multicolumn{1}{c}{5489.05} & \multicolumn{1}{c}{5288} &       & \multicolumn{1}{c}{52.91} & \multicolumn{1}{c}{94.54} & \multicolumn{1}{c}{0.00} & \multicolumn{1}{c}{1220.07} & \multicolumn{1}{c}{1257} & \multicolumn{1}{c}{97.03} & \multicolumn{1}{c}{293.02} & \multicolumn{1}{c}{0.00} & \multicolumn{1}{c}{5489.05} & \multicolumn{1}{c}{1269} &       & \multicolumn{1}{c}{Finbra} & \multicolumn{1}{c}{2000} \\
    \multicolumn{1}{l}{Revenue by Source - (\% Total Revenue)} &       &       &       &       &       &       &       &       &       &       &       &       &       &       &       &       &       &       &  \\
    \multicolumn{1}{l}{Tax Revenue} & \multicolumn{1}{c}{0.05} & \multicolumn{1}{c}{0.06} & \multicolumn{1}{c}{0.00} & \multicolumn{1}{c}{0.62} & \multicolumn{1}{c}{5288} &       & \multicolumn{1}{c}{0.04} & \multicolumn{1}{c}{0.06} & \multicolumn{1}{c}{0.00} & \multicolumn{1}{c}{0.48} & \multicolumn{1}{c}{1257} & \multicolumn{1}{c}{0.05} & \multicolumn{1}{c}{0.07} & \multicolumn{1}{c}{0.00} & \multicolumn{1}{c}{0.62} & \multicolumn{1}{c}{1269} &       & \multicolumn{1}{c}{Finbra} & \multicolumn{1}{c}{2000} \\
    \multicolumn{1}{l}{Transfers Revenue} & \multicolumn{1}{c}{0.90} & \multicolumn{1}{c}{0.11} & \multicolumn{1}{c}{0.19} & \multicolumn{1}{c}{1.00} & \multicolumn{1}{c}{5288} &       & \multicolumn{1}{c}{0.92} & \multicolumn{1}{c}{0.10} & \multicolumn{1}{c}{0.30} & \multicolumn{1}{c}{1.00} & \multicolumn{1}{c}{1257} & \multicolumn{1}{c}{0.89} & \multicolumn{1}{c}{0.12} & \multicolumn{1}{c}{0.19} & \multicolumn{1}{c}{1.00} & \multicolumn{1}{c}{1269} &       & \multicolumn{1}{c}{Finbra} & \multicolumn{1}{c}{2000} \\
    \multicolumn{1}{l}{Other Revenues} & \multicolumn{1}{c}{0.05} & \multicolumn{1}{c}{0.07} & \multicolumn{1}{c}{0.00} & \multicolumn{1}{c}{0.77} & \multicolumn{1}{c}{5288} &       & \multicolumn{1}{c}{0.04} & \multicolumn{1}{c}{0.06} & \multicolumn{1}{c}{0.00} & \multicolumn{1}{c}{0.70} & \multicolumn{1}{c}{1257} & \multicolumn{1}{c}{0.06} & \multicolumn{1}{c}{0.08} & \multicolumn{1}{c}{0.00} & \multicolumn{1}{c}{0.77} & \multicolumn{1}{c}{1269} &       & \multicolumn{1}{c}{Finbra} & \multicolumn{1}{c}{2000} \\
          &       &       &       &       &       &       &       &       &       &       &       &       &       &       &       &       &       &       &  \\
    \multicolumn{1}{l}{\textbf{Municipality Public Spending}} &       &       &       &       &       &       &       &       &       &       &       &       &       &       &       &       &       &       &  \\
    \multicolumn{1}{l}{Total Spending per capita} & \multicolumn{1}{c}{1284.77} & \multicolumn{1}{c}{2395.06} & \multicolumn{1}{c}{129.74} & \multicolumn{1}{c}{127974.26} & \multicolumn{1}{c}{5304} &       & \multicolumn{1}{c}{1209.09} & \multicolumn{1}{c}{3627.32} & \multicolumn{1}{c}{237.93} & \multicolumn{1}{c}{127974.26} & \multicolumn{1}{c}{1258} & \multicolumn{1}{c}{1298.79} & \multicolumn{1}{c}{727.39} & \multicolumn{1}{c}{282.02} & \multicolumn{1}{c}{8527.37} & \multicolumn{1}{c}{1277} &       & \multicolumn{1}{c}{Finbra} & \multicolumn{1}{c}{2000} \\
          &       &       &       &       &       &       &       &       &       &       &       &       &       &       &       &       &       &       &  \\
    \multicolumn{1}{l}{Spending by Type - per capita} &       &       &       &       &       &       &       &       &       &       &       &       &       &       &       &       &       &       &  \\
    \multicolumn{1}{l}{Human Resources} & \multicolumn{1}{c}{502.32} & \multicolumn{1}{c}{985.60} & \multicolumn{1}{c}{0.00} & \multicolumn{1}{c}{60697.09} & \multicolumn{1}{c}{5304} &       & \multicolumn{1}{c}{456.65} & \multicolumn{1}{c}{957.40} & \multicolumn{1}{c}{0.00} & \multicolumn{1}{c}{33164.37} & \multicolumn{1}{c}{1258} & \multicolumn{1}{c}{512.71} & \multicolumn{1}{c}{310.79} & \multicolumn{1}{c}{16.96} & \multicolumn{1}{c}{3028.27} & \multicolumn{1}{c}{1277} &       & \multicolumn{1}{c}{Finbra} & \multicolumn{1}{c}{2000} \\
    \multicolumn{1}{l}{Investment} & \multicolumn{1}{c}{153.70} & \multicolumn{1}{c}{277.13} & \multicolumn{1}{c}{0.00} & \multicolumn{1}{c}{14815.46} & \multicolumn{1}{c}{5304} &       & \multicolumn{1}{c}{148.48} & \multicolumn{1}{c}{436.17} & \multicolumn{1}{c}{0.00} & \multicolumn{1}{c}{14815.46} & \multicolumn{1}{c}{1258} & \multicolumn{1}{c}{158.51} & \multicolumn{1}{c}{186.73} & \multicolumn{1}{c}{0.00} & \multicolumn{1}{c}{2737.47} & \multicolumn{1}{c}{1277} &       & \multicolumn{1}{c}{Finbra} & \multicolumn{1}{c}{2000} \\
    \multicolumn{1}{l}{Other} & \multicolumn{1}{c}{628.75} & \multicolumn{1}{c}{1280.34} & \multicolumn{1}{c}{0.00} & \multicolumn{1}{c}{79994.44} & \multicolumn{1}{c}{5304} &       & \multicolumn{1}{c}{603.96} & \multicolumn{1}{c}{2263.11} & \multicolumn{1}{c}{0.00} & \multicolumn{1}{c}{79994.44} & \multicolumn{1}{c}{1258} & \multicolumn{1}{c}{627.58} & \multicolumn{1}{c}{354.68} & \multicolumn{1}{c}{10.60} & \multicolumn{1}{c}{3339.68} & \multicolumn{1}{c}{1277} &       & \multicolumn{1}{c}{Finbra} & \multicolumn{1}{c}{2000} \\
    \multicolumn{1}{l}{Spending by Type - (\% Total Spending)} &       &       &       &       &       &       &       &       &       &       &       &       &       &       &       &       &       &       &  \\
    \multicolumn{1}{l}{Human Resources} & \multicolumn{1}{c}{0.40} & \multicolumn{1}{c}{0.10} & \multicolumn{1}{c}{0.00} & \multicolumn{1}{c}{1.00} & \multicolumn{1}{c}{5304} &       & \multicolumn{1}{c}{0.39} & \multicolumn{1}{c}{0.11} & \multicolumn{1}{c}{0.00} & \multicolumn{1}{c}{1.00} & \multicolumn{1}{c}{1258} & \multicolumn{1}{c}{0.40} & \multicolumn{1}{c}{0.10} & \multicolumn{1}{c}{0.02} & \multicolumn{1}{c}{0.92} & \multicolumn{1}{c}{1277} &       & \multicolumn{1}{c}{Finbra} & \multicolumn{1}{c}{2000} \\
    \multicolumn{1}{l}{Investment} & \multicolumn{1}{c}{0.12} & \multicolumn{1}{c}{0.07} & \multicolumn{1}{c}{0.00} & \multicolumn{1}{c}{0.71} & \multicolumn{1}{c}{5304} &       & \multicolumn{1}{c}{0.12} & \multicolumn{1}{c}{0.08} & \multicolumn{1}{c}{0.00} & \multicolumn{1}{c}{0.71} & \multicolumn{1}{c}{1258} & \multicolumn{1}{c}{0.12} & \multicolumn{1}{c}{0.07} & \multicolumn{1}{c}{0.00} & \multicolumn{1}{c}{0.54} & \multicolumn{1}{c}{1277} &       & \multicolumn{1}{c}{Finbra} & \multicolumn{1}{c}{2000} \\
    \multicolumn{1}{l}{Other} & \multicolumn{1}{c}{0.49} & \multicolumn{1}{c}{0.11} & \multicolumn{1}{c}{0.00} & \multicolumn{1}{c}{1.00} & \multicolumn{1}{c}{5304} &       & \multicolumn{1}{c}{0.49} & \multicolumn{1}{c}{0.11} & \multicolumn{1}{c}{0.00} & \multicolumn{1}{c}{0.94} & \multicolumn{1}{c}{1258} & \multicolumn{1}{c}{0.49} & \multicolumn{1}{c}{0.10} & \multicolumn{1}{c}{0.00} & \multicolumn{1}{c}{0.89} & \multicolumn{1}{c}{1277} &       & \multicolumn{1}{c}{Finbra} & \multicolumn{1}{c}{2000} \\
          &       &       &       &       &       &       &       &       &       &       &       &       &       &       &       &       &       &       &  \\
    \multicolumn{1}{l}{Spending by Category - per capita} &       &       &       &       &       &       &       &       &       &       &       &       &       &       &       &       &       &       &  \\
    \multicolumn{1}{l}{Health and Sanitation} & \multicolumn{1}{c}{217.08} & \multicolumn{1}{c}{276.14} & \multicolumn{1}{c}{0.04} & \multicolumn{1}{c}{12559.61} & \multicolumn{1}{c}{5286} &       & \multicolumn{1}{c}{161.53} & \multicolumn{1}{c}{358.67} & \multicolumn{1}{c}{0.99} & \multicolumn{1}{c}{12471.29} & \multicolumn{1}{c}{1252} & \multicolumn{1}{c}{274.57} & \multicolumn{1}{c}{167.33} & \multicolumn{1}{c}{6.96} & \multicolumn{1}{c}{1736.83} & \multicolumn{1}{c}{1275} &       & \multicolumn{1}{c}{Finbra} & \multicolumn{1}{c}{2000} \\
    \multicolumn{1}{l}{Transport} & \multicolumn{1}{c}{91.55} & \multicolumn{1}{c}{138.28} & \multicolumn{1}{c}{0.00} & \multicolumn{1}{c}{5865.79} & \multicolumn{1}{c}{5304} &       & \multicolumn{1}{c}{89.77} & \multicolumn{1}{c}{123.32} & \multicolumn{1}{c}{0.00} & \multicolumn{1}{c}{1146.75} & \multicolumn{1}{c}{1258} & \multicolumn{1}{c}{77.51} & \multicolumn{1}{c}{95.58} & \multicolumn{1}{c}{0.00} & \multicolumn{1}{c}{1030.91} & \multicolumn{1}{c}{1277} &       & \multicolumn{1}{c}{Finbra} & \multicolumn{1}{c}{2000} \\
    \multicolumn{1}{l}{Education and Culture} & \multicolumn{1}{c}{419.95} & \multicolumn{1}{c}{640.29} & \multicolumn{1}{c}{0.00} & \multicolumn{1}{c}{36319.15} & \multicolumn{1}{c}{5304} &       & \multicolumn{1}{c}{399.03} & \multicolumn{1}{c}{1032.67} & \multicolumn{1}{c}{0.00} & \multicolumn{1}{c}{36319.15} & \multicolumn{1}{c}{1258} & \multicolumn{1}{c}{422.82} & \multicolumn{1}{c}{214.62} & \multicolumn{1}{c}{6.97} & \multicolumn{1}{c}{1811.43} & \multicolumn{1}{c}{1277} &       & \multicolumn{1}{c}{Finbra} & \multicolumn{1}{c}{2000} \\
    \multicolumn{1}{l}{Housing and Urban} & \multicolumn{1}{c}{116.05} & \multicolumn{1}{c}{301.25} & \multicolumn{1}{c}{0.00} & \multicolumn{1}{c}{19842.15} & \multicolumn{1}{c}{5304} &       & \multicolumn{1}{c}{118.96} & \multicolumn{1}{c}{565.01} & \multicolumn{1}{c}{0.00} & \multicolumn{1}{c}{19842.15} & \multicolumn{1}{c}{1258} & \multicolumn{1}{c}{120.68} & \multicolumn{1}{c}{115.52} & \multicolumn{1}{c}{0.00} & \multicolumn{1}{c}{1207.96} & \multicolumn{1}{c}{1277} &       & \multicolumn{1}{c}{Finbra} & \multicolumn{1}{c}{2000} \\
    \multicolumn{1}{l}{Social Assistance} & \multicolumn{1}{c}{84.05} & \multicolumn{1}{c}{253.84} & \multicolumn{1}{c}{0.00} & \multicolumn{1}{c}{13814.63} & \multicolumn{1}{c}{5304} &       & \multicolumn{1}{c}{83.29} & \multicolumn{1}{c}{393.78} & \multicolumn{1}{c}{0.00} & \multicolumn{1}{c}{13814.63} & \multicolumn{1}{c}{1258} & \multicolumn{1}{c}{79.22} & \multicolumn{1}{c}{71.73} & \multicolumn{1}{c}{0.00} & \multicolumn{1}{c}{870.44} & \multicolumn{1}{c}{1277} &       & \multicolumn{1}{c}{Finbra} & \multicolumn{1}{c}{2000} \\
    \multicolumn{1}{l}{Other Categories} & \multicolumn{1}{c}{472.88} & \multicolumn{1}{c}{1201.13} & \multicolumn{1}{c}{32.00} & \multicolumn{1}{c}{65369.18} & \multicolumn{1}{c}{5304} &       & \multicolumn{1}{c}{476.25} & \multicolumn{1}{c}{1851.75} & \multicolumn{1}{c}{85.06} & \multicolumn{1}{c}{65369.18} & \multicolumn{1}{c}{1258} & \multicolumn{1}{c}{445.10} & \multicolumn{1}{c}{323.86} & \multicolumn{1}{c}{86.34} & \multicolumn{1}{c}{3698.10} & \multicolumn{1}{c}{1277} &       & \multicolumn{1}{c}{Finbra} & \multicolumn{1}{c}{2000} \\
    \multicolumn{1}{l}{Spending by Category - (\% Total Spending)} &       &       &       &       &       &       &       &       &       &       &       &       &       &       &       &       &       &       &  \\
    \multicolumn{1}{l}{Health and Sanitation} & \multicolumn{1}{c}{0.17} & \multicolumn{1}{c}{0.07} & \multicolumn{1}{c}{0.00} & \multicolumn{1}{c}{0.53} & \multicolumn{1}{c}{5304} &       & \multicolumn{1}{c}{0.14} & \multicolumn{1}{c}{0.06} & \multicolumn{1}{c}{0.00} & \multicolumn{1}{c}{0.46} & \multicolumn{1}{c}{1258} & \multicolumn{1}{c}{0.21} & \multicolumn{1}{c}{0.06} & \multicolumn{1}{c}{0.00} & \multicolumn{1}{c}{0.47} & \multicolumn{1}{c}{1277} &       & \multicolumn{1}{c}{Finbra} & \multicolumn{1}{c}{2000} \\
    \multicolumn{1}{l}{Transport Spending} & \multicolumn{1}{c}{0.07} & \multicolumn{1}{c}{0.06} & \multicolumn{1}{c}{0.00} & \multicolumn{1}{c}{0.37} & \multicolumn{1}{c}{5304} &       & \multicolumn{1}{c}{0.07} & \multicolumn{1}{c}{0.07} & \multicolumn{1}{c}{0.00} & \multicolumn{1}{c}{0.37} & \multicolumn{1}{c}{1258} & \multicolumn{1}{c}{0.06} & \multicolumn{1}{c}{0.05} & \multicolumn{1}{c}{0.00} & \multicolumn{1}{c}{0.32} & \multicolumn{1}{c}{1277} &       & \multicolumn{1}{c}{Finbra} & \multicolumn{1}{c}{2000} \\
    \multicolumn{1}{l}{Education and Culture Spending} & \multicolumn{1}{c}{0.34} & \multicolumn{1}{c}{0.08} & \multicolumn{1}{c}{0.00} & \multicolumn{1}{c}{0.67} & \multicolumn{1}{c}{5304} &       & \multicolumn{1}{c}{0.34} & \multicolumn{1}{c}{0.08} & \multicolumn{1}{c}{0.00} & \multicolumn{1}{c}{0.64} & \multicolumn{1}{c}{1258} & \multicolumn{1}{c}{0.34} & \multicolumn{1}{c}{0.08} & \multicolumn{1}{c}{0.01} & \multicolumn{1}{c}{0.61} & \multicolumn{1}{c}{1277} &       & \multicolumn{1}{c}{Finbra} & \multicolumn{1}{c}{2000} \\
    \multicolumn{1}{l}{Housing and Urban Spending} & \multicolumn{1}{c}{0.09} & \multicolumn{1}{c}{0.06} & \multicolumn{1}{c}{0.00} & \multicolumn{1}{c}{0.63} & \multicolumn{1}{c}{5304} &       & \multicolumn{1}{c}{0.10} & \multicolumn{1}{c}{0.07} & \multicolumn{1}{c}{0.00} & \multicolumn{1}{c}{0.59} & \multicolumn{1}{c}{1258} & \multicolumn{1}{c}{0.09} & \multicolumn{1}{c}{0.06} & \multicolumn{1}{c}{0.00} & \multicolumn{1}{c}{0.33} & \multicolumn{1}{c}{1277} &       & \multicolumn{1}{c}{Finbra} & \multicolumn{1}{c}{2000} \\
    \multicolumn{1}{l}{Social Assistance Spending} & \multicolumn{1}{c}{0.06} & \multicolumn{1}{c}{0.04} & \multicolumn{1}{c}{0.00} & \multicolumn{1}{c}{0.33} & \multicolumn{1}{c}{5304} &       & \multicolumn{1}{c}{0.07} & \multicolumn{1}{c}{0.04} & \multicolumn{1}{c}{0.00} & \multicolumn{1}{c}{0.29} & \multicolumn{1}{c}{1258} & \multicolumn{1}{c}{0.06} & \multicolumn{1}{c}{0.04} & \multicolumn{1}{c}{0.00} & \multicolumn{1}{c}{0.33} & \multicolumn{1}{c}{1277} &       & \multicolumn{1}{c}{Finbra} & \multicolumn{1}{c}{2000} \\
    \multicolumn{1}{l}{Spending in Other Areas} & \multicolumn{1}{c}{0.36} & \multicolumn{1}{c}{0.09} & \multicolumn{1}{c}{0.12} & \multicolumn{1}{c}{1.00} & \multicolumn{1}{c}{5304} &       & \multicolumn{1}{c}{0.38} & \multicolumn{1}{c}{0.10} & \multicolumn{1}{c}{0.15} & \multicolumn{1}{c}{1.00} & \multicolumn{1}{c}{1258} & \multicolumn{1}{c}{0.33} & \multicolumn{1}{c}{0.08} & \multicolumn{1}{c}{0.12} & \multicolumn{1}{c}{0.79} & \multicolumn{1}{c}{1277} &       & \multicolumn{1}{c}{Finbra} & \multicolumn{1}{c}{2000} \\
          &       &       &       &       &       &       &       &       &       &       &       &       &       &       &       &       &       &       &  \\
    \midrule
    \midrule
          &       &       &       &       &       &       &       &       &       &       &       &       &       &       &       &       &       &       &  \\
    \end{tabular}%
    
  \label{table:stats}%

\end{threeparttable}
}
\end{center}
\end{footnotesize}
\end{table}
\end{sidewaystable}
\begin{table}[H]
\begin{footnotesize}
\begin{center}
\scalebox{0.75}{
\begin{threeparttable}[b]


  \centering
  \caption*{Table A.1: Descriptive Statistics (at the baseline year) -- \emph{Cont.}}
  
  \begin{tabular}{rrrrrrrr}
          &       &       &       &       &       &       &  \\
    \midrule
    \midrule
          & \multicolumn{1}{c}{Mean} & \multicolumn{1}{c}{Std. Dev.} & \multicolumn{1}{c}{Min} & \multicolumn{1}{c}{Max} & \multicolumn{1}{c}{Obs.} &       & \multicolumn{1}{c}{Source of Data} \\
\cmidrule{1-6}\cmidrule{8-8}    \multicolumn{1}{l}{\textbf{Primary Care Coverage}} &       &       &       &       &       &       &  \\
    \multicolumn{1}{l}{Extensive Margin} &       &       &       &       &       &       &  \\
    \multicolumn{1}{l}{Population covered (share) by Community Health Agents} & \multicolumn{1}{c}{0.635} & \multicolumn{1}{c}{0.409} & \multicolumn{1}{c}{0} & \multicolumn{1}{c}{1} & \multicolumn{1}{c}{5507} &       & \multicolumn{1}{c}{Datasus/SIAB} \\
    \multicolumn{1}{l}{Population covered (share) by Family Health Agents} & \multicolumn{1}{c}{0.311} & \multicolumn{1}{c}{0.383} & \multicolumn{1}{c}{0} & \multicolumn{1}{c}{1} & \multicolumn{1}{c}{5507} &       & \multicolumn{1}{c}{Datasus/SIAB} \\
          &       &       &       &       &       &       &  \\
    \multicolumn{1}{l}{Intensive Margin} &       &       &       &       &       &       &  \\
    \multicolumn{1}{l}{N. of People Visited by Primary Care Agents (per capita)} & \multicolumn{1}{c}{0.271} & \multicolumn{1}{c}{0.285} & \multicolumn{1}{c}{0} & \multicolumn{1}{c}{2.798} & \multicolumn{1}{c}{5507} &       & \multicolumn{1}{c}{Datasus/SIAB} \\
    \multicolumn{1}{l}{N. of People Visited by Community Health Agents (per capita)} & \multicolumn{1}{c}{0.121} & \multicolumn{1}{c}{0.18} & \multicolumn{1}{c}{0} & \multicolumn{1}{c}{1.518} & \multicolumn{1}{c}{5507} &       & \multicolumn{1}{c}{Datasus/SIAB} \\
    \multicolumn{1}{l}{N. of People Visited by Family Health Agents (per capita)} & \multicolumn{1}{c}{0.15} & \multicolumn{1}{c}{0.252} & \multicolumn{1}{c}{0} & \multicolumn{1}{c}{1.834} & \multicolumn{1}{c}{5507} &       & \multicolumn{1}{c}{Datasus/SIAB} \\
    \multicolumn{1}{l}{N. of Household Visits \& Appointments (per capita)} & \multicolumn{1}{c}{1.876} & \multicolumn{1}{c}{2.541} & \multicolumn{1}{c}{0} & \multicolumn{1}{c}{88.85} & \multicolumn{1}{c}{5507} &       & \multicolumn{1}{c}{Datasus/SIAB} \\
    \multicolumn{1}{l}{N. of Household Visits \& Appointments by Community Health Agents (per capita)} & \multicolumn{1}{c}{1.072} & \multicolumn{1}{c}{2.156} & \multicolumn{1}{c}{0} & \multicolumn{1}{c}{85.989} & \multicolumn{1}{c}{5507} &       & \multicolumn{1}{c}{Datasus/SIAB} \\
    \multicolumn{1}{l}{N. of Household Visits \& Appointments by Family Health Agents (per capita)} & \multicolumn{1}{c}{0.8} & \multicolumn{1}{c}{1.505} & \multicolumn{1}{c}{0} & \multicolumn{1}{c}{43.389} & \multicolumn{1}{c}{5507} &       & \multicolumn{1}{c}{Datasus/SIAB} \\
          &       &       &       &       &       &       &  \\
    \multicolumn{1}{l}{\textbf{Health Human Resources}} &       &       &       &       &       &       &  \\
    \multicolumn{1}{l}{N. of Health Professionals (per capita*1000)} & \multicolumn{1}{c}{5.104} & \multicolumn{1}{c}{4.825} & \multicolumn{1}{c}{0} & \multicolumn{1}{c}{187.904} & \multicolumn{1}{c}{5507} &       & \multicolumn{1}{c}{IBGE/AMS} \\
    \multicolumn{1}{l}{N. of Doctors (per capita*1000)} & \multicolumn{1}{c}{1.529} & \multicolumn{1}{c}{2.385} & \multicolumn{1}{c}{0} & \multicolumn{1}{c}{95.132} & \multicolumn{1}{c}{5507} &       & \multicolumn{1}{c}{IBGE/AMS} \\
    \multicolumn{1}{l}{N. of Nurses (per capita*1000)} & \multicolumn{1}{c}{1.159} & \multicolumn{1}{c}{1.636} & \multicolumn{1}{c}{0} & \multicolumn{1}{c}{95.097} & \multicolumn{1}{c}{5507} &       & \multicolumn{1}{c}{IBGE/AMS} \\
    \multicolumn{1}{l}{N. of Nursing Assistants (per capita*1000)} & \multicolumn{1}{c}{1.26} & \multicolumn{1}{c}{1.456} & \multicolumn{1}{c}{0} & \multicolumn{1}{c}{22.009} & \multicolumn{1}{c}{5507} &       & \multicolumn{1}{c}{IBGE/AMS} \\
    \multicolumn{1}{l}{N. of Administrative Professionals (per capita*1000)} & \multicolumn{1}{c}{1.155} & \multicolumn{1}{c}{1.251} & \multicolumn{1}{c}{0} & \multicolumn{1}{c}{36.599} & \multicolumn{1}{c}{5507} &       & \multicolumn{1}{c}{IBGE/AMS} \\
          &       &       &       &       &       &       &  \\
    \multicolumn{1}{l}{\textbf{Health Infrastructure}} &       &       &       &       &       &       &  \\
    \multicolumn{1}{l}{N. of Municipal Hospitals (per capita*1000)} & \multicolumn{1}{c}{0.06} & \multicolumn{1}{c}{0.138} & \multicolumn{1}{c}{0} & \multicolumn{1}{c}{1.357} & \multicolumn{1}{c}{5507} &       & \multicolumn{1}{c}{IBGE/AMS} \\
    \multicolumn{1}{l}{N. of Federal and State Hospitals (per capita*1000)} & \multicolumn{1}{c}{0.015} & \multicolumn{1}{c}{0.084} & \multicolumn{1}{c}{0} & \multicolumn{1}{c}{1.892} & \multicolumn{1}{c}{5507} &       & \multicolumn{1}{c}{IBGE/AMS} \\
    \multicolumn{1}{l}{N. of Private Hospitals (per capita*1000)} & \multicolumn{1}{c}{0.03} & \multicolumn{1}{c}{0.058} & \multicolumn{1}{c}{0} & \multicolumn{1}{c}{0.609} & \multicolumn{1}{c}{5507} &       & \multicolumn{1}{c}{IBGE/AMS} \\
    \multicolumn{1}{l}{N. of Health Facilities (per capita*1000) with Ambulatory Service} & \multicolumn{1}{c}{0.517} & \multicolumn{1}{c}{0.355} & \multicolumn{1}{c}{0} & \multicolumn{1}{c}{3.628} & \multicolumn{1}{c}{5493} &       & \multicolumn{1}{c}{Datasus/SIA} \\
          &       &       &       &       &       &       &  \\
    \multicolumn{1}{l}{\textbf{Primary Care Related Infrastructure and Human Resources}} &       &       &       &       &       &       &  \\
    \multicolumn{1}{l}{Number of Health Facilities (per capita * 1000) with} &       &       &       &       &       &       &  \\
    \multicolumn{1}{l}{Ambulatory Service and ACS Teams} & \multicolumn{1}{c}{0.14} & \multicolumn{1}{c}{0.197} & \multicolumn{1}{c}{0} & \multicolumn{1}{c}{2.41} & \multicolumn{1}{c}{5493} &       & \multicolumn{1}{c}{Datasus/SIA} \\
    \multicolumn{1}{l}{Ambulatory Service and Community Doctors} & \multicolumn{1}{c}{0.082} & \multicolumn{1}{c}{0.154} & \multicolumn{1}{c}{0} & \multicolumn{1}{c}{1.957} & \multicolumn{1}{c}{5493} &       & \multicolumn{1}{c}{Datasus/SIA} \\
    \multicolumn{1}{l}{Ambulatory Service and ACS Nurses} & \multicolumn{1}{c}{0.072} & \multicolumn{1}{c}{0.156} & \multicolumn{1}{c}{0} & \multicolumn{1}{c}{2.41} & \multicolumn{1}{c}{5493} &       & \multicolumn{1}{c}{Datasus/SIA} \\
    \multicolumn{1}{l}{Ambulatory Service and PSF Teams} & \multicolumn{1}{c}{0.083} & \multicolumn{1}{c}{0.159} & \multicolumn{1}{c}{0} & \multicolumn{1}{c}{2.41} & \multicolumn{1}{c}{5493} &       & \multicolumn{1}{c}{Datasus/SIA} \\
    \multicolumn{1}{l}{Ambulatory Service and PSF Doctors} & \multicolumn{1}{c}{0.077} & \multicolumn{1}{c}{0.149} & \multicolumn{1}{c}{0} & \multicolumn{1}{c}{1.957} & \multicolumn{1}{c}{5493} &       & \multicolumn{1}{c}{Datasus/SIA} \\
    \multicolumn{1}{l}{Ambulatory Service and PSF Nurses} & \multicolumn{1}{c}{0.075} & \multicolumn{1}{c}{0.149} & \multicolumn{1}{c}{0} & \multicolumn{1}{c}{2.41} & \multicolumn{1}{c}{5493} &       & \multicolumn{1}{c}{Datasus/SIA} \\
    \multicolumn{1}{l}{Ambulatory Service and PSF Nursing Assistants} & \multicolumn{1}{c}{0.05} & \multicolumn{1}{c}{0.123} & \multicolumn{1}{c}{0} & \multicolumn{1}{c}{1.957} & \multicolumn{1}{c}{5493} &       & \multicolumn{1}{c}{Datasus/SIA} \\
          &       &       &       &       &       &       &  \\
    \multicolumn{1}{l}{\textbf{Access to Health Services}} &       &       &       &       &       &       &  \\
    \multicolumn{1}{l}{Prenatal Ignored} & \multicolumn{1}{c}{0.044} & \multicolumn{1}{c}{0.094} & \multicolumn{1}{c}{0} & \multicolumn{1}{c}{1} & \multicolumn{1}{c}{5460} &       &  \\
    \multicolumn{1}{l}{Prenatal Visits None} & \multicolumn{1}{c}{0.053} & \multicolumn{1}{c}{0.077} & \multicolumn{1}{c}{0} & \multicolumn{1}{c}{0.921} & \multicolumn{1}{c}{5437} &       & \multicolumn{1}{c}{Datasus/SINASC} \\
    \multicolumn{1}{l}{Prenatal Visits 1-6} & \multicolumn{1}{c}{0.53} & \multicolumn{1}{c}{0.216} & \multicolumn{1}{c}{0} & \multicolumn{1}{c}{1} & \multicolumn{1}{c}{5507} &       & \multicolumn{1}{c}{Datasus/SINASC} \\
    \multicolumn{1}{l}{Prenatal Visits 7+} & \multicolumn{1}{c}{0.375} & \multicolumn{1}{c}{0.235} & \multicolumn{1}{c}{0} & \multicolumn{1}{c}{1} & \multicolumn{1}{c}{5507} &       & \multicolumn{1}{c}{Datasus/SINASC} \\
          &       &       &       &       &       &       &  \\
    \multicolumn{1}{l}{\textbf{Ambulatorial Production}} &       &       &       &       &       &       &  \\
    \multicolumn{1}{l}{N. Outpatient Procedures (per capita)} & \multicolumn{1}{c}{8.8} & \multicolumn{1}{c}{4.55} & \multicolumn{1}{c}{0} & \multicolumn{1}{c}{48.258} & \multicolumn{1}{c}{5507} &       & \multicolumn{1}{c}{Datasus/SIA} \\
    \multicolumn{1}{l}{N. Primary Care Outpatient Procedures (per capita)} & \multicolumn{1}{c}{7.415} & \multicolumn{1}{c}{3.974} & \multicolumn{1}{c}{0} & \multicolumn{1}{c}{39.367} & \multicolumn{1}{c}{5507} &       & \multicolumn{1}{c}{Datasus/SIA} \\
    \multicolumn{1}{l}{N. Low \& Mid Complexity Outpatient Procedures (per capita)} & \multicolumn{1}{c}{9.467} & \multicolumn{1}{c}{5.801} & \multicolumn{1}{c}{0} & \multicolumn{1}{c}{171.126} & \multicolumn{1}{c}{5493} &       & \multicolumn{1}{c}{Datasus/SIA} \\
    \multicolumn{1}{l}{N. High Complexity Outpatient Procedures (per capita)} & \multicolumn{1}{c}{0.005} & \multicolumn{1}{c}{0.052} & \multicolumn{1}{c}{0} & \multicolumn{1}{c}{2.58} & \multicolumn{1}{c}{5493} &       & \multicolumn{1}{c}{Datasus/SIA} \\
          &       &       &       &       &       &       &  \\
    \bottomrule
    \bottomrule
    \end{tabular}%
    



\end{threeparttable}
}
\end{center}
\end{footnotesize}
\end{table}
\begin{table}[H]
\begin{footnotesize}
\begin{center}
\scalebox{0.8}{
\begin{threeparttable}[b]


  \centering
  \caption*{Table A.1: Descriptive Statistics (at the baseline year) -- \emph{Cont.}}
  
  \begin{tabular}{rrrrrrrr}
          &       &       &       &       &       &       &  \\
    \midrule
    \midrule
          & \multicolumn{1}{c}{Mean} & \multicolumn{1}{c}{Std. Dev.} & \multicolumn{1}{c}{Min} & \multicolumn{1}{c}{Max} & \multicolumn{1}{c}{Obs.} &       & \multicolumn{1}{c}{Source of Data} \\
\cmidrule{1-6}\cmidrule{8-8}    \multicolumn{1}{l}{\textbf{Infant Mortality Rate}} &       &       &       &       &       &       &  \\
    \multicolumn{1}{l}{Total} & \multicolumn{1}{c}{23.07} & \multicolumn{1}{c}{26.16} & \multicolumn{1}{c}{0.00} & \multicolumn{1}{c}{1000.00} & \multicolumn{1}{c}{5507} &       & \multicolumn{1}{c}{Datasus/SIM} \\
    \multicolumn{1}{l}{APC} & \multicolumn{1}{c}{2.10} & \multicolumn{1}{c}{7.10} & \multicolumn{1}{c}{0.00} & \multicolumn{1}{c}{333.33} & \multicolumn{1}{c}{5507} &       & \multicolumn{1}{c}{Datasus/SIM} \\
    \multicolumn{1}{l}{non-APC} & \multicolumn{1}{c}{20.97} & \multicolumn{1}{c}{22.29} & \multicolumn{1}{c}{0.00} & \multicolumn{1}{c}{666.67} & \multicolumn{1}{c}{5507} &       & \multicolumn{1}{c}{Datasus/SIM} \\
    \multicolumn{1}{l}{Fetal} & \multicolumn{1}{c}{0.00} & \multicolumn{1}{c}{0.08} & \multicolumn{1}{c}{0.00} & \multicolumn{1}{c}{3.57} & \multicolumn{1}{c}{5507} &       & \multicolumn{1}{c}{Datasus/SIM} \\
    \multicolumn{1}{l}{Within 24h} & \multicolumn{1}{c}{5.55} & \multicolumn{1}{c}{10.15} & \multicolumn{1}{c}{0.00} & \multicolumn{1}{c}{333.33} & \multicolumn{1}{c}{5507} &       & \multicolumn{1}{c}{Datasus/SIM} \\
    \multicolumn{1}{l}{1 to 27 days} & \multicolumn{1}{c}{13.73} & \multicolumn{1}{c}{15.89} & \multicolumn{1}{c}{0.00} & \multicolumn{1}{c}{333.33} & \multicolumn{1}{c}{5507} &       & \multicolumn{1}{c}{Datasus/SIM} \\
    \multicolumn{1}{l}{27 days to 1 year} & \multicolumn{1}{c}{9.34} & \multicolumn{1}{c}{16.34} & \multicolumn{1}{c}{0.00} & \multicolumn{1}{c}{666.67} & \multicolumn{1}{c}{5507} &       & \multicolumn{1}{c}{Datasus/SIM} \\
    \multicolumn{1}{l}{Infectious} & \multicolumn{1}{c}{2.00} & \multicolumn{1}{c}{7.03} & \multicolumn{1}{c}{0.00} & \multicolumn{1}{c}{333.33} & \multicolumn{1}{c}{5507} &       & \multicolumn{1}{c}{Datasus/SIM} \\
    \multicolumn{1}{l}{Respiratory} & \multicolumn{1}{c}{1.52} & \multicolumn{1}{c}{4.45} & \multicolumn{1}{c}{0.00} & \multicolumn{1}{c}{142.86} & \multicolumn{1}{c}{5507} &       & \multicolumn{1}{c}{Datasus/SIM} \\
    \multicolumn{1}{l}{Perinatal} & \multicolumn{1}{c}{11.04} & \multicolumn{1}{c}{16.32} & \multicolumn{1}{c}{0.00} & \multicolumn{1}{c}{666.67} & \multicolumn{1}{c}{5507} &       & \multicolumn{1}{c}{Datasus/SIM} \\
    \multicolumn{1}{l}{Congenital} & \multicolumn{1}{c}{2.13} & \multicolumn{1}{c}{5.01} & \multicolumn{1}{c}{0.00} & \multicolumn{1}{c}{93.02} & \multicolumn{1}{c}{5507} &       & \multicolumn{1}{c}{Datasus/SIM} \\
    \multicolumn{1}{l}{External} & \multicolumn{1}{c}{0.37} & \multicolumn{1}{c}{1.91} & \multicolumn{1}{c}{0.00} & \multicolumn{1}{c}{43.48} & \multicolumn{1}{c}{5507} &       & \multicolumn{1}{c}{Datasus/SIM} \\
    \multicolumn{1}{l}{Nutritional} & \multicolumn{1}{c}{0.60} & \multicolumn{1}{c}{3.22} & \multicolumn{1}{c}{0.00} & \multicolumn{1}{c}{166.67} & \multicolumn{1}{c}{5507} &       & \multicolumn{1}{c}{Datasus/SIM} \\
    \multicolumn{1}{l}{Other} & \multicolumn{1}{c}{0.87} & \multicolumn{1}{c}{3.60} & \multicolumn{1}{c}{0.00} & \multicolumn{1}{c}{142.86} & \multicolumn{1}{c}{5507} &       & \multicolumn{1}{c}{Datasus/SIM} \\
    \multicolumn{1}{l}{Ill-Defined} & \multicolumn{1}{c}{4.55} & \multicolumn{1}{c}{10.68} & \multicolumn{1}{c}{0.00} & \multicolumn{1}{c}{142.86} & \multicolumn{1}{c}{5507} &       & \multicolumn{1}{c}{Datasus/SIM} \\
          &       &       &       &       &       &       &  \\
    \multicolumn{1}{l}{\textbf{Fertility}} &       &       &       &       &       &       &  \\
    \multicolumn{1}{l}{Rates of Birth per Woman (10-49y)} & \multicolumn{1}{c}{0.06} & \multicolumn{1}{c}{0.02} & \multicolumn{1}{c}{0.00} & \multicolumn{1}{c}{0.17} & \multicolumn{1}{c}{5507} &       & \multicolumn{1}{c}{Datasus/SINASC} \\
          &       &       &       &       &       &       &  \\
    \multicolumn{1}{l}{\textbf{Birth Oucomes}} &       &       &       &       &       &       &  \\
    \multicolumn{1}{l}{Apgar 1} & \multicolumn{1}{c}{8.18} & \multicolumn{1}{c}{0.90} & \multicolumn{1}{c}{1.00} & \multicolumn{1}{c}{9.00} & \multicolumn{1}{c}{5428} &       & \multicolumn{1}{c}{Datasus/SINASC} \\
    \multicolumn{1}{l}{Apgar 5} & \multicolumn{1}{c}{8.66} & \multicolumn{1}{c}{0.89} & \multicolumn{1}{c}{1.00} & \multicolumn{1}{c}{9.00} & \multicolumn{1}{c}{5082} &       & \multicolumn{1}{c}{Datasus/SINASC} \\
    \multicolumn{1}{l}{Low Birth Weight (<2.5k)} & \multicolumn{1}{c}{0.07} & \multicolumn{1}{c}{0.03} & \multicolumn{1}{c}{0.00} & \multicolumn{1}{c}{0.50} & \multicolumn{1}{c}{5507} &       & \multicolumn{1}{c}{Datasus/SINASC} \\
    \multicolumn{1}{l}{Premature Birth} & \multicolumn{1}{c}{0.09} & \multicolumn{1}{c}{0.11} & \multicolumn{1}{c}{0.00} & \multicolumn{1}{c}{1.00} & \multicolumn{1}{c}{5507} &       & \multicolumn{1}{c}{Datasus/SINASC} \\
    \multicolumn{1}{l}{Sex Ratio at Birth} & \multicolumn{1}{c}{1.07} & \multicolumn{1}{c}{0.25} & \multicolumn{1}{c}{0.15} & \multicolumn{1}{c}{5.00} & \multicolumn{1}{c}{5505} &       & \multicolumn{1}{c}{Datasus/SINASC} \\
          &       &       &       &       &       &       &  \\
    \midrule
    \midrule
          &       &       &       &       &       &       &  \\
    \end{tabular}%
    



\end{threeparttable}
}
\end{center}
\end{footnotesize}
\end{table}

\begin{sidewaystable}
\begin{table}[H]
\begin{footnotesize}
\begin{center}
\scalebox{0.6}{
\begin{threeparttable}[b]


  \centering
  \caption*{Table 1: Descriptive Statistics (at the baseline year) -- \emph{Cont.}}
  
  \begin{tabular}{rrrrrrrrrrrrrrrrrrrr}
          &       &       &       &       &       &       &       &       &       &       &       &       &       &       &       &       &       &       &  \\
    \midrule
    \midrule
          & \multicolumn{5}{c}{Full Sample}       &       & \multicolumn{5}{c}{Bottom Quartile of OR Spent in Public Health} & \multicolumn{5}{c}{Top Quartile of OR Spent in Public Health} &       &       &  \\
\cmidrule{2-6}\cmidrule{8-17}          & \multicolumn{1}{c}{Mean} & \multicolumn{1}{c}{Std. Dev.} & \multicolumn{1}{c}{Min} & \multicolumn{1}{c}{Max} & \multicolumn{1}{c}{Obs.} &       & \multicolumn{1}{c}{Mean} & \multicolumn{1}{c}{Std. Dev.} & \multicolumn{1}{c}{Min} & \multicolumn{1}{c}{Max} & \multicolumn{1}{c}{Obs.} & \multicolumn{1}{c}{Mean} & \multicolumn{1}{c}{Std. Dev.} & \multicolumn{1}{c}{Min} & \multicolumn{1}{c}{Max} & \multicolumn{1}{c}{Obs.} &       & \multicolumn{1}{c}{Source of Data} & \multicolumn{1}{c}{Baseline Year} \\
\cmidrule{1-6}\cmidrule{8-17}\cmidrule{19-20}    \multicolumn{1}{l}{\textbf{Infant Mortality Rate}} &       &       &       &       &       &       &       &       &       &       &       &       &       &       &       &       &       &       &  \\
    \multicolumn{1}{l}{Total} & \multicolumn{1}{c}{23.07} & \multicolumn{1}{c}{26.16} & \multicolumn{1}{c}{0.00} & \multicolumn{1}{c}{1000.00} & \multicolumn{1}{c}{5507} &       & \multicolumn{1}{c}{23.46} & \multicolumn{1}{c}{19.05} & \multicolumn{1}{c}{0.00} & \multicolumn{1}{c}{166.67} & \multicolumn{1}{c}{1306} & \multicolumn{1}{c}{22.84} & \multicolumn{1}{c}{32.80} & \multicolumn{1}{c}{0.00} & \multicolumn{1}{c}{1000.00} & \multicolumn{1}{c}{1306} &       & \multicolumn{1}{c}{Datasus/SIM} & \multicolumn{1}{c}{2000} \\
    \multicolumn{1}{l}{APC} & \multicolumn{1}{c}{2.10} & \multicolumn{1}{c}{7.10} & \multicolumn{1}{c}{0.00} & \multicolumn{1}{c}{333.33} & \multicolumn{1}{c}{5507} &       & \multicolumn{1}{c}{2.06} & \multicolumn{1}{c}{5.50} & \multicolumn{1}{c}{0.00} & \multicolumn{1}{c}{142.86} & \multicolumn{1}{c}{1306} & \multicolumn{1}{c}{2.18} & \multicolumn{1}{c}{10.30} & \multicolumn{1}{c}{0.00} & \multicolumn{1}{c}{333.33} & \multicolumn{1}{c}{1306} &       & \multicolumn{1}{c}{Datasus/SIM} & \multicolumn{1}{c}{2000} \\
    \multicolumn{1}{l}{non-APC} & \multicolumn{1}{c}{20.97} & \multicolumn{1}{c}{22.29} & \multicolumn{1}{c}{0.00} & \multicolumn{1}{c}{666.67} & \multicolumn{1}{c}{5507} &       & \multicolumn{1}{c}{21.40} & \multicolumn{1}{c}{17.83} & \multicolumn{1}{c}{0.00} & \multicolumn{1}{c}{166.67} & \multicolumn{1}{c}{1306} & \multicolumn{1}{c}{20.66} & \multicolumn{1}{c}{24.73} & \multicolumn{1}{c}{0.00} & \multicolumn{1}{c}{666.67} & \multicolumn{1}{c}{1306} &       & \multicolumn{1}{c}{Datasus/SIM} & \multicolumn{1}{c}{2000} \\
    \multicolumn{1}{l}{Fetal} & \multicolumn{1}{c}{0.00} & \multicolumn{1}{c}{0.08} & \multicolumn{1}{c}{0.00} & \multicolumn{1}{c}{3.57} & \multicolumn{1}{c}{5507} &       & \multicolumn{1}{c}{0.01} & \multicolumn{1}{c}{0.09} & \multicolumn{1}{c}{0.00} & \multicolumn{1}{c}{2.21} & \multicolumn{1}{c}{1306} & \multicolumn{1}{c}{0.00} & \multicolumn{1}{c}{0.08} & \multicolumn{1}{c}{0.00} & \multicolumn{1}{c}{2.87} & \multicolumn{1}{c}{1306} &       & \multicolumn{1}{c}{Datasus/SIM} & \multicolumn{1}{c}{2000} \\
    \multicolumn{1}{l}{Within 24h} & \multicolumn{1}{c}{5.55} & \multicolumn{1}{c}{10.15} & \multicolumn{1}{c}{0.00} & \multicolumn{1}{c}{333.33} & \multicolumn{1}{c}{5507} &       & \multicolumn{1}{c}{5.58} & \multicolumn{1}{c}{8.27} & \multicolumn{1}{c}{0.00} & \multicolumn{1}{c}{80.00} & \multicolumn{1}{c}{1306} & \multicolumn{1}{c}{5.53} & \multicolumn{1}{c}{12.02} & \multicolumn{1}{c}{0.00} & \multicolumn{1}{c}{333.33} & \multicolumn{1}{c}{1306} &       & \multicolumn{1}{c}{Datasus/SIM} & \multicolumn{1}{c}{2000} \\
    \multicolumn{1}{l}{1 to 27 days} & \multicolumn{1}{c}{13.73} & \multicolumn{1}{c}{15.89} & \multicolumn{1}{c}{0.00} & \multicolumn{1}{c}{333.33} & \multicolumn{1}{c}{5507} &       & \multicolumn{1}{c}{13.61} & \multicolumn{1}{c}{13.28} & \multicolumn{1}{c}{0.00} & \multicolumn{1}{c}{166.67} & \multicolumn{1}{c}{1306} & \multicolumn{1}{c}{13.46} & \multicolumn{1}{c}{15.55} & \multicolumn{1}{c}{0.00} & \multicolumn{1}{c}{333.33} & \multicolumn{1}{c}{1306} &       & \multicolumn{1}{c}{Datasus/SIM} & \multicolumn{1}{c}{2000} \\
    \multicolumn{1}{l}{27 days to 1 year} & \multicolumn{1}{c}{9.34} & \multicolumn{1}{c}{16.34} & \multicolumn{1}{c}{0.00} & \multicolumn{1}{c}{666.67} & \multicolumn{1}{c}{5507} &       & \multicolumn{1}{c}{9.85} & \multicolumn{1}{c}{12.67} & \multicolumn{1}{c}{0.00} & \multicolumn{1}{c}{142.86} & \multicolumn{1}{c}{1306} & \multicolumn{1}{c}{9.38} & \multicolumn{1}{c}{21.80} & \multicolumn{1}{c}{0.00} & \multicolumn{1}{c}{666.67} & \multicolumn{1}{c}{1306} &       & \multicolumn{1}{c}{Datasus/SIM} & \multicolumn{1}{c}{2000} \\
    \multicolumn{1}{l}{Infectious} & \multicolumn{1}{c}{2.00} & \multicolumn{1}{c}{7.03} & \multicolumn{1}{c}{0.00} & \multicolumn{1}{c}{333.33} & \multicolumn{1}{c}{5507} &       & \multicolumn{1}{c}{1.96} & \multicolumn{1}{c}{3.95} & \multicolumn{1}{c}{0.00} & \multicolumn{1}{c}{31.25} & \multicolumn{1}{c}{1306} & \multicolumn{1}{c}{2.17} & \multicolumn{1}{c}{10.56} & \multicolumn{1}{c}{0.00} & \multicolumn{1}{c}{333.33} & \multicolumn{1}{c}{1306} &       & \multicolumn{1}{c}{Datasus/SIM} & \multicolumn{1}{c}{2000} \\
    \multicolumn{1}{l}{Respiratory} & \multicolumn{1}{c}{1.52} & \multicolumn{1}{c}{4.45} & \multicolumn{1}{c}{0.00} & \multicolumn{1}{c}{142.86} & \multicolumn{1}{c}{5507} &       & \multicolumn{1}{c}{1.67} & \multicolumn{1}{c}{5.40} & \multicolumn{1}{c}{0.00} & \multicolumn{1}{c}{142.86} & \multicolumn{1}{c}{1306} & \multicolumn{1}{c}{1.58} & \multicolumn{1}{c}{4.20} & \multicolumn{1}{c}{0.00} & \multicolumn{1}{c}{52.63} & \multicolumn{1}{c}{1306} &       & \multicolumn{1}{c}{Datasus/SIM} & \multicolumn{1}{c}{2000} \\
    \multicolumn{1}{l}{Perinatal} & \multicolumn{1}{c}{11.04} & \multicolumn{1}{c}{16.32} & \multicolumn{1}{c}{0.00} & \multicolumn{1}{c}{666.67} & \multicolumn{1}{c}{5507} &       & \multicolumn{1}{c}{10.76} & \multicolumn{1}{c}{11.92} & \multicolumn{1}{c}{0.00} & \multicolumn{1}{c}{166.67} & \multicolumn{1}{c}{1306} & \multicolumn{1}{c}{10.99} & \multicolumn{1}{c}{21.33} & \multicolumn{1}{c}{0.00} & \multicolumn{1}{c}{666.67} & \multicolumn{1}{c}{1306} &       & \multicolumn{1}{c}{Datasus/SIM} & \multicolumn{1}{c}{2000} \\
    \multicolumn{1}{l}{Congenital} & \multicolumn{1}{c}{2.13} & \multicolumn{1}{c}{5.01} & \multicolumn{1}{c}{0.00} & \multicolumn{1}{c}{93.02} & \multicolumn{1}{c}{5507} &       & \multicolumn{1}{c}{2.19} & \multicolumn{1}{c}{5.11} & \multicolumn{1}{c}{0.00} & \multicolumn{1}{c}{55.56} & \multicolumn{1}{c}{1306} & \multicolumn{1}{c}{1.95} & \multicolumn{1}{c}{4.24} & \multicolumn{1}{c}{0.00} & \multicolumn{1}{c}{52.63} & \multicolumn{1}{c}{1306} &       & \multicolumn{1}{c}{Datasus/SIM} & \multicolumn{1}{c}{2000} \\
    \multicolumn{1}{l}{External} & \multicolumn{1}{c}{0.37} & \multicolumn{1}{c}{1.91} & \multicolumn{1}{c}{0.00} & \multicolumn{1}{c}{43.48} & \multicolumn{1}{c}{5507} &       & \multicolumn{1}{c}{0.34} & \multicolumn{1}{c}{2.03} & \multicolumn{1}{c}{0.00} & \multicolumn{1}{c}{41.67} & \multicolumn{1}{c}{1306} & \multicolumn{1}{c}{0.36} & \multicolumn{1}{c}{1.56} & \multicolumn{1}{c}{0.00} & \multicolumn{1}{c}{19.61} & \multicolumn{1}{c}{1306} &       & \multicolumn{1}{c}{Datasus/SIM} & \multicolumn{1}{c}{2000} \\
    \multicolumn{1}{l}{Nutritional} & \multicolumn{1}{c}{0.60} & \multicolumn{1}{c}{3.22} & \multicolumn{1}{c}{0.00} & \multicolumn{1}{c}{166.67} & \multicolumn{1}{c}{5507} &       & \multicolumn{1}{c}{0.56} & \multicolumn{1}{c}{1.92} & \multicolumn{1}{c}{0.00} & \multicolumn{1}{c}{23.26} & \multicolumn{1}{c}{1306} & \multicolumn{1}{c}{0.55} & \multicolumn{1}{c}{2.13} & \multicolumn{1}{c}{0.00} & \multicolumn{1}{c}{32.26} & \multicolumn{1}{c}{1306} &       & \multicolumn{1}{c}{Datasus/SIM} & \multicolumn{1}{c}{2000} \\
    \multicolumn{1}{l}{Other} & \multicolumn{1}{c}{0.87} & \multicolumn{1}{c}{3.60} & \multicolumn{1}{c}{0.00} & \multicolumn{1}{c}{142.86} & \multicolumn{1}{c}{5507} &       & \multicolumn{1}{c}{0.79} & \multicolumn{1}{c}{3.49} & \multicolumn{1}{c}{0.00} & \multicolumn{1}{c}{83.33} & \multicolumn{1}{c}{1306} & \multicolumn{1}{c}{0.89} & \multicolumn{1}{c}{4.75} & \multicolumn{1}{c}{0.00} & \multicolumn{1}{c}{142.86} & \multicolumn{1}{c}{1306} &       & \multicolumn{1}{c}{Datasus/SIM} & \multicolumn{1}{c}{2000} \\
    \multicolumn{1}{l}{Ill-Defined} & \multicolumn{1}{c}{4.55} & \multicolumn{1}{c}{10.68} & \multicolumn{1}{c}{0.00} & \multicolumn{1}{c}{142.86} & \multicolumn{1}{c}{5507} &       & \multicolumn{1}{c}{5.20} & \multicolumn{1}{c}{10.71} & \multicolumn{1}{c}{0.00} & \multicolumn{1}{c}{102.94} & \multicolumn{1}{c}{1306} & \multicolumn{1}{c}{4.35} & \multicolumn{1}{c}{9.19} & \multicolumn{1}{c}{0.00} & \multicolumn{1}{c}{80.65} & \multicolumn{1}{c}{1306} &       & \multicolumn{1}{c}{Datasus/SIM} & \multicolumn{1}{c}{2000} \\
          &       &       &       &       &       &       &       &       &       &       &       &       &       &       &       &       &       &       &  \\
    \multicolumn{1}{l}{\textbf{Adult Mortality Rate}} &       &       &       &       &       &       &       &       &       &       &       &       &       &       &       &       &       &       &  \\
    \multicolumn{1}{l}{Total} & \multicolumn{1}{c}{3.31} & \multicolumn{1}{c}{1.52} & \multicolumn{1}{c}{0.00} & \multicolumn{1}{c}{12.53} & \multicolumn{1}{c}{5507} &       & \multicolumn{1}{c}{3.19} & \multicolumn{1}{c}{1.43} & \multicolumn{1}{c}{0.00} & \multicolumn{1}{c}{9.10} & \multicolumn{1}{c}{1306} & \multicolumn{1}{c}{3.43} & \multicolumn{1}{c}{1.55} & \multicolumn{1}{c}{0.00} & \multicolumn{1}{c}{12.53} & \multicolumn{1}{c}{1306} &       & \multicolumn{1}{c}{Datasus/SIM} & \multicolumn{1}{c}{2000} \\
    \multicolumn{1}{l}{Amenable to Primary Care} & \multicolumn{1}{c}{0.53} & \multicolumn{1}{c}{0.48} & \multicolumn{1}{c}{0.00} & \multicolumn{1}{c}{4.62} & \multicolumn{1}{c}{5507} &       & \multicolumn{1}{c}{0.50} & \multicolumn{1}{c}{0.48} & \multicolumn{1}{c}{0.00} & \multicolumn{1}{c}{4.62} & \multicolumn{1}{c}{1306} & \multicolumn{1}{c}{0.56} & \multicolumn{1}{c}{0.48} & \multicolumn{1}{c}{0.00} & \multicolumn{1}{c}{4.30} & \multicolumn{1}{c}{1306} &       & \multicolumn{1}{c}{Datasus/SIM} & \multicolumn{1}{c}{2000} \\
    \multicolumn{1}{l}{non-Amenable to Primary Care} & \multicolumn{1}{c}{2.78} & \multicolumn{1}{c}{1.33} & \multicolumn{1}{c}{0.00} & \multicolumn{1}{c}{11.59} & \multicolumn{1}{c}{5507} &       & \multicolumn{1}{c}{2.69} & \multicolumn{1}{c}{1.27} & \multicolumn{1}{c}{0.00} & \multicolumn{1}{c}{8.53} & \multicolumn{1}{c}{1306} & \multicolumn{1}{c}{2.88} & \multicolumn{1}{c}{1.32} & \multicolumn{1}{c}{0.00} & \multicolumn{1}{c}{10.44} & \multicolumn{1}{c}{1306} &       & \multicolumn{1}{c}{Datasus/SIM} & \multicolumn{1}{c}{2000} \\
    \multicolumn{1}{l}{Circulatory} & \multicolumn{1}{c}{0.72} & \multicolumn{1}{c}{0.60} & \multicolumn{1}{c}{0.00} & \multicolumn{1}{c}{8.60} & \multicolumn{1}{c}{5507} &       & \multicolumn{1}{c}{0.67} & \multicolumn{1}{c}{0.55} & \multicolumn{1}{c}{0.00} & \multicolumn{1}{c}{4.62} & \multicolumn{1}{c}{1306} & \multicolumn{1}{c}{0.74} & \multicolumn{1}{c}{0.64} & \multicolumn{1}{c}{0.00} & \multicolumn{1}{c}{8.60} & \multicolumn{1}{c}{1306} &       & \multicolumn{1}{c}{Datasus/SIM} & \multicolumn{1}{c}{2000} \\
    \multicolumn{1}{l}{Neoplasm} & \multicolumn{1}{c}{0.42} & \multicolumn{1}{c}{0.45} & \multicolumn{1}{c}{0.00} & \multicolumn{1}{c}{3.95} & \multicolumn{1}{c}{5507} &       & \multicolumn{1}{c}{0.40} & \multicolumn{1}{c}{0.45} & \multicolumn{1}{c}{0.00} & \multicolumn{1}{c}{3.95} & \multicolumn{1}{c}{1306} & \multicolumn{1}{c}{0.42} & \multicolumn{1}{c}{0.42} & \multicolumn{1}{c}{0.00} & \multicolumn{1}{c}{3.19} & \multicolumn{1}{c}{1306} &       & \multicolumn{1}{c}{Datasus/SIM} & \multicolumn{1}{c}{2000} \\
    \multicolumn{1}{l}{Respiratory} & \multicolumn{1}{c}{0.17} & \multicolumn{1}{c}{0.25} & \multicolumn{1}{c}{0.00} & \multicolumn{1}{c}{2.89} & \multicolumn{1}{c}{5507} &       & \multicolumn{1}{c}{0.15} & \multicolumn{1}{c}{0.23} & \multicolumn{1}{c}{0.00} & \multicolumn{1}{c}{1.91} & \multicolumn{1}{c}{1306} & \multicolumn{1}{c}{0.19} & \multicolumn{1}{c}{0.27} & \multicolumn{1}{c}{0.00} & \multicolumn{1}{c}{2.89} & \multicolumn{1}{c}{1306} &       & \multicolumn{1}{c}{Datasus/SIM} & \multicolumn{1}{c}{2000} \\
    \multicolumn{1}{l}{Endocrine} & \multicolumn{1}{c}{0.17} & \multicolumn{1}{c}{0.26} & \multicolumn{1}{c}{0.00} & \multicolumn{1}{c}{3.37} & \multicolumn{1}{c}{5507} &       & \multicolumn{1}{c}{0.16} & \multicolumn{1}{c}{0.24} & \multicolumn{1}{c}{0.00} & \multicolumn{1}{c}{2.10} & \multicolumn{1}{c}{1306} & \multicolumn{1}{c}{0.18} & \multicolumn{1}{c}{0.26} & \multicolumn{1}{c}{0.00} & \multicolumn{1}{c}{2.56} & \multicolumn{1}{c}{1306} &       & \multicolumn{1}{c}{Datasus/SIM} & \multicolumn{1}{c}{2000} \\
    \multicolumn{1}{l}{External} & \multicolumn{1}{c}{0.67} & \multicolumn{1}{c}{0.60} & \multicolumn{1}{c}{0.00} & \multicolumn{1}{c}{5.57} & \multicolumn{1}{c}{5507} &       & \multicolumn{1}{c}{0.68} & \multicolumn{1}{c}{0.60} & \multicolumn{1}{c}{0.00} & \multicolumn{1}{c}{5.57} & \multicolumn{1}{c}{1306} & \multicolumn{1}{c}{0.68} & \multicolumn{1}{c}{0.60} & \multicolumn{1}{c}{0.00} & \multicolumn{1}{c}{5.32} & \multicolumn{1}{c}{1306} &       & \multicolumn{1}{c}{Datasus/SIM} & \multicolumn{1}{c}{2000} \\
    \multicolumn{1}{l}{Nutritional} & \multicolumn{1}{c}{0.22} & \multicolumn{1}{c}{0.30} & \multicolumn{1}{c}{0.00} & \multicolumn{1}{c}{6.92} & \multicolumn{1}{c}{5507} &       & \multicolumn{1}{c}{0.18} & \multicolumn{1}{c}{0.25} & \multicolumn{1}{c}{0.00} & \multicolumn{1}{c}{2.20} & \multicolumn{1}{c}{1306} & \multicolumn{1}{c}{0.24} & \multicolumn{1}{c}{0.35} & \multicolumn{1}{c}{0.00} & \multicolumn{1}{c}{6.92} & \multicolumn{1}{c}{1306} &       & \multicolumn{1}{c}{Datasus/SIM} & \multicolumn{1}{c}{2000} \\
    \multicolumn{1}{l}{Ill-Defined} & \multicolumn{1}{c}{0.64} & \multicolumn{1}{c}{0.72} & \multicolumn{1}{c}{0.00} & \multicolumn{1}{c}{7.89} & \multicolumn{1}{c}{5507} &       & \multicolumn{1}{c}{0.65} & \multicolumn{1}{c}{0.73} & \multicolumn{1}{c}{0.00} & \multicolumn{1}{c}{5.63} & \multicolumn{1}{c}{1306} & \multicolumn{1}{c}{0.65} & \multicolumn{1}{c}{0.71} & \multicolumn{1}{c}{0.00} & \multicolumn{1}{c}{5.96} & \multicolumn{1}{c}{1306} &       & \multicolumn{1}{c}{Datasus/SIM} & \multicolumn{1}{c}{2000} \\
    \multicolumn{1}{l}{Other} & \multicolumn{1}{c}{0.31} & \multicolumn{1}{c}{0.35} & \multicolumn{1}{c}{0.00} & \multicolumn{1}{c}{3.54} & \multicolumn{1}{c}{5507} &       & \multicolumn{1}{c}{0.29} & \multicolumn{1}{c}{0.32} & \multicolumn{1}{c}{0.00} & \multicolumn{1}{c}{2.04} & \multicolumn{1}{c}{1306} & \multicolumn{1}{c}{0.34} & \multicolumn{1}{c}{0.37} & \multicolumn{1}{c}{0.00} & \multicolumn{1}{c}{3.54} & \multicolumn{1}{c}{1306} &       & \multicolumn{1}{c}{Datasus/SIM} & \multicolumn{1}{c}{2000} \\
    \multicolumn{1}{l}{Diabetes} & \multicolumn{1}{c}{0.09} & \multicolumn{1}{c}{0.18} & \multicolumn{1}{c}{0.00} & \multicolumn{1}{c}{1.59} & \multicolumn{1}{c}{5507} &       & \multicolumn{1}{c}{0.09} & \multicolumn{1}{c}{0.17} & \multicolumn{1}{c}{0.00} & \multicolumn{1}{c}{1.33} & \multicolumn{1}{c}{1306} & \multicolumn{1}{c}{0.10} & \multicolumn{1}{c}{0.18} & \multicolumn{1}{c}{0.00} & \multicolumn{1}{c}{1.41} & \multicolumn{1}{c}{1306} &       & \multicolumn{1}{c}{Datasus/SIM} & \multicolumn{1}{c}{2000} \\
    \multicolumn{1}{l}{Hypertension} & \multicolumn{1}{c}{0.07} & \multicolumn{1}{c}{0.15} & \multicolumn{1}{c}{0.00} & \multicolumn{1}{c}{2.43} & \multicolumn{1}{c}{5507} &       & \multicolumn{1}{c}{0.06} & \multicolumn{1}{c}{0.15} & \multicolumn{1}{c}{0.00} & \multicolumn{1}{c}{2.43} & \multicolumn{1}{c}{1306} & \multicolumn{1}{c}{0.07} & \multicolumn{1}{c}{0.17} & \multicolumn{1}{c}{0.00} & \multicolumn{1}{c}{2.15} & \multicolumn{1}{c}{1306} &       & \multicolumn{1}{c}{Datasus/SIM} & \multicolumn{1}{c}{2000} \\
          &       &       &       &       &       &       &       &       &       &       &       &       &       &       &       &       &       &       &  \\
    \multicolumn{1}{l}{\textbf{Controls}} &       &       &       &       &       &       &       &       &       &       &       &       &       &       &       &       &       &       &  \\
    \multicolumn{1}{l}{Population (1,000)} & \multicolumn{1}{c}{29.67} & \multicolumn{1}{c}{181.18} & \multicolumn{1}{c}{0.71} & \multicolumn{1}{c}{9968.49} & \multicolumn{1}{c}{5224} &       & \multicolumn{1}{c}{29.15} & \multicolumn{1}{c}{104.17} & \multicolumn{1}{c}{0.85} & \multicolumn{1}{c}{2302.83} & \multicolumn{1}{c}{1306} & \multicolumn{1}{c}{30.49} & \multicolumn{1}{c}{101.71} & \multicolumn{1}{c}{1.16} & \multicolumn{1}{c}{2139.125} & \multicolumn{1}{c}{1306} &       & \multicolumn{1}{c}{2000 Census} & \multicolumn{1}{c}{2000} \\
    \multicolumn{1}{l}{Life Expectancy} & \multicolumn{1}{c}{68.54} & \multicolumn{1}{c}{3.93} & \multicolumn{1}{c}{57.46} & \multicolumn{1}{c}{77.24} & \multicolumn{1}{c}{5224} &       & \multicolumn{1}{c}{67.85} & \multicolumn{1}{c}{3.96} & \multicolumn{1}{c}{57.65} & \multicolumn{1}{c}{77.18} & \multicolumn{1}{c}{1306} & \multicolumn{1}{c}{68.37} & \multicolumn{1}{c}{3.99} & \multicolumn{1}{c}{58.02} & \multicolumn{1}{c}{76.11} & \multicolumn{1}{c}{1306} &       & \multicolumn{1}{c}{2000 Census} & \multicolumn{1}{c}{2000} \\
    \multicolumn{1}{l}{Expected Years of Study} & \multicolumn{1}{c}{8.42} & \multicolumn{1}{c}{1.77} & \multicolumn{1}{c}{2.29} & \multicolumn{1}{c}{13.02} & \multicolumn{1}{c}{5224} &       & \multicolumn{1}{c}{8.06} & \multicolumn{1}{c}{1.82} & \multicolumn{1}{c}{2.65} & \multicolumn{1}{c}{13.02} & \multicolumn{1}{c}{1306} & \multicolumn{1}{c}{8.39} & \multicolumn{1}{c}{1.70} & \multicolumn{1}{c}{2.91} & \multicolumn{1}{c}{12.27} & \multicolumn{1}{c}{1306} &       & \multicolumn{1}{c}{2000 Census} & \multicolumn{1}{c}{2000} \\
    \multicolumn{1}{l}{Iliteracy Rate (above 18y old)} & \multicolumn{1}{c}{23.18} & \multicolumn{1}{c}{13.44} & \multicolumn{1}{c}{1.00} & \multicolumn{1}{c}{63.01} & \multicolumn{1}{c}{5224} &       & \multicolumn{1}{c}{25.10} & \multicolumn{1}{c}{13.40} & \multicolumn{1}{c}{2.03} & \multicolumn{1}{c}{58.71} & \multicolumn{1}{c}{1306} & \multicolumn{1}{c}{23.65} & \multicolumn{1}{c}{13.73} & \multicolumn{1}{c}{1.00} & \multicolumn{1}{c}{60.79} & \multicolumn{1}{c}{1306} &       & \multicolumn{1}{c}{2000 Census} & \multicolumn{1}{c}{2000} \\
    \multicolumn{1}{l}{Income per capita} & \multicolumn{1}{c}{345.06} & \multicolumn{1}{c}{192.94} & \multicolumn{1}{c}{62.65} & \multicolumn{1}{c}{1759.76} & \multicolumn{1}{c}{5224} &       & \multicolumn{1}{c}{315.10} & \multicolumn{1}{c}{183.59} & \multicolumn{1}{c}{64.91} & \multicolumn{1}{c}{1639.93} & \multicolumn{1}{c}{1306} & \multicolumn{1}{c}{343.36} & \multicolumn{1}{c}{201.14} & \multicolumn{1}{c}{76.32} & \multicolumn{1}{c}{1596.51} & \multicolumn{1}{c}{1306} &       & \multicolumn{1}{c}{2000 Census} & \multicolumn{1}{c}{2000} \\
    \multicolumn{1}{l}{Share of Population Below Poverty Line} & \multicolumn{1}{c}{0.40} & \multicolumn{1}{c}{0.23} & \multicolumn{1}{c}{0.01} & \multicolumn{1}{c}{0.91} & \multicolumn{1}{c}{5224} &       & \multicolumn{1}{c}{0.45} & \multicolumn{1}{c}{0.22} & \multicolumn{1}{c}{0.01} & \multicolumn{1}{c}{0.86} & \multicolumn{1}{c}{1306} & \multicolumn{1}{c}{0.40} & \multicolumn{1}{c}{0.24} & \multicolumn{1}{c}{0.01} & \multicolumn{1}{c}{0.866} & \multicolumn{1}{c}{1306} &       & \multicolumn{1}{c}{2000 Census} & \multicolumn{1}{c}{2000} \\
    \multicolumn{1}{l}{Gini Coefficient} & \multicolumn{1}{c}{0.55} & \multicolumn{1}{c}{0.07} & \multicolumn{1}{c}{0.30} & \multicolumn{1}{c}{0.87} & \multicolumn{1}{c}{5224} &       & \multicolumn{1}{c}{0.55} & \multicolumn{1}{c}{0.07} & \multicolumn{1}{c}{0.31} & \multicolumn{1}{c}{0.81} & \multicolumn{1}{c}{1306} & \multicolumn{1}{c}{0.54} & \multicolumn{1}{c}{0.07} & \multicolumn{1}{c}{0.34} & \multicolumn{1}{c}{0.8} & \multicolumn{1}{c}{1306} &       & \multicolumn{1}{c}{2000 Census} & \multicolumn{1}{c}{2000} \\
    \multicolumn{1}{l}{Access to Sewage Network} & \multicolumn{1}{c}{0.26} & \multicolumn{1}{c}{0.31} & \multicolumn{1}{c}{0.00} & \multicolumn{1}{c}{0.99} & \multicolumn{1}{c}{5224} &       & \multicolumn{1}{c}{0.17} & \multicolumn{1}{c}{0.25} & \multicolumn{1}{c}{0.00} & \multicolumn{1}{c}{0.99} & \multicolumn{1}{c}{1306} & \multicolumn{1}{c}{0.32} & \multicolumn{1}{c}{0.33} & \multicolumn{1}{c}{0.00} & \multicolumn{1}{c}{0.981} & \multicolumn{1}{c}{1306} &       & \multicolumn{1}{c}{2000 Census} & \multicolumn{1}{c}{2000} \\
    \multicolumn{1}{l}{Access to Garbage Collection Service} & \multicolumn{1}{c}{0.55} & \multicolumn{1}{c}{0.27} & \multicolumn{1}{c}{0.00} & \multicolumn{1}{c}{1.00} & \multicolumn{1}{c}{5224} &       & \multicolumn{1}{c}{0.50} & \multicolumn{1}{c}{0.26} & \multicolumn{1}{c}{0.00} & \multicolumn{1}{c}{1.00} & \multicolumn{1}{c}{1306} & \multicolumn{1}{c}{0.57} & \multicolumn{1}{c}{0.28} & \multicolumn{1}{c}{0.00} & \multicolumn{1}{c}{0.999} & \multicolumn{1}{c}{1306} &       & \multicolumn{1}{c}{2000 Census} & \multicolumn{1}{c}{2000} \\
    \multicolumn{1}{l}{Access to Water Network} & \multicolumn{1}{c}{0.59} & \multicolumn{1}{c}{0.24} & \multicolumn{1}{c}{0.00} & \multicolumn{1}{c}{1.00} & \multicolumn{1}{c}{5224} &       & \multicolumn{1}{c}{0.56} & \multicolumn{1}{c}{0.23} & \multicolumn{1}{c}{0.00} & \multicolumn{1}{c}{1.00} & \multicolumn{1}{c}{1306} & \multicolumn{1}{c}{0.61} & \multicolumn{1}{c}{0.24} & \multicolumn{1}{c}{0.00} & \multicolumn{1}{c}{1} & \multicolumn{1}{c}{1306} &       & \multicolumn{1}{c}{2000 Census} & \multicolumn{1}{c}{2000} \\
    \multicolumn{1}{l}{Access to Electricity} & \multicolumn{1}{c}{0.88} & \multicolumn{1}{c}{0.16} & \multicolumn{1}{c}{0.08} & \multicolumn{1}{c}{1.00} & \multicolumn{1}{c}{5224} &       & \multicolumn{1}{c}{0.85} & \multicolumn{1}{c}{0.18} & \multicolumn{1}{c}{0.08} & \multicolumn{1}{c}{1.00} & \multicolumn{1}{c}{1306} & \multicolumn{1}{c}{0.88} & \multicolumn{1}{c}{0.16} & \multicolumn{1}{c}{0.19} & \multicolumn{1}{c}{1} & \multicolumn{1}{c}{1306} &       & \multicolumn{1}{c}{2000 Census} & \multicolumn{1}{c}{2000} \\
    \multicolumn{1}{l}{Urbanization Rate} & \multicolumn{1}{c}{0.61} & \multicolumn{1}{c}{0.23} & \multicolumn{1}{c}{0.00} & \multicolumn{1}{c}{1.00} & \multicolumn{1}{c}{5224} &       & \multicolumn{1}{c}{0.58} & \multicolumn{1}{c}{0.22} & \multicolumn{1}{c}{0.03} & \multicolumn{1}{c}{1.00} & \multicolumn{1}{c}{1306} & \multicolumn{1}{c}{0.62} & \multicolumn{1}{c}{0.23} & \multicolumn{1}{c}{0.05} & \multicolumn{1}{c}{1} & \multicolumn{1}{c}{1306} &       & \multicolumn{1}{c}{2000 Census} & \multicolumn{1}{c}{2000} \\
    \multicolumn{1}{l}{Average Neighbors Spending Health Spending per capita} & \multicolumn{1}{c}{208.47} & \multicolumn{1}{c}{124.00} & \multicolumn{1}{c}{1.74} & \multicolumn{1}{c}{3298.40} & \multicolumn{1}{c}{5222} &       & \multicolumn{1}{c}{181.06} & \multicolumn{1}{c}{105.13} & \multicolumn{1}{c}{42.02} & \multicolumn{1}{c}{2287.30} & \multicolumn{1}{c}{1306} & \multicolumn{1}{c}{228.64} & \multicolumn{1}{c}{148.82} & \multicolumn{1}{c}{40.65} & \multicolumn{1}{c}{3298.403} & \multicolumn{1}{c}{1305} &       & \multicolumn{1}{c}{Finbra} & \multicolumn{1}{c}{2000} \\
    \multicolumn{1}{l}{Noncompliance with Fiscal Responsibility Law (HR Spending)} & \multicolumn{1}{c}{0.03} & \multicolumn{1}{c}{0.17} & \multicolumn{1}{c}{0.00} & \multicolumn{1}{c}{1.00} & \multicolumn{1}{c}{5099} &       & \multicolumn{1}{c}{0.03} & \multicolumn{1}{c}{0.18} & \multicolumn{1}{c}{0.00} & \multicolumn{1}{c}{1.00} & \multicolumn{1}{c}{1258} & \multicolumn{1}{c}{0.03} & \multicolumn{1}{c}{0.17} & \multicolumn{1}{c}{0.00} & \multicolumn{1}{c}{1} & \multicolumn{1}{c}{1277} &       & \multicolumn{1}{c}{Finbra} & \multicolumn{1}{c}{2000} \\
          &       &       &       &       &       &       &       &       &       &       &       &       &       &       &       &       &       &       &  \\
    \midrule
    \midrule
          &       &       &       &       &       &       &       &       &       &       &       &       &       &       &       &       &       &       &  \\
    \end{tabular}%
    



\end{threeparttable}
}
\end{center}
\end{footnotesize}
\end{table}
\end{sidewaystable}

\pagebreak
\section{Dynamic Fiscal Reactions}\label{app:fiscal}

\begin{figure}[h!]
    \begin{center}
    \caption{Fiscal Reactions}\label{fig:b1}
    \begin{subfigure}{0.49\textwidth}
        \caption{\scriptsize Total Revenue}\label{fig:b1a}
        \centering
        \includegraphics[width=\textwidth]{plots/finbra_reccorr_pcapita_dist_ec29_baseline_dist_ec29_baseline_B1.pdf}
    \end{subfigure}
    \begin{subfigure}{0.49\textwidth}
        \centering
        \caption{\scriptsize Total Public Spending}\label{fig:b1b}
        \includegraphics[width=\textwidth]{plots/finbra_desp_o_pcapita_dist_ec29_baseline_dist_ec29_baseline_B1.pdf}
    \end{subfigure}
    
    \end{center}
    \scriptsize{Notes: The number of observations is 64224. DiD Estimates from Equation \ref{eq:2}. Independent variable is the distance to the EC/29 target in p.p. Square dots represent the baseline model with municipality and state-year fixed effects. Round dots represent fully saturated specification (Column 4 in regression Tables). Lines represent 95\% confidence intervals. Arrows, when present, indicate confidence intervals out of the plot bounds. Standard errors are clustered in the municipality level.}
    
\end{figure}
\input{figures/Figure_B2}

\clearpage
\pagebreak

\section{Effects on Infant Mortality Rate - Extended}\label{app:imr}


\begin{table}[h!]
\begin{footnotesize}
\begin{center}
\scalebox{0.9}{
\begin{threeparttable}[b]

  \centering
  \caption{Infant Mortality Rates - Extended}
     \begin{tabular}{rrccccccc}
          &       &       &       &       &       &       &       &  \\
          &       &       &       &       &       &       &       &  \\
    \midrule
    \midrule
          &       & \multicolumn{4}{c}{With IMR Ill-defined Trend} &       & \multicolumn{2}{c}{Without IMR Trends} \\
\cmidrule{3-6}\cmidrule{8-9}          &       & (1)   & (2)   & (3)   & (4)   &       & (5)   & (6) \\
    \midrule
    \multicolumn{1}{p{15.145em}}{\textbf{A. Infant Mortality Rate}} &       &       &       &       &       &       &       &  \\
    \multicolumn{1}{p{15.145em}}{Total} &       & -5.015 & -3.772 & -3.831 & -3.889 &       & -2.062 & -2.12 \\
          &       & (3.435) & (2.853) & (2.836) & (2.828) &       & (3.128) & (3.121) \\
    \multicolumn{1}{p{15.145em}}{Amenable to Primary Care} &       & -0.361 & -0.866 & -0.893 & -0.905 &       & -0.841 & -0.853 \\
          &       & (0.603) & (0.553) & (0.553) & (0.554) &       & (0.553) & (0.553) \\
    \multicolumn{1}{p{15.145em}}{Non-Amenable to Primary Care} &       & -4.653 & -2.907 & -2.939 & -2.984 &       & -1.222 & -1.267 \\
          &       & (3.245) & (2.645) & (2.632) & (2.624) &       & (2.924) & (2.918) \\
          &       &       &       &       &       &       &       &  \\
    \midrule
    \multicolumn{1}{p{15.145em}}{\textbf{B. By timing}} &       &       &       &       &       &       &       &  \\
    \multicolumn{1}{p{15.145em}}{Fetal} &       & -0.008 & -0.007 & -0.008 & -0.008 &       & -0.008 & -0.008 \\
          &       & (0.008) & (0.008) & (0.008) & (0.008) &       & (0.008) & (0.008) \\
    \multicolumn{1}{p{15.145em}}{Within 24h} &       & -2.275* & -2.083** & -2.07** & -2.071** &       & -1.896* & -1.898** \\
          &       & (1.225) & (0.98) & (0.979) & (0.976) &       & (0.969) & (0.967) \\
    \multicolumn{1}{p{15.145em}}{1 to 27 days} &       & -4.228* & -2.883 & -2.911 & -2.922 &       & -2.239 & -2.249 \\
          &       & (2.555) & (2.064) & (2.052) & (2.046) &       & (2.073) & (2.067) \\
    \multicolumn{1}{p{15.145em}}{27 days to 1 year} &       & -0.787 & -0.89 & -0.92 & -0.967 &       & 0.176 & 0.129 \\
          &       & (1.435) & (1.248) & (1.246) & (1.243) &       & (1.561) & (1.559) \\
          &       &       &       &       &       &       &       &  \\
    \midrule
    \multicolumn{1}{p{15.145em}}{\textbf{C. By Cause of Death}} &       &       &       &       &       &       &       &  \\
    \multicolumn{1}{p{15.145em}}{Infectious} &       & -0.374 & -0.811 & -0.82 & -0.831 &       & -0.794 & -0.805 \\
          &       & (0.567) & (0.535) & (0.535) & (0.534) &       & (0.538) & (0.538) \\
    \multicolumn{1}{p{15.145em}}{Respiratory} &       & -0.494 & -0.507 & -0.511 & -0.517 &       & -0.465 & -0.471 \\
          &       & (0.474) & (0.411) & (0.409) & (0.409) &       & (0.406) & (0.406) \\
    \multicolumn{1}{p{15.145em}}{Perinatal} &       & -5.349** & -3.648* & -3.69* & -3.707* &       & -3.159 & -3.176 \\
          &       & (2.571) & (2.015) & (2.007) & (2.002) &       & (1.962) & (1.957) \\
    \multicolumn{1}{p{15.145em}}{Congenital} &       & -0.235 & -0.169 & -0.16 & -0.157 &       & -0.15 & -0.147 \\
          &       & (0.463) & (0.436) & (0.434) & (0.434) &       & (0.435) & (0.434) \\
    \multicolumn{1}{p{15.145em}}{External} &       & 0.024 & -0.049 & -0.037 & -0.034 &       & -0.036 & -0.033 \\
          &       & (0.183) & (0.165) & (0.165) & (0.166) &       & (0.165) & (0.166) \\
    \multicolumn{1}{p{15.145em}}{Nutritional} &       & -0.204 & -0.328 & -0.33 & -0.343 &       & -0.298 & -0.311 \\
          &       & (0.246) & (0.231) & (0.232) & (0.232) &       & (0.225) & (0.226) \\
    \multicolumn{1}{p{15.145em}}{Other} &       & -0.183 & -0.123 & -0.132 & -0.139 &       & -0.138 & -0.144 \\
          &       & (0.201) & (0.199) & (0.198) & (0.198) &       & (0.198) & (0.198) \\
    \multicolumn{1}{p{15.145em}}{Ill-Defined} &       & 1.8** & 1.862** & 1.849** & 1.84** &       & 2.977** & 2.968** \\
          &       & (0.849) & (0.776) & (0.779) & (0.779) &       & (1.379) & (1.38) \\
          &       &       &       &       &       &       &       &  \\
    \bottomrule
    \bottomrule
    \end{tabular}%
    
    
    \begin{tablenotes}
  \scriptsize{\underline{Notes}: The number of observations is 64482. DiD Estimates from Equation \ref{eq:1}. Independent variable is the distance to the EC/29 target in p.p. Column 1 presents the baseline model with municipality and state-year fixed effects. Column 2 adds baseline socioeconomic controls from the Census interacted with time. Column 3 adds controls for GDP per capita and \emph{Bolsa Familia} transfers per capita. Column 4 adds fiscal controls. Column 5 removes from the specification in Column 3 the trend of baseline ill-defined infant mortality. Column 6 removes this trend for the specification in Column 4. Covariates omitted. Standard errors in brackets are clustered in the municipality level. ∗p < 0.10, ∗ ∗ p < 0.05, ∗ ∗ ∗p < 0.01}
  \end{tablenotes}
    
    
  \label{table:app_imr}%

\end{threeparttable}
}
\end{center}
\end{footnotesize}
\end{table}

\pagebreak

\section{Infant Mortality Rates Elasticity}\label{app:elasticity}

\begin{table}[h!]
\begin{footnotesize}
\begin{center}
\scalebox{0.8}{
\begin{threeparttable}[b]

  \centering
  \caption{\emph{Back of the Envelope} Infant Mortality Rates Elasticity}
    \begin{tabular}{rrrrrrrrrrr}
          &       &       &       &       &       &       &       &       &       &  \\
    \midrule
    \midrule
          &       & \multicolumn{4}{p{14.28em}}{Health and Sanitation Spending (Finbra)} &       & \multicolumn{4}{p{14.28em}}{Health Spending (Siops)} \\
\cmidrule{3-6}\cmidrule{8-11}          &       & \multicolumn{1}{p{3.57em}}{(1)} & \multicolumn{1}{p{3.57em}}{(2)} & \multicolumn{1}{p{3.57em}}{(3)} & \multicolumn{1}{p{3.57em}}{(4)} &       & \multicolumn{1}{p{3.57em}}{(1)} & \multicolumn{1}{p{3.57em}}{(2)} & \multicolumn{1}{p{3.57em}}{(3)} & \multicolumn{1}{p{3.57em}}{(4)} \\
\cmidrule{3-11}          &       &       &       &       &       &       &       &       &       &  \\
    \multicolumn{1}{p{15.145em}}{\textbf{Infant Mortality Rate}} &       &       &       &       &       &       &       &       &       &  \\
    \multicolumn{1}{p{15.145em}}{Total} &       & \multicolumn{1}{c}{-0.156} & \multicolumn{1}{c}{-0.116} & \multicolumn{1}{c}{-0.115} & \multicolumn{1}{c}{-0.119} &       & \multicolumn{1}{c}{-0.138} & \multicolumn{1}{c}{-0.059} & \multicolumn{1}{c}{-0.060} & \multicolumn{1}{c}{-0.061} \\
    \multicolumn{1}{p{15.145em}}{Amenable to Primary Care} &       & \multicolumn{1}{c}{-0.123} & \multicolumn{1}{c}{-0.292} & \multicolumn{1}{c}{-0.294} & \multicolumn{1}{c}{-0.305} &       & \multicolumn{1}{c}{-0.109} & \multicolumn{1}{c}{-0.150} & \multicolumn{1}{c}{-0.154} & \multicolumn{1}{c}{-0.157} \\
    \multicolumn{1}{p{15.145em}}{Non-Amenable to Primary Care} &       & \multicolumn{1}{c}{-0.159} & \multicolumn{1}{c}{-0.098} & \multicolumn{1}{c}{-0.097} & \multicolumn{1}{c}{-0.101} &       & \multicolumn{1}{c}{-0.141} & \multicolumn{1}{c}{-0.050} & \multicolumn{1}{c}{-0.051} & \multicolumn{1}{c}{-0.052} \\
          &       &       &       &       &       &       &       &       &       &  \\
    \multicolumn{1}{p{15.145em}}{\textbf{By timing}} &       &       &       &       &       &       &       &       &       &  \\
    \multicolumn{1}{p{15.145em}}{Fetal} &       & \multicolumn{1}{c}{-1.912} & \multicolumn{1}{c}{-1.650} & \multicolumn{1}{c}{-1.842} & \multicolumn{1}{c}{-1.887} &       & \multicolumn{1}{c}{-1.696} & \multicolumn{1}{c}{-0.847} & \multicolumn{1}{c}{-0.967} & \multicolumn{1}{c}{-0.971} \\
    \multicolumn{1}{p{15.145em}}{Within 24h} &       & \multicolumn{1}{c}{-0.294} & \multicolumn{1}{c}{-0.265} & \multicolumn{1}{c}{-0.257} & \multicolumn{1}{c}{-0.264} &       & \multicolumn{1}{c}{-0.261} & \multicolumn{1}{c}{-0.136} & \multicolumn{1}{c}{-0.135} & \multicolumn{1}{c}{-0.136} \\
    \multicolumn{1}{p{15.145em}}{1 to 27 days} &       & \multicolumn{1}{c}{-0.221} & \multicolumn{1}{c}{-0.148} & \multicolumn{1}{c}{-0.146} & \multicolumn{1}{c}{-0.151} &       & \multicolumn{1}{c}{-0.196} & \multicolumn{1}{c}{-0.076} & \multicolumn{1}{c}{-0.077} & \multicolumn{1}{c}{-0.077} \\
    \multicolumn{1}{p{15.145em}}{27 days to 1 year} &       & \multicolumn{1}{c}{-0.060} & \multicolumn{1}{c}{-0.067} & \multicolumn{1}{c}{-0.068} & \multicolumn{1}{c}{-0.073} &       & \multicolumn{1}{c}{-0.054} & \multicolumn{1}{c}{-0.035} & \multicolumn{1}{c}{-0.036} & \multicolumn{1}{c}{-0.038} \\
          &       &       &       &       &       &       &       &       &       &  \\
    \multicolumn{1}{p{15.145em}}{\textbf{By Cause of Death}} &       &       &       &       &       &       &       &       &       &  \\
    \multicolumn{1}{p{15.145em}}{Infectious} &       & \multicolumn{1}{c}{-0.134} & \multicolumn{1}{c}{-0.287} & \multicolumn{1}{c}{-0.283} & \multicolumn{1}{c}{-0.294} &       & \multicolumn{1}{c}{-0.119} & \multicolumn{1}{c}{-0.147} & \multicolumn{1}{c}{-0.149} & \multicolumn{1}{c}{-0.151} \\
    \multicolumn{1}{p{15.145em}}{Respiratory} &       & \multicolumn{1}{c}{-0.234} & \multicolumn{1}{c}{-0.237} & \multicolumn{1}{c}{-0.233} & \multicolumn{1}{c}{-0.241} &       & \multicolumn{1}{c}{-0.207} & \multicolumn{1}{c}{-0.122} & \multicolumn{1}{c}{-0.122} & \multicolumn{1}{c}{-0.124} \\
    \multicolumn{1}{p{15.145em}}{Perinatal} &       & \multicolumn{1}{c}{-0.347} & \multicolumn{1}{c}{-0.234} & \multicolumn{1}{c}{-0.231} & \multicolumn{1}{c}{-0.238} &       & \multicolumn{1}{c}{-0.308} & \multicolumn{1}{c}{-0.120} & \multicolumn{1}{c}{-0.121} & \multicolumn{1}{c}{-0.122} \\
    \multicolumn{1}{p{15.145em}}{Congenital} &       & \multicolumn{1}{c}{-0.079} & \multicolumn{1}{c}{-0.056} & \multicolumn{1}{c}{-0.052} & \multicolumn{1}{c}{-0.052} &       & \multicolumn{1}{c}{-0.070} & \multicolumn{1}{c}{-0.029} & \multicolumn{1}{c}{-0.027} & \multicolumn{1}{c}{-0.027} \\
    \multicolumn{1}{p{15.145em}}{External} &       & \multicolumn{1}{c}{0.047} & \multicolumn{1}{c}{-0.095} & \multicolumn{1}{c}{-0.070} & \multicolumn{1}{c}{-0.066} &       & \multicolumn{1}{c}{0.042} & \multicolumn{1}{c}{-0.049} & \multicolumn{1}{c}{-0.037} & \multicolumn{1}{c}{-0.034} \\
    \multicolumn{1}{p{15.145em}}{Nutritional} &       & \multicolumn{1}{c}{-0.243} & \multicolumn{1}{c}{-0.386} & \multicolumn{1}{c}{-0.379} & \multicolumn{1}{c}{-0.404} &       & \multicolumn{1}{c}{-0.216} & \multicolumn{1}{c}{-0.198} & \multicolumn{1}{c}{-0.199} & \multicolumn{1}{c}{-0.208} \\
    \multicolumn{1}{p{15.145em}}{Other} &       & \multicolumn{1}{c}{-0.151} & \multicolumn{1}{c}{-0.100} & \multicolumn{1}{c}{-0.105} & \multicolumn{1}{c}{-0.113} &       & \multicolumn{1}{c}{-0.134} & \multicolumn{1}{c}{-0.051} & \multicolumn{1}{c}{-0.055} & \multicolumn{1}{c}{-0.058} \\
    \multicolumn{1}{p{15.145em}}{Ill-Defined} &       & \multicolumn{1}{c}{0.284} & \multicolumn{1}{c}{0.289} & \multicolumn{1}{c}{0.281} & \multicolumn{1}{c}{0.286} &       & \multicolumn{1}{c}{0.252} & \multicolumn{1}{c}{0.149} & \multicolumn{1}{c}{0.147} & \multicolumn{1}{c}{0.147} \\
          &       &       &       &       &       &       &       &       &       &  \\
    \bottomrule
    \bottomrule
    \end{tabular}%
    
    
        \begin{tablenotes}
  \scriptsize{\underline{Notes}: Elasticity of Infant Mortality Rates estimated using $ImrE_s = \frac{\% \, \, \textrm{ Change in Infant Mortality Rates}}{\% \, \, \textrm{ Change in Health Spending}}$. Percent changes for IMRs calculated using effects presented in Table \ref{table:imr} and baseline IMRs. Percent changes for Finbra and SIOPS spending calculated using effects presented in Table \ref{table:fiscal} and baseline spending. See Table \ref{app:stats} for baseline statistics. The different specifications columns uses the estimates from the correspondent column in the regression tables}
  \end{tablenotes}
    
  \label{table:elasticity}%

\end{threeparttable}
}
\end{center}
\end{footnotesize}
\end{table}



\end{document} 

